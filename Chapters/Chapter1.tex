% Chapter 1

\chapter{Introducción general} % Main chapter title

\label{Chapter1} % For referencing the chapter elsewhere, use \ref{Chapter1} 
\label{IntroGeneral}

En este capítulo se hace una breve introducción a la necesidad que condujo al
desarrollo del trabajo. Se presenta el concepto de cultivo vertical, y el estado del arte de su integración con la tecnología. Asimismo, se explica el objetivo y los alcances del trabajo.

%----------------------------------------------------------------------------------------

% Define some commands to keep the formatting separated from the content 
\newcommand{\keyword}[1]{\textbf{#1}}
\newcommand{\tabhead}[1]{\textbf{#1}}
\newcommand{\code}[1]{\texttt{#1}}
\newcommand{\file}[1]{\texttt{\bfseries#1}}
\newcommand{\option}[1]{\texttt{\itshape#1}}
\newcommand{\grados}{$^{\circ}$}

%----------------------------------------------------------------------------------------

%\section{Introducción}

%----------------------------------------------------------------------------------------
\section{Introducción a los cultivos verticales}

%\LaTeX{} no es \textsc{WYSIWYG} (What You See is What You Get), a diferencia de los procesadores de texto como Microsoft Word o Pages de Apple o incluso LibreOffice en el mundo open-source. En lugar de ello, un documento escrito para \LaTeX{} es en realidad un archivo de texto simple o llano que \emph{no contiene formato} . Nosotros le decimos a \LaTeX{} cómo deseamos que se aplique el formato en el documento final escribiendo comandos simples entre el texto, por ejemplo, si quiero usar texto en itálicas para dar énfasis, escribo \verb|\it{texto}| y pongo el texto que quiero en itálicas entre medio de las llaves. Esto significa que \LaTeX{} es un lenguaje del tipo \enquote{mark-up}, muy parecido a HTML.

Con una población mundial que supera los 8.000 millones de personas y continúa en crecimiento, la agricultura enfrenta el desafío de volverse más eficiente y sostenible. En este contexto, los cultivos verticales emergen como una solución innovadora que optimiza el uso del espacio y los recursos, especialmente en entornos urbanos, donde el suelo es limitado y costoso.

Este tipo de agricultura emplea técnicas como la hidroponía, que permite un uso preciso del agua, y la aeroponía, que maximiza la oxigenación de las raíces. Además, las granjas verticales aprovechan áreas infrautilizadas, como edificios abandonados o naves industriales, permitiendo el cultivo de alimentos en zonas donde la agricultura tradicional resulta difícil de implementar. De este modo, se logra una mayor densidad de cultivo y se contribuye a la sostenibilidad, reduciendo el uso de pesticidas y fertilizantes.

Aunque los cultivos verticales requieren altos niveles de tecnología y conllevan ciertos costos energéticos, su capacidad para optimizar recursos, reducir la huella de carbono y fomentar la producción local y el autoconsumo los posiciona como una alternativa clave para la agricultura del futuro.

Este método, también conocido como agricultura urbana, permite el crecimiento de plantas en interiores sin necesidad de luz solar directa, gracias a la aplicación de tecnologías agrícolas avanzadas. Estas optimizan las condiciones de cultivo, facilitando la propagación de plantas jóvenes y la producción de alimentos más saludables sin el uso de pesticidas.

Además, los cultivos verticales maximizan la producción al emplear sistemas de iluminación especializados de nueva generación, diseñados para estructuras multicapa. Gracias a esta tecnología, es posible obtener mayores rendimientos en un espacio reducido, consolidando a la agricultura vertical como una solución sostenible y eficiente para la producción de alimentos en entornos urbanos y con recursos limitados.

\subsection{Tipos de cultivos verticales}

%https://agroquivir.com/agricultura-vertical-como-funciona-y-tipos/

%Si sos nuevo en \LaTeX{}, hay un muy buen libro electrónico - disponible gratuitamente en Internet como un archivo PDF - llamado, \enquote{A (not so short) Introduction to \LaTeX{}}. El título del libro es generalmente acortado a simplemente \emph{lshort}. Puede descargar la versión más reciente en inglés (ya que se actualiza de vez en cuando) desde aquí:
%\url{http://www.ctan.org/tex-archive/info/lshort/english/lshort.pdf}

%Se puede encontrar la versión en español en la lista en esta página: \url{http://www.ctan.org/tex-archive/info/lshort/}

Existen distintos tipos de agricultura vertical, que optimizan el crecimiento de las plantas sin grandes extensiones de suelo y emplean estructuras especializadas para maximizar la producción.

\subsubsection{Hidroponía}
La hidroponía es un método de cultivo sin suelo que suministra nutrientes a las plantas a través de una solución acuosa, con raíces en un medio inerte como lana de roca o perlita. Es un sistema eficiente en el uso de agua, permite un control preciso de nutrientes y favorece una mayor densidad de cultivo y un crecimiento más rápido que los métodos tradicionales.
\subsubsection{Aeroponía}
La aeroponía es un método de cultivo en el que las raíces de las plantas quedan suspendidas en el aire y se rocían con una solución nutritiva. Maximiza la oxigenación, acelera el crecimiento y optimiza el uso de nutrientes, aunque requiere un control preciso para evitar la desecación de las raíces.
\subsubsection{Acuaponía}
La acuaponía combina la cría de peces con la hidroponía, utilizando los desechos de los peces como fertilizante para las plantas, mientras estas purifican el agua. Es un sistema sostenible y eficiente que minimiza el desperdicio, promueve la biodiversidad y es viable en entornos urbanos y rurales.

Acá tiene un ejemplo de una ``subsubsección'' que es el cuarto nivel de ordenamiento del texto, después de capítulo, sección y subsección.  Como se puede ver, las subsubsecciones no van numeradas en el cuerpo del documento ni en el índice.  El formato está definido por la plantilla y no debe ser modificado.

\subsection{Sistemas de agricultura vertical más conocidos}
Los sistemas de agricultura vertical varían según el tipo de cultivo y la técnica más adecuada para cada necesidad.

\subsubsection{Sistemas de torre}
Las torres verticales permiten el cultivo de plantas en estructuras cilíndricas apiladas, maximizando el uso del espacio y facilitando tanto la recolección como el mantenimiento.
Este sistema es ideal para cultivos de hoja verde y hierbas, y se integra fácilmente en espacios urbanos como balcones y terrazas.

\subsubsection{Sistemas en rack o estanterías}
Estos sistemas emplean estanterías apiladas donde las plantas crecen en bandejas o contenedores, siendo ideales para invernaderos y ambientes controlados. Facilitan el riego, la recolección y el monitoreo, optimizando el uso del espacio.
Los sistemas en rack o estanterías son especialmente útiles para cultivos que requieren diferentes niveles de luz y humedad.

\subsubsection{Sistemas montados en pared}
En estos sistemas, las plantas crecen en módulos montados en paredes, aprovechando superficies verticales en entornos urbanos. Su integración en la arquitectura no solo optimiza el espacio, sino que también aporta beneficios estéticos y funcionales.
Los jardines verticales pueden mejorar la calidad del aire y contribuir al aislamiento térmico de los edificios.

\subsubsection{Sistemas de bastidor en A}
El bastidor en forma de “A” proporciona una estructura estable y accesible para el cultivo vertical, facilitando el acceso a la luz y el riego.
Este diseño es particularmente útil para cultivos que necesitan un soporte adicional, como tomates y pepinos, y puede ser utilizado tanto en interiores como en exteriores.

\subsubsection{Sistemas de contenedores}
Estos sistemas utilizan contenedores apilables que pueden moverse y configurarse según las necesidades, adaptándose a diversos entornos y tipos de cultivo.
Los contenedores son ideales para cultivos modulares y pueden ser personalizados para optimizar el uso del espacio y los recursos.
%Si estás escribiendo un documento con mucho contenido matemático, entonces es posible que desees leer el documento de la AMS (American Mathematical Society) llamado, \enquote{A Short Math Guide for \LaTeX{}}. Se puede encontrar en línea en el siguiente link: \url{http://www.ams.org/tex/amslatex.html} en la sección \enquote{Additional Documentation} hacia la parte inferior de la página.


%----------------------------------------------------------------------------------------

\section{Motivación}

Este trabajo surge como un desarrollo de interés personal, impulsado por el deseo de contribuir positivamente al desarrollo sustentable y la optimización de los recursos naturales en la producción de alimentos. En la actualidad, el crecimiento de la población mundial y el aumento en la demanda de productos agrícolas presentan grandes desafíos ambientales y económicos.

La producción tradicional de alimentos enfrenta dificultades debido a la explotación intensiva de suelos, el uso excesivo de fertilizantes y pesticidas, y el desperdicio de recursos esenciales como el agua. Con el fin de maximizar sus ganancias, muchas empresas recurren a métodos poco sostenibles, los cuales generan un impacto ambiental negativo, afectando la biodiversidad, la calidad del suelo y el bienestar de las comunidades locales que dependen de estas fuentes de producción.

Uno de los problemas más críticos es el desperdicio de agua en la agricultura tradicional. Se estima que casi el 90 \% del agua utilizada en el riego se pierde por evaporación, drenaje o absorción ineficiente. En un contexto donde la crisis hídrica es una amenaza global, desarrollar sistemas más eficientes y sostenibles se vuelve una necesidad urgente.

Por otro lado, existe una creciente preocupación entre los consumidores sobre la calidad y procedencia de los alimentos. Cada vez más personas optan por un estilo de vida más saludable, adoptando hábitos de consumo responsables, como el vegetarianismo, el veganismo y el autocultivo. Sin embargo, muchos de estos consumidores no cuentan con espacios adecuados ni conocimientos técnicos para producir sus propios alimentos, lo que genera una gran oportunidad para soluciones tecnológicas que faciliten la agricultura doméstica o urbana.

A partir de estas problemáticas, nace la motivación para desarrollar un sistema hidropónico inteligente, que no solo optimice el uso del agua y los nutrientes, sino que también permita una gestión eficiente del cultivo con mínima intervención del usuario. Este trabajo ofrece una alternativa viable y escalable para la producción de alimentos frescos, reduciendo la dependencia de sistemas agrícolas tradicionales y promoviendo un modelo más sustentable para el futuro.


\section{Estado del arte}
Durante la etapa de investigación, se llevó a cabo una búsqueda de productos comerciales tanto en el mercado local como en el internacional. Se identificaron diversas soluciones con características similares al sistema desarrollado.

A continuación, se describen los productos identificados, los cuales presentan variaciones en cuanto a la tecnología utilizada.

\subsubsection{NIDO PRO}
NIDO PRO es un sistema hidropónico inteligente diseñado para cultivos verticales. Permite el control y la gestión de la solución nutritiva, estableciendo valores de pH y conductividad eléctrica (CE) desde una aplicación móvil. Sus algoritmos realizan comprobaciones automáticas diarias para garantizar la estabilidad de la solución nutritiva. Además, puede gestionar hasta cuatro ranuras para fertilizantes líquidos y ajustes de pH.
%https://www.agrointec.com/producto/nidopro-sistema-hidroponico-inteligente-cultivo-vertical/?utm_source=chatgpt.com

\subsubsection{Xiaomi Mi Flower Sensor}
Xiaomi Mi Flower Care Plant Sensor permite monitorear variables ambientales como luz, humedad, nutrientes y temperatura del sustrato. Estos sensores se conectan a aplicaciones móviles, facilitando el seguimiento y control de las condiciones del cultivo.
%https://elpais.com/tecnologia/tu-tecnologia/2024-12-07/dispositivos-tecnologicos-para-que-tus-plantas-sean-la-envidia-de-cualquiera.html?utm_source=chatgpt.com

\subsubsection{Kit de Riego Hydro de Konyks}
El Kit de Riego Hydro de Konyks ofrece control remoto del riego, permitiendo regular el caudal y programar horarios mediante una aplicación móvil. Esto asegura un suministro de agua adecuado y constante para los cultivos hidropónicos.
%https://elpais.com/tecnologia/tu-tecnologia/2024-12-07/dispositivos-tecnologicos-para-que-tus-plantas-sean-la-envidia-de-cualquiera.html?utm_source=chatgpt.com

\subsubsection{Smart 9 Pro de Click \& Grow}
El Smart 9 Pro de Click \& Grow es un jardín inteligente que facilita el cultivo de hortalizas y verduras en espacios interiores. Cuenta con sistemas de riego automático y luz artificial, proporcionando un entorno controlado y óptimo para el crecimiento de las plantas.
%https://elpais.com/tecnologia/tu-tecnologia/2024-12-07/dispositivos-tecnologicos-para-que-tus-plantas-sean-la-envidia-de-cualquiera.html?utm_source=chatgpt.com

\section{Objetivos y alcances}


\subsection{Objetivos}
\begin{itemize}
    \item Desarrollar un prototipo de sistema hidropónico automatizado para monitoreo y gestión eficiente del cultivo.
    \item Implementar sensores para medir temperatura, humedad, pH, conductividad eléctrica, nivel de agua e intensidad lumínica.
    \item Optimizar el uso de recursos, reduciendo el consumo de agua y nutrientes mediante control automatizado.
    \item Diseñar una interfaz web y móvil que permita la supervisión y configuración remota del sistema.
    \item Evaluar el rendimiento del sistema en comparación con métodos tradicionales, midiendo eficiencia y productividad.
\end{itemize}

\subsection{Alcances}

\begin{itemize}
    \item El prototipo permitirá el monitoreo en tiempo real y la gestión automatizada del riego, iluminación y ventilación.
    \item Se desarrollará un módulo de comunicación IoT, asegurando conectividad con dispositivos móviles y servidores locales.
    \item La solución se enfocará en cultivos hidropónicos verticales de pequeña y mediana escala.
    \item Se realizarán pruebas de funcionamiento y validación en un entorno controlado, pero no se contempla una implementación comercial en esta etapa.
\end{itemize}

%Si estás familiarizado con \LaTeX{}, entonces podés explorar la estructura de directorios de esta plantilla y proceder a personalizarla agregando tu información en el bloque \emph{INFORMACIÓN DE LA PORTADA} en el archivo \file{memoria.tex}.  

%Se puede continuar luego modificando el resto de los archivos siguiendo los lineamientos que se describen en la sección \ref{sec:FillingFile} en la página \pageref{sec:FillingFile}.

%Debés asegurarte de leer el capítulo \ref{Chapter2} acerca de las convenciones utilizadas para las Memoria de los Trabajos Finales de la \degreename.

%Si sos nuevo en \LaTeX{}, se recomienda que continúes leyendo el documento ya que contiene información básica para aprovechar el potencial de esta herramienta.


%----------------------------------------------------------------------------------------

\section{Qué incluye esta plantilla}

\subsection{Carpetas}

Esta plantilla se distribuye como una único archivo .zip que se puede descomprimir en varios archivos y carpetas. Asimismo, se puede consultar el repositorio git para obtener la última versión de los archivos, \url{https://github.com/patriciobos/Plantilla-CESE.git}. Los nombres de las carpetas son, o pretender ser, auto-explicativos.

\keyword{Appendices} -- Esta es la carpeta donde se deben poner los apéndices. Cada apéndice debe ir en su propio archivo \file{.tex}. Se incluye un ejemplo y una plantilla en la carpeta.

\keyword{Chapters} -- Esta es la carpeta donde se deben poner los capítulos de la memoria. Cada capítulo debe ir un su propio archivo \file{.tex} por separado.  Se ofrece por defecto, la siguiente estructura de capítulos y se recomienda su utilización dentro de lo posible:

\begin{itemize}
\item Capítulo 1: Introducción general	
\item Capítulo 2: Introducción específica
\item Capítulo 3: Diseño e implementación
\item Capítulo 4: Ensayos y resultados
\item Capítulo 5: Conclusiones

\end{itemize}

Esta estructura de capítulos es la que se recomienda para las memorias de la especialización.

\keyword{Figures} -- Esta carpeta contiene todas las figuras de la memoria.  Estas son las versiones finales de las imágenes que van a ser incluidas en la memoria.  Pueden ser imágenes en formato \textit{raster}\footnote{\url{https://en.wikipedia.org/wiki/Raster_graphics}} como \file{.png}, \file{.jpg} o en formato vectoriales\footnote{\url{https://en.wikipedia.org/wiki/Vector_graphics}} como \file{.pdf}, \file{.ps}.  Se debe notar que utilizar imágenes vectoriales disminuye notablemente el peso del documento final y acelera el tiempo de compilación por lo que es recomendable su utilización siempre que sea posible.

\subsection{Archivos}

También están incluidos varios archivos, la mayoría de ellos son de texto plano y se puede ver su contenido en un editor de texto. Después de la compilación inicial, se verá que más archivos auxiliares son creados por \ LaTeX{} o BibTeX, pero son de uso interno y no es necesario hacer nada en particular con ellos.  Toda la información necesaria para compilar el documento se encuentra en los archivos \file{.tex}, \file{.bib}, \file{.cls} y en las imágenes de la carpeta Figures.

\keyword{referencias.bib} - este es un archivo importante que contiene toda la información de referencias bibliográficas que se utilizarán para las citas en la memoria en conjunto con BibTeX. Usted puede escribir las entradas bibliográficas en forma manual, aunque existen también programas de gestión de referencias que facilitan la creación y gestión de las referencias y permiten exportarlas en formato BibTeX.  También hay disponibles sitios web como \url{books.google.com} que permiten obtener toda la información necesaria para una cita en formato BibTeX. Ver sección \ref{sec:biblio}

\keyword{MastersDoctoralThesis.cls} -- este es un archivo importante. Es el archivos con la clase que le informa a \LaTeX{} cómo debe dar formato a la memoria. El usuario de la plantilla no debería necesitar modificar nada de este archivo.

\keyword{memoria.pdf} -- esta es su memoria con una tipografía bellamente compuesta (en formato de archivo PDF) creado por \LaTeX{}. Se distribuye con la plantilla y después de compilar por primera vez sin hacer ningún cambio se debería obtener una versión idéntica a este documento.

\keyword{memoria.tex} -- este es un archivo importante. Este es el archivo que tiene que compilar \LaTeX{} para producir la memoria como un archivo PDF. Contiene un marco de trabajo y estructuras que le indican a \LaTeX{} cómo diagramar la memoria.  Está altamente comentado para que se pueda entender qué es lo que realiza cada línea de código y por qué está incluida en ese lugar.  En este archivo se debe completar la información personalizada de las primeras sección según se indica en la sección \ref{sec:FillingFile}.

Archivos que \emph{no} forman parte de la distribución de la plantilla pero que son generados por \LaTeX{} como archivos auxiliares necesarios para la producción de la memoria.pdf son:

\keyword{memoria.aux} -- este es un archivo auxiliar generado por \LaTeX{}, si se borra \LaTeX{} simplemente lo regenera cuando se compila el archivo principal \file{memoria.tex}.

\keyword{memoria.bbl} -- este es un archivo auxiliar generado por BibTeX, si se borra BibTeX simplemente lo regenera cuando se compila el archivo principal \file{memoria.tex}. Mientras que el archivo \file{.bib} contiene todas las referencias que hay, este archivo \file{.bbl} contine sólo las referencias que han sido citadas y se utiliza para la construcción de la bibiografía.

\keyword{memoria.blg} -- este es un archivo auxiliar generado por BibTeX, si se borra BibTeX simplemente lo regenera cuando se compila el archivo principal \file{memoria.tex}.

\keyword{memoria.lof} -- este es un archivo auxiliar generado por \LaTeX{}, si se borra \LaTeX{} simplemente lo regenera cuando se compila el archivo principal \file{memoria.tex}.  Le indica a \LaTeX{} cómo construir la sección \emph{Lista de Figuras}.
 
\keyword{memoria.log} --  este es un archivo auxiliar generado por \LaTeX{}, si se borra \LaTeX{} simplemente lo regenera cuando se compila el archivo principal \file{memoria.tex}. Contiene mensajes de \LaTeX{}. Si se reciben errores o advertencias durante la compilación, se guardan en este archivo \file{.log}.

\keyword{memoria.lot} -- este es un archivo auxiliar generado por \LaTeX{}, si se borra \LaTeX{} simplemente lo regenera cuando se compila el archivo principal \file{memoria.tex}.  Le indica a \LaTeX{} cómo construir la sección \emph{Lista de Tablas}.

\keyword{memoria.out} -- este es un archivo auxiliar generado por \LaTeX{}, si se borra \LaTeX{} simplemente lo regenera cuando se compila el archivo principal \file{memoria.tex}.

De esta larga lista de archivos, sólo aquellos con la extensión \file{.bib}, \file{.cls} y \file{.tex} son importantes.  Los otros archivos auxiliares pueden ser ignorados o borrados ya que \LaTeX{} y BibTeX los regenerarán durante la compilación.

%----------------------------------------------------------------------------------------

\section{Entorno de trabajo}

Ante de comenzar a editar la plantilla debemos tener un editor \LaTeX{} instalado en nuestra computadora.  En forma análoga a lo que sucede en lenguaje C, que se puede crear y editar código con casi cualquier editor, existen ciertos entornos de trabajo que nos pueden simplificar mucho la tarea.  En este sentido, se recomienda, sobre todo para los principiantes en \LaTeX{} la utilización de TexMaker, un programa gratuito y multi-plantaforma que está disponible tanto para windows como para sistemas GNU/linux.

La versión más reciente de TexMaker es la 4.5 y se puede descargar del siguiente link: \url{http://www.xm1math.net/texmaker/download.html}. Se puede consultar el manual de usuario en el siguiente link: \url{http://www.xm1math.net/texmaker/doc.html}.
 

\subsection{Paquetes adicionales}

Si bien durante el proceso de instalación de TexMaker, o cualquier otro editor que se haya elegido, se instalarán en el sistema los paquetes básicos necesarios para trabajar con \LaTeX{}, la plantilla de los trabajos de Especialización y Maestría requieren de paquete adicionales.

Se indican a continuación los comandos que se deben introducir en la consola de Ubuntu (ctrl + alt + t) para instalarlos:

\begin{lstlisting}[language=bash]
  $ sudo apt install texlive-lang-spanish texlive-science 
  $ sudo apt install texlive-bibtex-extra biber
  $ sudo apt install texlive texlive-fonts-recommended
  $ sudo apt install texlive-latex-extra
\end{lstlisting}


\subsection{Configurando TexMaker}
\label{subsec:configurando}



Una vez instalado el programa y los paquetes adicionales se debe abrir el archivo memoria.tex con el editor para ver una pantalla similar a la que se puede apreciar en la figura \ref{fig:texmaker}. 
Una vez instalado el programa y los paquetes adicionales se debe abrir el archivo memoria.tex con el editor para ver una pantalla similar a la que se puede apreciar en la figura \ref{fig:texmaker}. 
Una vez instalado el programa y los paquetes adicionales se debe abrir el archivo memoria.tex con el editor para ver una pantalla similar a la que se puede apreciar en la figura \ref{fig:texmaker}. 
Una vez instalado el programa y los paquetes adicionales se debe abrir el archivo memoria.tex con el editor para ver una pantalla similar a la que se puede apreciar en la figura \ref{fig:texmaker}. 

\vspace{1cm}

\begin{figure}[htbp]
	\centering
	\includegraphics[width=.5\textwidth]{./Figures/texmaker.png}
	\caption{Entorno de trabajo de texMaker.}
	\label{fig:texmaker}
\end{figure}

\vspace{1cm}

Notar que existe una vista llamada Estructura a la izquierda de la interfaz que nos permite abrir desde dentro del programa los archivos individuales de los capítulos.  A la derecha se encuentra una vista con el archivo propiamente dicho para su edición. Hacia la parte inferior se encuentra una vista del log con información de los resultados de la compilación.  En esta última vista pueden aparecen advertencias o \textit{warning}, que normalmente pueden ser ignorados, y los errores que se indican en color rojo y deben resolverse para que se genere el PDF de salida.

Recordar que el archivo que se debe compilar con PDFLaTeX es \file{memoria.tex}, si se tratara de compilar alguno de los capítulos saldría un error.  Para salvar la molestia de tener que cambiar de archivo para compilar cada vez que se realice una modificación en un capítulo, se puede definir el archivo \file{memoria.tex} como ``documento maestro'' yendo al menú opciones -> ``definir documento actual como documento maestro'', lo que permite compilar con PDFLaTeX memoria.tex directamente desde cualquier archivo que se esté modificando . Se muestra esta opción en la figura \ref{fig:docMaestro}.

\begin{figure}[h]
	\centering
	\includegraphics[width=\textwidth]{./Figures/docMaestro.png}
	\caption{Definir memoria.tex como documento maestro.}
	\label{fig:docMaestro}
\end{figure}

En el menú herramientas se encuentran las opciones de compilación.  Para producir un archivo PDF a partir de un archivo .tex se debe ejecutar PDFLaTeX (el shortcut es F6). Para incorporar nueva bibliografía se debe utilizar la opción BibTeX del mismo menú herramientas (el shortcut es F11).

Notar que para actualizar las tablas de contenidos se debe ejecutar PDFLaTeX dos veces.  Esto se debe a que es necesario actualizar algunos archivos auxiliares antes de obtener el resultado final.  En forma similar, para actualizar las referencias bibliográficas se debe ejecutar primero PDFLaTeX, después BibTeX y finalmente PDFLaTeX dos veces por idénticos motivos.

\section{Personalizando la plantilla, el archivo \file{memoria.tex}}
\label{sec:FillingFile}

Para personalizar la plantilla se debe incorporar la información propia en los distintos archivos \file{.tex}. 

Primero abrir \file{memoria.tex} con TexMaker (o el editor de su preferencia). Se debe ubicar dentro del archivo el bloque de código titulado \emph{INFORMACIÓN DE LA PORTADA} donde se deben incorporar los primeros datos personales con los que se construirá automáticamente la portada.


%----------------------------------------------------------------------------------------

\section{El código del archivo \file{memoria.tex} explicado}

El archivo \file{memoria.tex} contiene la estructura del documento y es el archivo de mayor jerarquía de la memoria.  Podría ser equiparable a la función \emph{main()} de un programa en C, o mejor dicho al archivo fuente .c donde se encuentra definida la función main().

La estructura básica de cualquier documento de \LaTeX{} comienza con la definición de clase del documento, es seguida por un preámbulo donde se pueden agregar funcionalidades con el uso de \texttt{paquetes} (equiparables a bibliotecas de C), y finalmente, termina con el cuerpo del documento, donde irá el contenido de la memoria.

\lstset{%
  basicstyle=\small\ttfamily,
  language=[LaTeX]{TeX}
}

\begin{lstlisting}
\documentclass{article}  <- Definicion de clase
\usepackage{listings}	 <- Preambulo

\begin{document}	 <- Comienzo del contenido propio 
	Hello world!
\end{document}
\end{lstlisting}


El archivo \file{memoria.tex} se encuentra densamente comentado para explicar qué páginas, secciones y elementos de formato está creando el código \LaTeX{} en cada línea. El código está dividido en bloques con nombres en mayúsculas para que resulte evidente qué es lo que hace esa porción de código en particular. Inicialmente puede parecer que hay mucho código \LaTeX{}, pero es principalmente código para dar formato a la memoria por lo que no requiere intervención del usuario de la plantilla.  Sí se deben personalizar con su información los bloques indicados como:

\begin{itemize}
	\item Informacion de la memoria
	\item Resumen
	\item Agradecimientos
	\item Dedicatoria
\end{itemize}

El índice de contenidos, las listas de figura de tablas se generan en forma automática y no requieren intervención ni edición manual por parte del usuario de la plantilla. 

En la parte final del documento se encuentran los capítulos y los apéndices.  Por defecto se incluyen los 5 capítulos propuestos que se encuentran en la carpeta /Chapters. Cada capítulo se debe escribir en un archivo .tex separado y se debe poner en la carpeta \emph{Chapters} con el nombre \file{Chapter1}, \file{Chapter2}, etc\ldots El código para incluir capítulos desde archivos externos se muestra a continuación.

\begin{verbatim}
	% Chapter 1

\chapter{Introducción general} % Main chapter title

\label{Chapter1} % For referencing the chapter elsewhere, use \ref{Chapter1} 
\label{IntroGeneral}

En este capítulo se hace una breve introducción a la necesidad que condujo al
desarrollo del trabajo. Se presenta el concepto de cultivo vertical, y el estado del arte de su integración con la tecnología. Asimismo, se explica el objetivo y los alcances del trabajo.

%----------------------------------------------------------------------------------------

% Define some commands to keep the formatting separated from the content 
\newcommand{\keyword}[1]{\textbf{#1}}
\newcommand{\tabhead}[1]{\textbf{#1}}
\newcommand{\code}[1]{\texttt{#1}}
\newcommand{\file}[1]{\texttt{\bfseries#1}}
\newcommand{\option}[1]{\texttt{\itshape#1}}
\newcommand{\grados}{$^{\circ}$}

%----------------------------------------------------------------------------------------

%\section{Introducción}

%----------------------------------------------------------------------------------------
\section{Introducción a los cultivos verticales}

%\LaTeX{} no es \textsc{WYSIWYG} (What You See is What You Get), a diferencia de los procesadores de texto como Microsoft Word o Pages de Apple o incluso LibreOffice en el mundo open-source. En lugar de ello, un documento escrito para \LaTeX{} es en realidad un archivo de texto simple o llano que \emph{no contiene formato} . Nosotros le decimos a \LaTeX{} cómo deseamos que se aplique el formato en el documento final escribiendo comandos simples entre el texto, por ejemplo, si quiero usar texto en itálicas para dar énfasis, escribo \verb|\it{texto}| y pongo el texto que quiero en itálicas entre medio de las llaves. Esto significa que \LaTeX{} es un lenguaje del tipo \enquote{mark-up}, muy parecido a HTML.

Con una población mundial que supera los 8.000 millones de personas y continúa en crecimiento, la agricultura enfrenta el desafío de volverse más eficiente y sostenible. En este contexto, los cultivos verticales emergen como una solución innovadora que optimiza el uso del espacio y los recursos, especialmente en entornos urbanos, donde el suelo es limitado y costoso.

Este tipo de agricultura emplea técnicas como la hidroponía, que permite un uso preciso del agua, y la aeroponía, que maximiza la oxigenación de las raíces. Además, las granjas verticales aprovechan áreas infrautilizadas, como edificios abandonados o naves industriales, permitiendo el cultivo de alimentos en zonas donde la agricultura tradicional resulta difícil de implementar. De este modo, se logra una mayor densidad de cultivo y se contribuye a la sostenibilidad, reduciendo el uso de pesticidas y fertilizantes.

Aunque los cultivos verticales requieren altos niveles de tecnología y conllevan ciertos costos energéticos, su capacidad para optimizar recursos, reducir la huella de carbono y fomentar la producción local y el autoconsumo los posiciona como una alternativa clave para la agricultura del futuro.

Este método, también conocido como agricultura urbana, permite el crecimiento de plantas en interiores sin necesidad de luz solar directa, gracias a la aplicación de tecnologías agrícolas avanzadas. Estas optimizan las condiciones de cultivo, facilitando la propagación de plantas jóvenes y la producción de alimentos más saludables sin el uso de pesticidas.

Además, los cultivos verticales maximizan la producción al emplear sistemas de iluminación especializados de nueva generación, diseñados para estructuras multicapa. Gracias a esta tecnología, es posible obtener mayores rendimientos en un espacio reducido, consolidando a la agricultura vertical como una solución sostenible y eficiente para la producción de alimentos en entornos urbanos y con recursos limitados.

\subsection{Tipos de cultivos verticales}

%https://agroquivir.com/agricultura-vertical-como-funciona-y-tipos/

%Si sos nuevo en \LaTeX{}, hay un muy buen libro electrónico - disponible gratuitamente en Internet como un archivo PDF - llamado, \enquote{A (not so short) Introduction to \LaTeX{}}. El título del libro es generalmente acortado a simplemente \emph{lshort}. Puede descargar la versión más reciente en inglés (ya que se actualiza de vez en cuando) desde aquí:
%\url{http://www.ctan.org/tex-archive/info/lshort/english/lshort.pdf}

%Se puede encontrar la versión en español en la lista en esta página: \url{http://www.ctan.org/tex-archive/info/lshort/}

Existen distintos tipos de agricultura vertical, que optimizan el crecimiento de las plantas sin grandes extensiones de suelo y emplean estructuras especializadas para maximizar la producción.

\subsubsection{Hidroponía}
La hidroponía es un método de cultivo sin suelo que suministra nutrientes a las plantas a través de una solución acuosa, con raíces en un medio inerte como lana de roca o perlita. Es un sistema eficiente en el uso de agua, permite un control preciso de nutrientes y favorece una mayor densidad de cultivo y un crecimiento más rápido que los métodos tradicionales.
\subsubsection{Aeroponía}
La aeroponía es un método de cultivo en el que las raíces de las plantas quedan suspendidas en el aire y se rocían con una solución nutritiva. Maximiza la oxigenación, acelera el crecimiento y optimiza el uso de nutrientes, aunque requiere un control preciso para evitar la desecación de las raíces.
\subsubsection{Acuaponía}
La acuaponía combina la cría de peces con la hidroponía, utilizando los desechos de los peces como fertilizante para las plantas, mientras estas purifican el agua. Es un sistema sostenible y eficiente que minimiza el desperdicio, promueve la biodiversidad y es viable en entornos urbanos y rurales.

Acá tiene un ejemplo de una ``subsubsección'' que es el cuarto nivel de ordenamiento del texto, después de capítulo, sección y subsección.  Como se puede ver, las subsubsecciones no van numeradas en el cuerpo del documento ni en el índice.  El formato está definido por la plantilla y no debe ser modificado.

\subsection{Sistemas de agricultura vertical más conocidos}
Los sistemas de agricultura vertical varían según el tipo de cultivo y la técnica más adecuada para cada necesidad.

\subsubsection{Sistemas de torre}
Las torres verticales permiten el cultivo de plantas en estructuras cilíndricas apiladas, maximizando el uso del espacio y facilitando tanto la recolección como el mantenimiento.
Este sistema es ideal para cultivos de hoja verde y hierbas, y se integra fácilmente en espacios urbanos como balcones y terrazas.

\subsubsection{Sistemas en rack o estanterías}
Estos sistemas emplean estanterías apiladas donde las plantas crecen en bandejas o contenedores, siendo ideales para invernaderos y ambientes controlados. Facilitan el riego, la recolección y el monitoreo, optimizando el uso del espacio.
Los sistemas en rack o estanterías son especialmente útiles para cultivos que requieren diferentes niveles de luz y humedad.

\subsubsection{Sistemas montados en pared}
En estos sistemas, las plantas crecen en módulos montados en paredes, aprovechando superficies verticales en entornos urbanos. Su integración en la arquitectura no solo optimiza el espacio, sino que también aporta beneficios estéticos y funcionales.
Los jardines verticales pueden mejorar la calidad del aire y contribuir al aislamiento térmico de los edificios.

\subsubsection{Sistemas de bastidor en A}
El bastidor en forma de “A” proporciona una estructura estable y accesible para el cultivo vertical, facilitando el acceso a la luz y el riego.
Este diseño es particularmente útil para cultivos que necesitan un soporte adicional, como tomates y pepinos, y puede ser utilizado tanto en interiores como en exteriores.

\subsubsection{Sistemas de contenedores}
Estos sistemas utilizan contenedores apilables que pueden moverse y configurarse según las necesidades, adaptándose a diversos entornos y tipos de cultivo.
Los contenedores son ideales para cultivos modulares y pueden ser personalizados para optimizar el uso del espacio y los recursos.
%Si estás escribiendo un documento con mucho contenido matemático, entonces es posible que desees leer el documento de la AMS (American Mathematical Society) llamado, \enquote{A Short Math Guide for \LaTeX{}}. Se puede encontrar en línea en el siguiente link: \url{http://www.ams.org/tex/amslatex.html} en la sección \enquote{Additional Documentation} hacia la parte inferior de la página.


%----------------------------------------------------------------------------------------

\section{Motivación}

Este trabajo surge como un desarrollo de interés personal, impulsado por el deseo de contribuir positivamente al desarrollo sustentable y la optimización de los recursos naturales en la producción de alimentos. En la actualidad, el crecimiento de la población mundial y el aumento en la demanda de productos agrícolas presentan grandes desafíos ambientales y económicos.

La producción tradicional de alimentos enfrenta dificultades debido a la explotación intensiva de suelos, el uso excesivo de fertilizantes y pesticidas, y el desperdicio de recursos esenciales como el agua. Con el fin de maximizar sus ganancias, muchas empresas recurren a métodos poco sostenibles, los cuales generan un impacto ambiental negativo, afectando la biodiversidad, la calidad del suelo y el bienestar de las comunidades locales que dependen de estas fuentes de producción.

Uno de los problemas más críticos es el desperdicio de agua en la agricultura tradicional. Se estima que casi el 90 \% del agua utilizada en el riego se pierde por evaporación, drenaje o absorción ineficiente. En un contexto donde la crisis hídrica es una amenaza global, desarrollar sistemas más eficientes y sostenibles se vuelve una necesidad urgente.

Por otro lado, existe una creciente preocupación entre los consumidores sobre la calidad y procedencia de los alimentos. Cada vez más personas optan por un estilo de vida más saludable, adoptando hábitos de consumo responsables, como el vegetarianismo, el veganismo y el autocultivo. Sin embargo, muchos de estos consumidores no cuentan con espacios adecuados ni conocimientos técnicos para producir sus propios alimentos, lo que genera una gran oportunidad para soluciones tecnológicas que faciliten la agricultura doméstica o urbana.

A partir de estas problemáticas, nace la motivación para desarrollar un sistema hidropónico inteligente, que no solo optimice el uso del agua y los nutrientes, sino que también permita una gestión eficiente del cultivo con mínima intervención del usuario. Este trabajo ofrece una alternativa viable y escalable para la producción de alimentos frescos, reduciendo la dependencia de sistemas agrícolas tradicionales y promoviendo un modelo más sustentable para el futuro.


\section{Estado del arte}
Durante la etapa de investigación, se llevó a cabo una búsqueda de productos comerciales tanto en el mercado local como en el internacional. Se identificaron diversas soluciones con características similares al sistema desarrollado.

A continuación, se describen los productos identificados, los cuales presentan variaciones en cuanto a la tecnología utilizada.

\subsubsection{NIDO PRO}
NIDO PRO es un sistema hidropónico inteligente diseñado para cultivos verticales. Permite el control y la gestión de la solución nutritiva, estableciendo valores de pH y conductividad eléctrica (CE) desde una aplicación móvil. Sus algoritmos realizan comprobaciones automáticas diarias para garantizar la estabilidad de la solución nutritiva. Además, puede gestionar hasta cuatro ranuras para fertilizantes líquidos y ajustes de pH.
%https://www.agrointec.com/producto/nidopro-sistema-hidroponico-inteligente-cultivo-vertical/?utm_source=chatgpt.com

\subsubsection{Xiaomi Mi Flower Sensor}
Xiaomi Mi Flower Care Plant Sensor permite monitorear variables ambientales como luz, humedad, nutrientes y temperatura del sustrato. Estos sensores se conectan a aplicaciones móviles, facilitando el seguimiento y control de las condiciones del cultivo.
%https://elpais.com/tecnologia/tu-tecnologia/2024-12-07/dispositivos-tecnologicos-para-que-tus-plantas-sean-la-envidia-de-cualquiera.html?utm_source=chatgpt.com

\subsubsection{Kit de Riego Hydro de Konyks}
El Kit de Riego Hydro de Konyks ofrece control remoto del riego, permitiendo regular el caudal y programar horarios mediante una aplicación móvil. Esto asegura un suministro de agua adecuado y constante para los cultivos hidropónicos.
%https://elpais.com/tecnologia/tu-tecnologia/2024-12-07/dispositivos-tecnologicos-para-que-tus-plantas-sean-la-envidia-de-cualquiera.html?utm_source=chatgpt.com

\subsubsection{Smart 9 Pro de Click \& Grow}
El Smart 9 Pro de Click \& Grow es un jardín inteligente que facilita el cultivo de hortalizas y verduras en espacios interiores. Cuenta con sistemas de riego automático y luz artificial, proporcionando un entorno controlado y óptimo para el crecimiento de las plantas.
%https://elpais.com/tecnologia/tu-tecnologia/2024-12-07/dispositivos-tecnologicos-para-que-tus-plantas-sean-la-envidia-de-cualquiera.html?utm_source=chatgpt.com

\section{Objetivos y alcances}


\subsection{Objetivos}
\begin{itemize}
    \item Desarrollar un prototipo de sistema hidropónico automatizado para monitoreo y gestión eficiente del cultivo.
    \item Implementar sensores para medir temperatura, humedad, pH, conductividad eléctrica, nivel de agua e intensidad lumínica.
    \item Optimizar el uso de recursos, reduciendo el consumo de agua y nutrientes mediante control automatizado.
    \item Diseñar una interfaz web y móvil que permita la supervisión y configuración remota del sistema.
    \item Evaluar el rendimiento del sistema en comparación con métodos tradicionales, midiendo eficiencia y productividad.
\end{itemize}

\subsection{Alcances}

\begin{itemize}
    \item El prototipo permitirá el monitoreo en tiempo real y la gestión automatizada del riego, iluminación y ventilación.
    \item Se desarrollará un módulo de comunicación IoT, asegurando conectividad con dispositivos móviles y servidores locales.
    \item La solución se enfocará en cultivos hidropónicos verticales de pequeña y mediana escala.
    \item Se realizarán pruebas de funcionamiento y validación en un entorno controlado, pero no se contempla una implementación comercial en esta etapa.
\end{itemize}

%Si estás familiarizado con \LaTeX{}, entonces podés explorar la estructura de directorios de esta plantilla y proceder a personalizarla agregando tu información en el bloque \emph{INFORMACIÓN DE LA PORTADA} en el archivo \file{memoria.tex}.  

%Se puede continuar luego modificando el resto de los archivos siguiendo los lineamientos que se describen en la sección \ref{sec:FillingFile} en la página \pageref{sec:FillingFile}.

%Debés asegurarte de leer el capítulo \ref{Chapter2} acerca de las convenciones utilizadas para las Memoria de los Trabajos Finales de la \degreename.

%Si sos nuevo en \LaTeX{}, se recomienda que continúes leyendo el documento ya que contiene información básica para aprovechar el potencial de esta herramienta.


%----------------------------------------------------------------------------------------

\section{Qué incluye esta plantilla}

\subsection{Carpetas}

Esta plantilla se distribuye como una único archivo .zip que se puede descomprimir en varios archivos y carpetas. Asimismo, se puede consultar el repositorio git para obtener la última versión de los archivos, \url{https://github.com/patriciobos/Plantilla-CESE.git}. Los nombres de las carpetas son, o pretender ser, auto-explicativos.

\keyword{Appendices} -- Esta es la carpeta donde se deben poner los apéndices. Cada apéndice debe ir en su propio archivo \file{.tex}. Se incluye un ejemplo y una plantilla en la carpeta.

\keyword{Chapters} -- Esta es la carpeta donde se deben poner los capítulos de la memoria. Cada capítulo debe ir un su propio archivo \file{.tex} por separado.  Se ofrece por defecto, la siguiente estructura de capítulos y se recomienda su utilización dentro de lo posible:

\begin{itemize}
\item Capítulo 1: Introducción general	
\item Capítulo 2: Introducción específica
\item Capítulo 3: Diseño e implementación
\item Capítulo 4: Ensayos y resultados
\item Capítulo 5: Conclusiones

\end{itemize}

Esta estructura de capítulos es la que se recomienda para las memorias de la especialización.

\keyword{Figures} -- Esta carpeta contiene todas las figuras de la memoria.  Estas son las versiones finales de las imágenes que van a ser incluidas en la memoria.  Pueden ser imágenes en formato \textit{raster}\footnote{\url{https://en.wikipedia.org/wiki/Raster_graphics}} como \file{.png}, \file{.jpg} o en formato vectoriales\footnote{\url{https://en.wikipedia.org/wiki/Vector_graphics}} como \file{.pdf}, \file{.ps}.  Se debe notar que utilizar imágenes vectoriales disminuye notablemente el peso del documento final y acelera el tiempo de compilación por lo que es recomendable su utilización siempre que sea posible.

\subsection{Archivos}

También están incluidos varios archivos, la mayoría de ellos son de texto plano y se puede ver su contenido en un editor de texto. Después de la compilación inicial, se verá que más archivos auxiliares son creados por \ LaTeX{} o BibTeX, pero son de uso interno y no es necesario hacer nada en particular con ellos.  Toda la información necesaria para compilar el documento se encuentra en los archivos \file{.tex}, \file{.bib}, \file{.cls} y en las imágenes de la carpeta Figures.

\keyword{referencias.bib} - este es un archivo importante que contiene toda la información de referencias bibliográficas que se utilizarán para las citas en la memoria en conjunto con BibTeX. Usted puede escribir las entradas bibliográficas en forma manual, aunque existen también programas de gestión de referencias que facilitan la creación y gestión de las referencias y permiten exportarlas en formato BibTeX.  También hay disponibles sitios web como \url{books.google.com} que permiten obtener toda la información necesaria para una cita en formato BibTeX. Ver sección \ref{sec:biblio}

\keyword{MastersDoctoralThesis.cls} -- este es un archivo importante. Es el archivos con la clase que le informa a \LaTeX{} cómo debe dar formato a la memoria. El usuario de la plantilla no debería necesitar modificar nada de este archivo.

\keyword{memoria.pdf} -- esta es su memoria con una tipografía bellamente compuesta (en formato de archivo PDF) creado por \LaTeX{}. Se distribuye con la plantilla y después de compilar por primera vez sin hacer ningún cambio se debería obtener una versión idéntica a este documento.

\keyword{memoria.tex} -- este es un archivo importante. Este es el archivo que tiene que compilar \LaTeX{} para producir la memoria como un archivo PDF. Contiene un marco de trabajo y estructuras que le indican a \LaTeX{} cómo diagramar la memoria.  Está altamente comentado para que se pueda entender qué es lo que realiza cada línea de código y por qué está incluida en ese lugar.  En este archivo se debe completar la información personalizada de las primeras sección según se indica en la sección \ref{sec:FillingFile}.

Archivos que \emph{no} forman parte de la distribución de la plantilla pero que son generados por \LaTeX{} como archivos auxiliares necesarios para la producción de la memoria.pdf son:

\keyword{memoria.aux} -- este es un archivo auxiliar generado por \LaTeX{}, si se borra \LaTeX{} simplemente lo regenera cuando se compila el archivo principal \file{memoria.tex}.

\keyword{memoria.bbl} -- este es un archivo auxiliar generado por BibTeX, si se borra BibTeX simplemente lo regenera cuando se compila el archivo principal \file{memoria.tex}. Mientras que el archivo \file{.bib} contiene todas las referencias que hay, este archivo \file{.bbl} contine sólo las referencias que han sido citadas y se utiliza para la construcción de la bibiografía.

\keyword{memoria.blg} -- este es un archivo auxiliar generado por BibTeX, si se borra BibTeX simplemente lo regenera cuando se compila el archivo principal \file{memoria.tex}.

\keyword{memoria.lof} -- este es un archivo auxiliar generado por \LaTeX{}, si se borra \LaTeX{} simplemente lo regenera cuando se compila el archivo principal \file{memoria.tex}.  Le indica a \LaTeX{} cómo construir la sección \emph{Lista de Figuras}.
 
\keyword{memoria.log} --  este es un archivo auxiliar generado por \LaTeX{}, si se borra \LaTeX{} simplemente lo regenera cuando se compila el archivo principal \file{memoria.tex}. Contiene mensajes de \LaTeX{}. Si se reciben errores o advertencias durante la compilación, se guardan en este archivo \file{.log}.

\keyword{memoria.lot} -- este es un archivo auxiliar generado por \LaTeX{}, si se borra \LaTeX{} simplemente lo regenera cuando se compila el archivo principal \file{memoria.tex}.  Le indica a \LaTeX{} cómo construir la sección \emph{Lista de Tablas}.

\keyword{memoria.out} -- este es un archivo auxiliar generado por \LaTeX{}, si se borra \LaTeX{} simplemente lo regenera cuando se compila el archivo principal \file{memoria.tex}.

De esta larga lista de archivos, sólo aquellos con la extensión \file{.bib}, \file{.cls} y \file{.tex} son importantes.  Los otros archivos auxiliares pueden ser ignorados o borrados ya que \LaTeX{} y BibTeX los regenerarán durante la compilación.

%----------------------------------------------------------------------------------------

\section{Entorno de trabajo}

Ante de comenzar a editar la plantilla debemos tener un editor \LaTeX{} instalado en nuestra computadora.  En forma análoga a lo que sucede en lenguaje C, que se puede crear y editar código con casi cualquier editor, existen ciertos entornos de trabajo que nos pueden simplificar mucho la tarea.  En este sentido, se recomienda, sobre todo para los principiantes en \LaTeX{} la utilización de TexMaker, un programa gratuito y multi-plantaforma que está disponible tanto para windows como para sistemas GNU/linux.

La versión más reciente de TexMaker es la 4.5 y se puede descargar del siguiente link: \url{http://www.xm1math.net/texmaker/download.html}. Se puede consultar el manual de usuario en el siguiente link: \url{http://www.xm1math.net/texmaker/doc.html}.
 

\subsection{Paquetes adicionales}

Si bien durante el proceso de instalación de TexMaker, o cualquier otro editor que se haya elegido, se instalarán en el sistema los paquetes básicos necesarios para trabajar con \LaTeX{}, la plantilla de los trabajos de Especialización y Maestría requieren de paquete adicionales.

Se indican a continuación los comandos que se deben introducir en la consola de Ubuntu (ctrl + alt + t) para instalarlos:

\begin{lstlisting}[language=bash]
  $ sudo apt install texlive-lang-spanish texlive-science 
  $ sudo apt install texlive-bibtex-extra biber
  $ sudo apt install texlive texlive-fonts-recommended
  $ sudo apt install texlive-latex-extra
\end{lstlisting}


\subsection{Configurando TexMaker}
\label{subsec:configurando}



Una vez instalado el programa y los paquetes adicionales se debe abrir el archivo memoria.tex con el editor para ver una pantalla similar a la que se puede apreciar en la figura \ref{fig:texmaker}. 
Una vez instalado el programa y los paquetes adicionales se debe abrir el archivo memoria.tex con el editor para ver una pantalla similar a la que se puede apreciar en la figura \ref{fig:texmaker}. 
Una vez instalado el programa y los paquetes adicionales se debe abrir el archivo memoria.tex con el editor para ver una pantalla similar a la que se puede apreciar en la figura \ref{fig:texmaker}. 
Una vez instalado el programa y los paquetes adicionales se debe abrir el archivo memoria.tex con el editor para ver una pantalla similar a la que se puede apreciar en la figura \ref{fig:texmaker}. 

\vspace{1cm}

\begin{figure}[htbp]
	\centering
	\includegraphics[width=.5\textwidth]{./Figures/texmaker.png}
	\caption{Entorno de trabajo de texMaker.}
	\label{fig:texmaker}
\end{figure}

\vspace{1cm}

Notar que existe una vista llamada Estructura a la izquierda de la interfaz que nos permite abrir desde dentro del programa los archivos individuales de los capítulos.  A la derecha se encuentra una vista con el archivo propiamente dicho para su edición. Hacia la parte inferior se encuentra una vista del log con información de los resultados de la compilación.  En esta última vista pueden aparecen advertencias o \textit{warning}, que normalmente pueden ser ignorados, y los errores que se indican en color rojo y deben resolverse para que se genere el PDF de salida.

Recordar que el archivo que se debe compilar con PDFLaTeX es \file{memoria.tex}, si se tratara de compilar alguno de los capítulos saldría un error.  Para salvar la molestia de tener que cambiar de archivo para compilar cada vez que se realice una modificación en un capítulo, se puede definir el archivo \file{memoria.tex} como ``documento maestro'' yendo al menú opciones -> ``definir documento actual como documento maestro'', lo que permite compilar con PDFLaTeX memoria.tex directamente desde cualquier archivo que se esté modificando . Se muestra esta opción en la figura \ref{fig:docMaestro}.

\begin{figure}[h]
	\centering
	\includegraphics[width=\textwidth]{./Figures/docMaestro.png}
	\caption{Definir memoria.tex como documento maestro.}
	\label{fig:docMaestro}
\end{figure}

En el menú herramientas se encuentran las opciones de compilación.  Para producir un archivo PDF a partir de un archivo .tex se debe ejecutar PDFLaTeX (el shortcut es F6). Para incorporar nueva bibliografía se debe utilizar la opción BibTeX del mismo menú herramientas (el shortcut es F11).

Notar que para actualizar las tablas de contenidos se debe ejecutar PDFLaTeX dos veces.  Esto se debe a que es necesario actualizar algunos archivos auxiliares antes de obtener el resultado final.  En forma similar, para actualizar las referencias bibliográficas se debe ejecutar primero PDFLaTeX, después BibTeX y finalmente PDFLaTeX dos veces por idénticos motivos.

\section{Personalizando la plantilla, el archivo \file{memoria.tex}}
\label{sec:FillingFile}

Para personalizar la plantilla se debe incorporar la información propia en los distintos archivos \file{.tex}. 

Primero abrir \file{memoria.tex} con TexMaker (o el editor de su preferencia). Se debe ubicar dentro del archivo el bloque de código titulado \emph{INFORMACIÓN DE LA PORTADA} donde se deben incorporar los primeros datos personales con los que se construirá automáticamente la portada.


%----------------------------------------------------------------------------------------

\section{El código del archivo \file{memoria.tex} explicado}

El archivo \file{memoria.tex} contiene la estructura del documento y es el archivo de mayor jerarquía de la memoria.  Podría ser equiparable a la función \emph{main()} de un programa en C, o mejor dicho al archivo fuente .c donde se encuentra definida la función main().

La estructura básica de cualquier documento de \LaTeX{} comienza con la definición de clase del documento, es seguida por un preámbulo donde se pueden agregar funcionalidades con el uso de \texttt{paquetes} (equiparables a bibliotecas de C), y finalmente, termina con el cuerpo del documento, donde irá el contenido de la memoria.

\lstset{%
  basicstyle=\small\ttfamily,
  language=[LaTeX]{TeX}
}

\begin{lstlisting}
\documentclass{article}  <- Definicion de clase
\usepackage{listings}	 <- Preambulo

\begin{document}	 <- Comienzo del contenido propio 
	Hello world!
\end{document}
\end{lstlisting}


El archivo \file{memoria.tex} se encuentra densamente comentado para explicar qué páginas, secciones y elementos de formato está creando el código \LaTeX{} en cada línea. El código está dividido en bloques con nombres en mayúsculas para que resulte evidente qué es lo que hace esa porción de código en particular. Inicialmente puede parecer que hay mucho código \LaTeX{}, pero es principalmente código para dar formato a la memoria por lo que no requiere intervención del usuario de la plantilla.  Sí se deben personalizar con su información los bloques indicados como:

\begin{itemize}
	\item Informacion de la memoria
	\item Resumen
	\item Agradecimientos
	\item Dedicatoria
\end{itemize}

El índice de contenidos, las listas de figura de tablas se generan en forma automática y no requieren intervención ni edición manual por parte del usuario de la plantilla. 

En la parte final del documento se encuentran los capítulos y los apéndices.  Por defecto se incluyen los 5 capítulos propuestos que se encuentran en la carpeta /Chapters. Cada capítulo se debe escribir en un archivo .tex separado y se debe poner en la carpeta \emph{Chapters} con el nombre \file{Chapter1}, \file{Chapter2}, etc\ldots El código para incluir capítulos desde archivos externos se muestra a continuación.

\begin{verbatim}
	% Chapter 1

\chapter{Introducción general} % Main chapter title

\label{Chapter1} % For referencing the chapter elsewhere, use \ref{Chapter1} 
\label{IntroGeneral}

En este capítulo se hace una breve introducción a la necesidad que condujo al
desarrollo del trabajo. Se presenta el concepto de cultivo vertical, y el estado del arte de su integración con la tecnología. Asimismo, se explica el objetivo y los alcances del trabajo.

%----------------------------------------------------------------------------------------

% Define some commands to keep the formatting separated from the content 
\newcommand{\keyword}[1]{\textbf{#1}}
\newcommand{\tabhead}[1]{\textbf{#1}}
\newcommand{\code}[1]{\texttt{#1}}
\newcommand{\file}[1]{\texttt{\bfseries#1}}
\newcommand{\option}[1]{\texttt{\itshape#1}}
\newcommand{\grados}{$^{\circ}$}

%----------------------------------------------------------------------------------------

%\section{Introducción}

%----------------------------------------------------------------------------------------
\section{Introducción a los cultivos verticales}

%\LaTeX{} no es \textsc{WYSIWYG} (What You See is What You Get), a diferencia de los procesadores de texto como Microsoft Word o Pages de Apple o incluso LibreOffice en el mundo open-source. En lugar de ello, un documento escrito para \LaTeX{} es en realidad un archivo de texto simple o llano que \emph{no contiene formato} . Nosotros le decimos a \LaTeX{} cómo deseamos que se aplique el formato en el documento final escribiendo comandos simples entre el texto, por ejemplo, si quiero usar texto en itálicas para dar énfasis, escribo \verb|\it{texto}| y pongo el texto que quiero en itálicas entre medio de las llaves. Esto significa que \LaTeX{} es un lenguaje del tipo \enquote{mark-up}, muy parecido a HTML.

Con una población mundial que supera los 8.000 millones de personas y continúa en crecimiento, la agricultura enfrenta el desafío de volverse más eficiente y sostenible. En este contexto, los cultivos verticales emergen como una solución innovadora que optimiza el uso del espacio y los recursos, especialmente en entornos urbanos, donde el suelo es limitado y costoso.

Este tipo de agricultura emplea técnicas como la hidroponía, que permite un uso preciso del agua, y la aeroponía, que maximiza la oxigenación de las raíces. Además, las granjas verticales aprovechan áreas infrautilizadas, como edificios abandonados o naves industriales, permitiendo el cultivo de alimentos en zonas donde la agricultura tradicional resulta difícil de implementar. De este modo, se logra una mayor densidad de cultivo y se contribuye a la sostenibilidad, reduciendo el uso de pesticidas y fertilizantes.

Aunque los cultivos verticales requieren altos niveles de tecnología y conllevan ciertos costos energéticos, su capacidad para optimizar recursos, reducir la huella de carbono y fomentar la producción local y el autoconsumo los posiciona como una alternativa clave para la agricultura del futuro.

Este método, también conocido como agricultura urbana, permite el crecimiento de plantas en interiores sin necesidad de luz solar directa, gracias a la aplicación de tecnologías agrícolas avanzadas. Estas optimizan las condiciones de cultivo, facilitando la propagación de plantas jóvenes y la producción de alimentos más saludables sin el uso de pesticidas.

Además, los cultivos verticales maximizan la producción al emplear sistemas de iluminación especializados de nueva generación, diseñados para estructuras multicapa. Gracias a esta tecnología, es posible obtener mayores rendimientos en un espacio reducido, consolidando a la agricultura vertical como una solución sostenible y eficiente para la producción de alimentos en entornos urbanos y con recursos limitados.

\subsection{Tipos de cultivos verticales}

%https://agroquivir.com/agricultura-vertical-como-funciona-y-tipos/

%Si sos nuevo en \LaTeX{}, hay un muy buen libro electrónico - disponible gratuitamente en Internet como un archivo PDF - llamado, \enquote{A (not so short) Introduction to \LaTeX{}}. El título del libro es generalmente acortado a simplemente \emph{lshort}. Puede descargar la versión más reciente en inglés (ya que se actualiza de vez en cuando) desde aquí:
%\url{http://www.ctan.org/tex-archive/info/lshort/english/lshort.pdf}

%Se puede encontrar la versión en español en la lista en esta página: \url{http://www.ctan.org/tex-archive/info/lshort/}

Existen distintos tipos de agricultura vertical, que optimizan el crecimiento de las plantas sin grandes extensiones de suelo y emplean estructuras especializadas para maximizar la producción.

\subsubsection{Hidroponía}
La hidroponía es un método de cultivo sin suelo que suministra nutrientes a las plantas a través de una solución acuosa, con raíces en un medio inerte como lana de roca o perlita. Es un sistema eficiente en el uso de agua, permite un control preciso de nutrientes y favorece una mayor densidad de cultivo y un crecimiento más rápido que los métodos tradicionales.
\subsubsection{Aeroponía}
La aeroponía es un método de cultivo en el que las raíces de las plantas quedan suspendidas en el aire y se rocían con una solución nutritiva. Maximiza la oxigenación, acelera el crecimiento y optimiza el uso de nutrientes, aunque requiere un control preciso para evitar la desecación de las raíces.
\subsubsection{Acuaponía}
La acuaponía combina la cría de peces con la hidroponía, utilizando los desechos de los peces como fertilizante para las plantas, mientras estas purifican el agua. Es un sistema sostenible y eficiente que minimiza el desperdicio, promueve la biodiversidad y es viable en entornos urbanos y rurales.

Acá tiene un ejemplo de una ``subsubsección'' que es el cuarto nivel de ordenamiento del texto, después de capítulo, sección y subsección.  Como se puede ver, las subsubsecciones no van numeradas en el cuerpo del documento ni en el índice.  El formato está definido por la plantilla y no debe ser modificado.

\subsection{Sistemas de agricultura vertical más conocidos}
Los sistemas de agricultura vertical varían según el tipo de cultivo y la técnica más adecuada para cada necesidad.

\subsubsection{Sistemas de torre}
Las torres verticales permiten el cultivo de plantas en estructuras cilíndricas apiladas, maximizando el uso del espacio y facilitando tanto la recolección como el mantenimiento.
Este sistema es ideal para cultivos de hoja verde y hierbas, y se integra fácilmente en espacios urbanos como balcones y terrazas.

\subsubsection{Sistemas en rack o estanterías}
Estos sistemas emplean estanterías apiladas donde las plantas crecen en bandejas o contenedores, siendo ideales para invernaderos y ambientes controlados. Facilitan el riego, la recolección y el monitoreo, optimizando el uso del espacio.
Los sistemas en rack o estanterías son especialmente útiles para cultivos que requieren diferentes niveles de luz y humedad.

\subsubsection{Sistemas montados en pared}
En estos sistemas, las plantas crecen en módulos montados en paredes, aprovechando superficies verticales en entornos urbanos. Su integración en la arquitectura no solo optimiza el espacio, sino que también aporta beneficios estéticos y funcionales.
Los jardines verticales pueden mejorar la calidad del aire y contribuir al aislamiento térmico de los edificios.

\subsubsection{Sistemas de bastidor en A}
El bastidor en forma de “A” proporciona una estructura estable y accesible para el cultivo vertical, facilitando el acceso a la luz y el riego.
Este diseño es particularmente útil para cultivos que necesitan un soporte adicional, como tomates y pepinos, y puede ser utilizado tanto en interiores como en exteriores.

\subsubsection{Sistemas de contenedores}
Estos sistemas utilizan contenedores apilables que pueden moverse y configurarse según las necesidades, adaptándose a diversos entornos y tipos de cultivo.
Los contenedores son ideales para cultivos modulares y pueden ser personalizados para optimizar el uso del espacio y los recursos.
%Si estás escribiendo un documento con mucho contenido matemático, entonces es posible que desees leer el documento de la AMS (American Mathematical Society) llamado, \enquote{A Short Math Guide for \LaTeX{}}. Se puede encontrar en línea en el siguiente link: \url{http://www.ams.org/tex/amslatex.html} en la sección \enquote{Additional Documentation} hacia la parte inferior de la página.


%----------------------------------------------------------------------------------------

\section{Motivación}

Este trabajo surge como un desarrollo de interés personal, impulsado por el deseo de contribuir positivamente al desarrollo sustentable y la optimización de los recursos naturales en la producción de alimentos. En la actualidad, el crecimiento de la población mundial y el aumento en la demanda de productos agrícolas presentan grandes desafíos ambientales y económicos.

La producción tradicional de alimentos enfrenta dificultades debido a la explotación intensiva de suelos, el uso excesivo de fertilizantes y pesticidas, y el desperdicio de recursos esenciales como el agua. Con el fin de maximizar sus ganancias, muchas empresas recurren a métodos poco sostenibles, los cuales generan un impacto ambiental negativo, afectando la biodiversidad, la calidad del suelo y el bienestar de las comunidades locales que dependen de estas fuentes de producción.

Uno de los problemas más críticos es el desperdicio de agua en la agricultura tradicional. Se estima que casi el 90 \% del agua utilizada en el riego se pierde por evaporación, drenaje o absorción ineficiente. En un contexto donde la crisis hídrica es una amenaza global, desarrollar sistemas más eficientes y sostenibles se vuelve una necesidad urgente.

Por otro lado, existe una creciente preocupación entre los consumidores sobre la calidad y procedencia de los alimentos. Cada vez más personas optan por un estilo de vida más saludable, adoptando hábitos de consumo responsables, como el vegetarianismo, el veganismo y el autocultivo. Sin embargo, muchos de estos consumidores no cuentan con espacios adecuados ni conocimientos técnicos para producir sus propios alimentos, lo que genera una gran oportunidad para soluciones tecnológicas que faciliten la agricultura doméstica o urbana.

A partir de estas problemáticas, nace la motivación para desarrollar un sistema hidropónico inteligente, que no solo optimice el uso del agua y los nutrientes, sino que también permita una gestión eficiente del cultivo con mínima intervención del usuario. Este trabajo ofrece una alternativa viable y escalable para la producción de alimentos frescos, reduciendo la dependencia de sistemas agrícolas tradicionales y promoviendo un modelo más sustentable para el futuro.


\section{Estado del arte}
Durante la etapa de investigación, se llevó a cabo una búsqueda de productos comerciales tanto en el mercado local como en el internacional. Se identificaron diversas soluciones con características similares al sistema desarrollado.

A continuación, se describen los productos identificados, los cuales presentan variaciones en cuanto a la tecnología utilizada.

\subsubsection{NIDO PRO}
NIDO PRO es un sistema hidropónico inteligente diseñado para cultivos verticales. Permite el control y la gestión de la solución nutritiva, estableciendo valores de pH y conductividad eléctrica (CE) desde una aplicación móvil. Sus algoritmos realizan comprobaciones automáticas diarias para garantizar la estabilidad de la solución nutritiva. Además, puede gestionar hasta cuatro ranuras para fertilizantes líquidos y ajustes de pH.
%https://www.agrointec.com/producto/nidopro-sistema-hidroponico-inteligente-cultivo-vertical/?utm_source=chatgpt.com

\subsubsection{Xiaomi Mi Flower Sensor}
Xiaomi Mi Flower Care Plant Sensor permite monitorear variables ambientales como luz, humedad, nutrientes y temperatura del sustrato. Estos sensores se conectan a aplicaciones móviles, facilitando el seguimiento y control de las condiciones del cultivo.
%https://elpais.com/tecnologia/tu-tecnologia/2024-12-07/dispositivos-tecnologicos-para-que-tus-plantas-sean-la-envidia-de-cualquiera.html?utm_source=chatgpt.com

\subsubsection{Kit de Riego Hydro de Konyks}
El Kit de Riego Hydro de Konyks ofrece control remoto del riego, permitiendo regular el caudal y programar horarios mediante una aplicación móvil. Esto asegura un suministro de agua adecuado y constante para los cultivos hidropónicos.
%https://elpais.com/tecnologia/tu-tecnologia/2024-12-07/dispositivos-tecnologicos-para-que-tus-plantas-sean-la-envidia-de-cualquiera.html?utm_source=chatgpt.com

\subsubsection{Smart 9 Pro de Click \& Grow}
El Smart 9 Pro de Click \& Grow es un jardín inteligente que facilita el cultivo de hortalizas y verduras en espacios interiores. Cuenta con sistemas de riego automático y luz artificial, proporcionando un entorno controlado y óptimo para el crecimiento de las plantas.
%https://elpais.com/tecnologia/tu-tecnologia/2024-12-07/dispositivos-tecnologicos-para-que-tus-plantas-sean-la-envidia-de-cualquiera.html?utm_source=chatgpt.com

\section{Objetivos y alcances}


\subsection{Objetivos}
\begin{itemize}
    \item Desarrollar un prototipo de sistema hidropónico automatizado para monitoreo y gestión eficiente del cultivo.
    \item Implementar sensores para medir temperatura, humedad, pH, conductividad eléctrica, nivel de agua e intensidad lumínica.
    \item Optimizar el uso de recursos, reduciendo el consumo de agua y nutrientes mediante control automatizado.
    \item Diseñar una interfaz web y móvil que permita la supervisión y configuración remota del sistema.
    \item Evaluar el rendimiento del sistema en comparación con métodos tradicionales, midiendo eficiencia y productividad.
\end{itemize}

\subsection{Alcances}

\begin{itemize}
    \item El prototipo permitirá el monitoreo en tiempo real y la gestión automatizada del riego, iluminación y ventilación.
    \item Se desarrollará un módulo de comunicación IoT, asegurando conectividad con dispositivos móviles y servidores locales.
    \item La solución se enfocará en cultivos hidropónicos verticales de pequeña y mediana escala.
    \item Se realizarán pruebas de funcionamiento y validación en un entorno controlado, pero no se contempla una implementación comercial en esta etapa.
\end{itemize}

%Si estás familiarizado con \LaTeX{}, entonces podés explorar la estructura de directorios de esta plantilla y proceder a personalizarla agregando tu información en el bloque \emph{INFORMACIÓN DE LA PORTADA} en el archivo \file{memoria.tex}.  

%Se puede continuar luego modificando el resto de los archivos siguiendo los lineamientos que se describen en la sección \ref{sec:FillingFile} en la página \pageref{sec:FillingFile}.

%Debés asegurarte de leer el capítulo \ref{Chapter2} acerca de las convenciones utilizadas para las Memoria de los Trabajos Finales de la \degreename.

%Si sos nuevo en \LaTeX{}, se recomienda que continúes leyendo el documento ya que contiene información básica para aprovechar el potencial de esta herramienta.


%----------------------------------------------------------------------------------------

\section{Qué incluye esta plantilla}

\subsection{Carpetas}

Esta plantilla se distribuye como una único archivo .zip que se puede descomprimir en varios archivos y carpetas. Asimismo, se puede consultar el repositorio git para obtener la última versión de los archivos, \url{https://github.com/patriciobos/Plantilla-CESE.git}. Los nombres de las carpetas son, o pretender ser, auto-explicativos.

\keyword{Appendices} -- Esta es la carpeta donde se deben poner los apéndices. Cada apéndice debe ir en su propio archivo \file{.tex}. Se incluye un ejemplo y una plantilla en la carpeta.

\keyword{Chapters} -- Esta es la carpeta donde se deben poner los capítulos de la memoria. Cada capítulo debe ir un su propio archivo \file{.tex} por separado.  Se ofrece por defecto, la siguiente estructura de capítulos y se recomienda su utilización dentro de lo posible:

\begin{itemize}
\item Capítulo 1: Introducción general	
\item Capítulo 2: Introducción específica
\item Capítulo 3: Diseño e implementación
\item Capítulo 4: Ensayos y resultados
\item Capítulo 5: Conclusiones

\end{itemize}

Esta estructura de capítulos es la que se recomienda para las memorias de la especialización.

\keyword{Figures} -- Esta carpeta contiene todas las figuras de la memoria.  Estas son las versiones finales de las imágenes que van a ser incluidas en la memoria.  Pueden ser imágenes en formato \textit{raster}\footnote{\url{https://en.wikipedia.org/wiki/Raster_graphics}} como \file{.png}, \file{.jpg} o en formato vectoriales\footnote{\url{https://en.wikipedia.org/wiki/Vector_graphics}} como \file{.pdf}, \file{.ps}.  Se debe notar que utilizar imágenes vectoriales disminuye notablemente el peso del documento final y acelera el tiempo de compilación por lo que es recomendable su utilización siempre que sea posible.

\subsection{Archivos}

También están incluidos varios archivos, la mayoría de ellos son de texto plano y se puede ver su contenido en un editor de texto. Después de la compilación inicial, se verá que más archivos auxiliares son creados por \ LaTeX{} o BibTeX, pero son de uso interno y no es necesario hacer nada en particular con ellos.  Toda la información necesaria para compilar el documento se encuentra en los archivos \file{.tex}, \file{.bib}, \file{.cls} y en las imágenes de la carpeta Figures.

\keyword{referencias.bib} - este es un archivo importante que contiene toda la información de referencias bibliográficas que se utilizarán para las citas en la memoria en conjunto con BibTeX. Usted puede escribir las entradas bibliográficas en forma manual, aunque existen también programas de gestión de referencias que facilitan la creación y gestión de las referencias y permiten exportarlas en formato BibTeX.  También hay disponibles sitios web como \url{books.google.com} que permiten obtener toda la información necesaria para una cita en formato BibTeX. Ver sección \ref{sec:biblio}

\keyword{MastersDoctoralThesis.cls} -- este es un archivo importante. Es el archivos con la clase que le informa a \LaTeX{} cómo debe dar formato a la memoria. El usuario de la plantilla no debería necesitar modificar nada de este archivo.

\keyword{memoria.pdf} -- esta es su memoria con una tipografía bellamente compuesta (en formato de archivo PDF) creado por \LaTeX{}. Se distribuye con la plantilla y después de compilar por primera vez sin hacer ningún cambio se debería obtener una versión idéntica a este documento.

\keyword{memoria.tex} -- este es un archivo importante. Este es el archivo que tiene que compilar \LaTeX{} para producir la memoria como un archivo PDF. Contiene un marco de trabajo y estructuras que le indican a \LaTeX{} cómo diagramar la memoria.  Está altamente comentado para que se pueda entender qué es lo que realiza cada línea de código y por qué está incluida en ese lugar.  En este archivo se debe completar la información personalizada de las primeras sección según se indica en la sección \ref{sec:FillingFile}.

Archivos que \emph{no} forman parte de la distribución de la plantilla pero que son generados por \LaTeX{} como archivos auxiliares necesarios para la producción de la memoria.pdf son:

\keyword{memoria.aux} -- este es un archivo auxiliar generado por \LaTeX{}, si se borra \LaTeX{} simplemente lo regenera cuando se compila el archivo principal \file{memoria.tex}.

\keyword{memoria.bbl} -- este es un archivo auxiliar generado por BibTeX, si se borra BibTeX simplemente lo regenera cuando se compila el archivo principal \file{memoria.tex}. Mientras que el archivo \file{.bib} contiene todas las referencias que hay, este archivo \file{.bbl} contine sólo las referencias que han sido citadas y se utiliza para la construcción de la bibiografía.

\keyword{memoria.blg} -- este es un archivo auxiliar generado por BibTeX, si se borra BibTeX simplemente lo regenera cuando se compila el archivo principal \file{memoria.tex}.

\keyword{memoria.lof} -- este es un archivo auxiliar generado por \LaTeX{}, si se borra \LaTeX{} simplemente lo regenera cuando se compila el archivo principal \file{memoria.tex}.  Le indica a \LaTeX{} cómo construir la sección \emph{Lista de Figuras}.
 
\keyword{memoria.log} --  este es un archivo auxiliar generado por \LaTeX{}, si se borra \LaTeX{} simplemente lo regenera cuando se compila el archivo principal \file{memoria.tex}. Contiene mensajes de \LaTeX{}. Si se reciben errores o advertencias durante la compilación, se guardan en este archivo \file{.log}.

\keyword{memoria.lot} -- este es un archivo auxiliar generado por \LaTeX{}, si se borra \LaTeX{} simplemente lo regenera cuando se compila el archivo principal \file{memoria.tex}.  Le indica a \LaTeX{} cómo construir la sección \emph{Lista de Tablas}.

\keyword{memoria.out} -- este es un archivo auxiliar generado por \LaTeX{}, si se borra \LaTeX{} simplemente lo regenera cuando se compila el archivo principal \file{memoria.tex}.

De esta larga lista de archivos, sólo aquellos con la extensión \file{.bib}, \file{.cls} y \file{.tex} son importantes.  Los otros archivos auxiliares pueden ser ignorados o borrados ya que \LaTeX{} y BibTeX los regenerarán durante la compilación.

%----------------------------------------------------------------------------------------

\section{Entorno de trabajo}

Ante de comenzar a editar la plantilla debemos tener un editor \LaTeX{} instalado en nuestra computadora.  En forma análoga a lo que sucede en lenguaje C, que se puede crear y editar código con casi cualquier editor, existen ciertos entornos de trabajo que nos pueden simplificar mucho la tarea.  En este sentido, se recomienda, sobre todo para los principiantes en \LaTeX{} la utilización de TexMaker, un programa gratuito y multi-plantaforma que está disponible tanto para windows como para sistemas GNU/linux.

La versión más reciente de TexMaker es la 4.5 y se puede descargar del siguiente link: \url{http://www.xm1math.net/texmaker/download.html}. Se puede consultar el manual de usuario en el siguiente link: \url{http://www.xm1math.net/texmaker/doc.html}.
 

\subsection{Paquetes adicionales}

Si bien durante el proceso de instalación de TexMaker, o cualquier otro editor que se haya elegido, se instalarán en el sistema los paquetes básicos necesarios para trabajar con \LaTeX{}, la plantilla de los trabajos de Especialización y Maestría requieren de paquete adicionales.

Se indican a continuación los comandos que se deben introducir en la consola de Ubuntu (ctrl + alt + t) para instalarlos:

\begin{lstlisting}[language=bash]
  $ sudo apt install texlive-lang-spanish texlive-science 
  $ sudo apt install texlive-bibtex-extra biber
  $ sudo apt install texlive texlive-fonts-recommended
  $ sudo apt install texlive-latex-extra
\end{lstlisting}


\subsection{Configurando TexMaker}
\label{subsec:configurando}



Una vez instalado el programa y los paquetes adicionales se debe abrir el archivo memoria.tex con el editor para ver una pantalla similar a la que se puede apreciar en la figura \ref{fig:texmaker}. 
Una vez instalado el programa y los paquetes adicionales se debe abrir el archivo memoria.tex con el editor para ver una pantalla similar a la que se puede apreciar en la figura \ref{fig:texmaker}. 
Una vez instalado el programa y los paquetes adicionales se debe abrir el archivo memoria.tex con el editor para ver una pantalla similar a la que se puede apreciar en la figura \ref{fig:texmaker}. 
Una vez instalado el programa y los paquetes adicionales se debe abrir el archivo memoria.tex con el editor para ver una pantalla similar a la que se puede apreciar en la figura \ref{fig:texmaker}. 

\vspace{1cm}

\begin{figure}[htbp]
	\centering
	\includegraphics[width=.5\textwidth]{./Figures/texmaker.png}
	\caption{Entorno de trabajo de texMaker.}
	\label{fig:texmaker}
\end{figure}

\vspace{1cm}

Notar que existe una vista llamada Estructura a la izquierda de la interfaz que nos permite abrir desde dentro del programa los archivos individuales de los capítulos.  A la derecha se encuentra una vista con el archivo propiamente dicho para su edición. Hacia la parte inferior se encuentra una vista del log con información de los resultados de la compilación.  En esta última vista pueden aparecen advertencias o \textit{warning}, que normalmente pueden ser ignorados, y los errores que se indican en color rojo y deben resolverse para que se genere el PDF de salida.

Recordar que el archivo que se debe compilar con PDFLaTeX es \file{memoria.tex}, si se tratara de compilar alguno de los capítulos saldría un error.  Para salvar la molestia de tener que cambiar de archivo para compilar cada vez que se realice una modificación en un capítulo, se puede definir el archivo \file{memoria.tex} como ``documento maestro'' yendo al menú opciones -> ``definir documento actual como documento maestro'', lo que permite compilar con PDFLaTeX memoria.tex directamente desde cualquier archivo que se esté modificando . Se muestra esta opción en la figura \ref{fig:docMaestro}.

\begin{figure}[h]
	\centering
	\includegraphics[width=\textwidth]{./Figures/docMaestro.png}
	\caption{Definir memoria.tex como documento maestro.}
	\label{fig:docMaestro}
\end{figure}

En el menú herramientas se encuentran las opciones de compilación.  Para producir un archivo PDF a partir de un archivo .tex se debe ejecutar PDFLaTeX (el shortcut es F6). Para incorporar nueva bibliografía se debe utilizar la opción BibTeX del mismo menú herramientas (el shortcut es F11).

Notar que para actualizar las tablas de contenidos se debe ejecutar PDFLaTeX dos veces.  Esto se debe a que es necesario actualizar algunos archivos auxiliares antes de obtener el resultado final.  En forma similar, para actualizar las referencias bibliográficas se debe ejecutar primero PDFLaTeX, después BibTeX y finalmente PDFLaTeX dos veces por idénticos motivos.

\section{Personalizando la plantilla, el archivo \file{memoria.tex}}
\label{sec:FillingFile}

Para personalizar la plantilla se debe incorporar la información propia en los distintos archivos \file{.tex}. 

Primero abrir \file{memoria.tex} con TexMaker (o el editor de su preferencia). Se debe ubicar dentro del archivo el bloque de código titulado \emph{INFORMACIÓN DE LA PORTADA} donde se deben incorporar los primeros datos personales con los que se construirá automáticamente la portada.


%----------------------------------------------------------------------------------------

\section{El código del archivo \file{memoria.tex} explicado}

El archivo \file{memoria.tex} contiene la estructura del documento y es el archivo de mayor jerarquía de la memoria.  Podría ser equiparable a la función \emph{main()} de un programa en C, o mejor dicho al archivo fuente .c donde se encuentra definida la función main().

La estructura básica de cualquier documento de \LaTeX{} comienza con la definición de clase del documento, es seguida por un preámbulo donde se pueden agregar funcionalidades con el uso de \texttt{paquetes} (equiparables a bibliotecas de C), y finalmente, termina con el cuerpo del documento, donde irá el contenido de la memoria.

\lstset{%
  basicstyle=\small\ttfamily,
  language=[LaTeX]{TeX}
}

\begin{lstlisting}
\documentclass{article}  <- Definicion de clase
\usepackage{listings}	 <- Preambulo

\begin{document}	 <- Comienzo del contenido propio 
	Hello world!
\end{document}
\end{lstlisting}


El archivo \file{memoria.tex} se encuentra densamente comentado para explicar qué páginas, secciones y elementos de formato está creando el código \LaTeX{} en cada línea. El código está dividido en bloques con nombres en mayúsculas para que resulte evidente qué es lo que hace esa porción de código en particular. Inicialmente puede parecer que hay mucho código \LaTeX{}, pero es principalmente código para dar formato a la memoria por lo que no requiere intervención del usuario de la plantilla.  Sí se deben personalizar con su información los bloques indicados como:

\begin{itemize}
	\item Informacion de la memoria
	\item Resumen
	\item Agradecimientos
	\item Dedicatoria
\end{itemize}

El índice de contenidos, las listas de figura de tablas se generan en forma automática y no requieren intervención ni edición manual por parte del usuario de la plantilla. 

En la parte final del documento se encuentran los capítulos y los apéndices.  Por defecto se incluyen los 5 capítulos propuestos que se encuentran en la carpeta /Chapters. Cada capítulo se debe escribir en un archivo .tex separado y se debe poner en la carpeta \emph{Chapters} con el nombre \file{Chapter1}, \file{Chapter2}, etc\ldots El código para incluir capítulos desde archivos externos se muestra a continuación.

\begin{verbatim}
	% Chapter 1

\chapter{Introducción general} % Main chapter title

\label{Chapter1} % For referencing the chapter elsewhere, use \ref{Chapter1} 
\label{IntroGeneral}

En este capítulo se hace una breve introducción a la necesidad que condujo al
desarrollo del trabajo. Se presenta el concepto de cultivo vertical, y el estado del arte de su integración con la tecnología. Asimismo, se explica el objetivo y los alcances del trabajo.

%----------------------------------------------------------------------------------------

% Define some commands to keep the formatting separated from the content 
\newcommand{\keyword}[1]{\textbf{#1}}
\newcommand{\tabhead}[1]{\textbf{#1}}
\newcommand{\code}[1]{\texttt{#1}}
\newcommand{\file}[1]{\texttt{\bfseries#1}}
\newcommand{\option}[1]{\texttt{\itshape#1}}
\newcommand{\grados}{$^{\circ}$}

%----------------------------------------------------------------------------------------

%\section{Introducción}

%----------------------------------------------------------------------------------------
\section{Introducción a los cultivos verticales}

%\LaTeX{} no es \textsc{WYSIWYG} (What You See is What You Get), a diferencia de los procesadores de texto como Microsoft Word o Pages de Apple o incluso LibreOffice en el mundo open-source. En lugar de ello, un documento escrito para \LaTeX{} es en realidad un archivo de texto simple o llano que \emph{no contiene formato} . Nosotros le decimos a \LaTeX{} cómo deseamos que se aplique el formato en el documento final escribiendo comandos simples entre el texto, por ejemplo, si quiero usar texto en itálicas para dar énfasis, escribo \verb|\it{texto}| y pongo el texto que quiero en itálicas entre medio de las llaves. Esto significa que \LaTeX{} es un lenguaje del tipo \enquote{mark-up}, muy parecido a HTML.

Con una población mundial que supera los 8.000 millones de personas y continúa en crecimiento, la agricultura enfrenta el desafío de volverse más eficiente y sostenible. En este contexto, los cultivos verticales emergen como una solución innovadora que optimiza el uso del espacio y los recursos, especialmente en entornos urbanos, donde el suelo es limitado y costoso.

Este tipo de agricultura emplea técnicas como la hidroponía, que permite un uso preciso del agua, y la aeroponía, que maximiza la oxigenación de las raíces. Además, las granjas verticales aprovechan áreas infrautilizadas, como edificios abandonados o naves industriales, permitiendo el cultivo de alimentos en zonas donde la agricultura tradicional resulta difícil de implementar. De este modo, se logra una mayor densidad de cultivo y se contribuye a la sostenibilidad, reduciendo el uso de pesticidas y fertilizantes.

Aunque los cultivos verticales requieren altos niveles de tecnología y conllevan ciertos costos energéticos, su capacidad para optimizar recursos, reducir la huella de carbono y fomentar la producción local y el autoconsumo los posiciona como una alternativa clave para la agricultura del futuro.

Este método, también conocido como agricultura urbana, permite el crecimiento de plantas en interiores sin necesidad de luz solar directa, gracias a la aplicación de tecnologías agrícolas avanzadas. Estas optimizan las condiciones de cultivo, facilitando la propagación de plantas jóvenes y la producción de alimentos más saludables sin el uso de pesticidas.

Además, los cultivos verticales maximizan la producción al emplear sistemas de iluminación especializados de nueva generación, diseñados para estructuras multicapa. Gracias a esta tecnología, es posible obtener mayores rendimientos en un espacio reducido, consolidando a la agricultura vertical como una solución sostenible y eficiente para la producción de alimentos en entornos urbanos y con recursos limitados.

\subsection{Tipos de cultivos verticales}

%https://agroquivir.com/agricultura-vertical-como-funciona-y-tipos/

%Si sos nuevo en \LaTeX{}, hay un muy buen libro electrónico - disponible gratuitamente en Internet como un archivo PDF - llamado, \enquote{A (not so short) Introduction to \LaTeX{}}. El título del libro es generalmente acortado a simplemente \emph{lshort}. Puede descargar la versión más reciente en inglés (ya que se actualiza de vez en cuando) desde aquí:
%\url{http://www.ctan.org/tex-archive/info/lshort/english/lshort.pdf}

%Se puede encontrar la versión en español en la lista en esta página: \url{http://www.ctan.org/tex-archive/info/lshort/}

Existen distintos tipos de agricultura vertical, que optimizan el crecimiento de las plantas sin grandes extensiones de suelo y emplean estructuras especializadas para maximizar la producción.

\subsubsection{Hidroponía}
La hidroponía es un método de cultivo sin suelo que suministra nutrientes a las plantas a través de una solución acuosa, con raíces en un medio inerte como lana de roca o perlita. Es un sistema eficiente en el uso de agua, permite un control preciso de nutrientes y favorece una mayor densidad de cultivo y un crecimiento más rápido que los métodos tradicionales.
\subsubsection{Aeroponía}
La aeroponía es un método de cultivo en el que las raíces de las plantas quedan suspendidas en el aire y se rocían con una solución nutritiva. Maximiza la oxigenación, acelera el crecimiento y optimiza el uso de nutrientes, aunque requiere un control preciso para evitar la desecación de las raíces.
\subsubsection{Acuaponía}
La acuaponía combina la cría de peces con la hidroponía, utilizando los desechos de los peces como fertilizante para las plantas, mientras estas purifican el agua. Es un sistema sostenible y eficiente que minimiza el desperdicio, promueve la biodiversidad y es viable en entornos urbanos y rurales.

Acá tiene un ejemplo de una ``subsubsección'' que es el cuarto nivel de ordenamiento del texto, después de capítulo, sección y subsección.  Como se puede ver, las subsubsecciones no van numeradas en el cuerpo del documento ni en el índice.  El formato está definido por la plantilla y no debe ser modificado.

\subsection{Sistemas de agricultura vertical más conocidos}
Los sistemas de agricultura vertical varían según el tipo de cultivo y la técnica más adecuada para cada necesidad.

\subsubsection{Sistemas de torre}
Las torres verticales permiten el cultivo de plantas en estructuras cilíndricas apiladas, maximizando el uso del espacio y facilitando tanto la recolección como el mantenimiento.
Este sistema es ideal para cultivos de hoja verde y hierbas, y se integra fácilmente en espacios urbanos como balcones y terrazas.

\subsubsection{Sistemas en rack o estanterías}
Estos sistemas emplean estanterías apiladas donde las plantas crecen en bandejas o contenedores, siendo ideales para invernaderos y ambientes controlados. Facilitan el riego, la recolección y el monitoreo, optimizando el uso del espacio.
Los sistemas en rack o estanterías son especialmente útiles para cultivos que requieren diferentes niveles de luz y humedad.

\subsubsection{Sistemas montados en pared}
En estos sistemas, las plantas crecen en módulos montados en paredes, aprovechando superficies verticales en entornos urbanos. Su integración en la arquitectura no solo optimiza el espacio, sino que también aporta beneficios estéticos y funcionales.
Los jardines verticales pueden mejorar la calidad del aire y contribuir al aislamiento térmico de los edificios.

\subsubsection{Sistemas de bastidor en A}
El bastidor en forma de “A” proporciona una estructura estable y accesible para el cultivo vertical, facilitando el acceso a la luz y el riego.
Este diseño es particularmente útil para cultivos que necesitan un soporte adicional, como tomates y pepinos, y puede ser utilizado tanto en interiores como en exteriores.

\subsubsection{Sistemas de contenedores}
Estos sistemas utilizan contenedores apilables que pueden moverse y configurarse según las necesidades, adaptándose a diversos entornos y tipos de cultivo.
Los contenedores son ideales para cultivos modulares y pueden ser personalizados para optimizar el uso del espacio y los recursos.
%Si estás escribiendo un documento con mucho contenido matemático, entonces es posible que desees leer el documento de la AMS (American Mathematical Society) llamado, \enquote{A Short Math Guide for \LaTeX{}}. Se puede encontrar en línea en el siguiente link: \url{http://www.ams.org/tex/amslatex.html} en la sección \enquote{Additional Documentation} hacia la parte inferior de la página.


%----------------------------------------------------------------------------------------

\section{Motivación}

Este trabajo surge como un desarrollo de interés personal, impulsado por el deseo de contribuir positivamente al desarrollo sustentable y la optimización de los recursos naturales en la producción de alimentos. En la actualidad, el crecimiento de la población mundial y el aumento en la demanda de productos agrícolas presentan grandes desafíos ambientales y económicos.

La producción tradicional de alimentos enfrenta dificultades debido a la explotación intensiva de suelos, el uso excesivo de fertilizantes y pesticidas, y el desperdicio de recursos esenciales como el agua. Con el fin de maximizar sus ganancias, muchas empresas recurren a métodos poco sostenibles, los cuales generan un impacto ambiental negativo, afectando la biodiversidad, la calidad del suelo y el bienestar de las comunidades locales que dependen de estas fuentes de producción.

Uno de los problemas más críticos es el desperdicio de agua en la agricultura tradicional. Se estima que casi el 90 \% del agua utilizada en el riego se pierde por evaporación, drenaje o absorción ineficiente. En un contexto donde la crisis hídrica es una amenaza global, desarrollar sistemas más eficientes y sostenibles se vuelve una necesidad urgente.

Por otro lado, existe una creciente preocupación entre los consumidores sobre la calidad y procedencia de los alimentos. Cada vez más personas optan por un estilo de vida más saludable, adoptando hábitos de consumo responsables, como el vegetarianismo, el veganismo y el autocultivo. Sin embargo, muchos de estos consumidores no cuentan con espacios adecuados ni conocimientos técnicos para producir sus propios alimentos, lo que genera una gran oportunidad para soluciones tecnológicas que faciliten la agricultura doméstica o urbana.

A partir de estas problemáticas, nace la motivación para desarrollar un sistema hidropónico inteligente, que no solo optimice el uso del agua y los nutrientes, sino que también permita una gestión eficiente del cultivo con mínima intervención del usuario. Este trabajo ofrece una alternativa viable y escalable para la producción de alimentos frescos, reduciendo la dependencia de sistemas agrícolas tradicionales y promoviendo un modelo más sustentable para el futuro.


\section{Estado del arte}
Durante la etapa de investigación, se llevó a cabo una búsqueda de productos comerciales tanto en el mercado local como en el internacional. Se identificaron diversas soluciones con características similares al sistema desarrollado.

A continuación, se describen los productos identificados, los cuales presentan variaciones en cuanto a la tecnología utilizada.

\subsubsection{NIDO PRO}
NIDO PRO es un sistema hidropónico inteligente diseñado para cultivos verticales. Permite el control y la gestión de la solución nutritiva, estableciendo valores de pH y conductividad eléctrica (CE) desde una aplicación móvil. Sus algoritmos realizan comprobaciones automáticas diarias para garantizar la estabilidad de la solución nutritiva. Además, puede gestionar hasta cuatro ranuras para fertilizantes líquidos y ajustes de pH.
%https://www.agrointec.com/producto/nidopro-sistema-hidroponico-inteligente-cultivo-vertical/?utm_source=chatgpt.com

\subsubsection{Xiaomi Mi Flower Sensor}
Xiaomi Mi Flower Care Plant Sensor permite monitorear variables ambientales como luz, humedad, nutrientes y temperatura del sustrato. Estos sensores se conectan a aplicaciones móviles, facilitando el seguimiento y control de las condiciones del cultivo.
%https://elpais.com/tecnologia/tu-tecnologia/2024-12-07/dispositivos-tecnologicos-para-que-tus-plantas-sean-la-envidia-de-cualquiera.html?utm_source=chatgpt.com

\subsubsection{Kit de Riego Hydro de Konyks}
El Kit de Riego Hydro de Konyks ofrece control remoto del riego, permitiendo regular el caudal y programar horarios mediante una aplicación móvil. Esto asegura un suministro de agua adecuado y constante para los cultivos hidropónicos.
%https://elpais.com/tecnologia/tu-tecnologia/2024-12-07/dispositivos-tecnologicos-para-que-tus-plantas-sean-la-envidia-de-cualquiera.html?utm_source=chatgpt.com

\subsubsection{Smart 9 Pro de Click \& Grow}
El Smart 9 Pro de Click \& Grow es un jardín inteligente que facilita el cultivo de hortalizas y verduras en espacios interiores. Cuenta con sistemas de riego automático y luz artificial, proporcionando un entorno controlado y óptimo para el crecimiento de las plantas.
%https://elpais.com/tecnologia/tu-tecnologia/2024-12-07/dispositivos-tecnologicos-para-que-tus-plantas-sean-la-envidia-de-cualquiera.html?utm_source=chatgpt.com

\section{Objetivos y alcances}


\subsection{Objetivos}
\begin{itemize}
    \item Desarrollar un prototipo de sistema hidropónico automatizado para monitoreo y gestión eficiente del cultivo.
    \item Implementar sensores para medir temperatura, humedad, pH, conductividad eléctrica, nivel de agua e intensidad lumínica.
    \item Optimizar el uso de recursos, reduciendo el consumo de agua y nutrientes mediante control automatizado.
    \item Diseñar una interfaz web y móvil que permita la supervisión y configuración remota del sistema.
    \item Evaluar el rendimiento del sistema en comparación con métodos tradicionales, midiendo eficiencia y productividad.
\end{itemize}

\subsection{Alcances}

\begin{itemize}
    \item El prototipo permitirá el monitoreo en tiempo real y la gestión automatizada del riego, iluminación y ventilación.
    \item Se desarrollará un módulo de comunicación IoT, asegurando conectividad con dispositivos móviles y servidores locales.
    \item La solución se enfocará en cultivos hidropónicos verticales de pequeña y mediana escala.
    \item Se realizarán pruebas de funcionamiento y validación en un entorno controlado, pero no se contempla una implementación comercial en esta etapa.
\end{itemize}

%Si estás familiarizado con \LaTeX{}, entonces podés explorar la estructura de directorios de esta plantilla y proceder a personalizarla agregando tu información en el bloque \emph{INFORMACIÓN DE LA PORTADA} en el archivo \file{memoria.tex}.  

%Se puede continuar luego modificando el resto de los archivos siguiendo los lineamientos que se describen en la sección \ref{sec:FillingFile} en la página \pageref{sec:FillingFile}.

%Debés asegurarte de leer el capítulo \ref{Chapter2} acerca de las convenciones utilizadas para las Memoria de los Trabajos Finales de la \degreename.

%Si sos nuevo en \LaTeX{}, se recomienda que continúes leyendo el documento ya que contiene información básica para aprovechar el potencial de esta herramienta.


%----------------------------------------------------------------------------------------

\section{Qué incluye esta plantilla}

\subsection{Carpetas}

Esta plantilla se distribuye como una único archivo .zip que se puede descomprimir en varios archivos y carpetas. Asimismo, se puede consultar el repositorio git para obtener la última versión de los archivos, \url{https://github.com/patriciobos/Plantilla-CESE.git}. Los nombres de las carpetas son, o pretender ser, auto-explicativos.

\keyword{Appendices} -- Esta es la carpeta donde se deben poner los apéndices. Cada apéndice debe ir en su propio archivo \file{.tex}. Se incluye un ejemplo y una plantilla en la carpeta.

\keyword{Chapters} -- Esta es la carpeta donde se deben poner los capítulos de la memoria. Cada capítulo debe ir un su propio archivo \file{.tex} por separado.  Se ofrece por defecto, la siguiente estructura de capítulos y se recomienda su utilización dentro de lo posible:

\begin{itemize}
\item Capítulo 1: Introducción general	
\item Capítulo 2: Introducción específica
\item Capítulo 3: Diseño e implementación
\item Capítulo 4: Ensayos y resultados
\item Capítulo 5: Conclusiones

\end{itemize}

Esta estructura de capítulos es la que se recomienda para las memorias de la especialización.

\keyword{Figures} -- Esta carpeta contiene todas las figuras de la memoria.  Estas son las versiones finales de las imágenes que van a ser incluidas en la memoria.  Pueden ser imágenes en formato \textit{raster}\footnote{\url{https://en.wikipedia.org/wiki/Raster_graphics}} como \file{.png}, \file{.jpg} o en formato vectoriales\footnote{\url{https://en.wikipedia.org/wiki/Vector_graphics}} como \file{.pdf}, \file{.ps}.  Se debe notar que utilizar imágenes vectoriales disminuye notablemente el peso del documento final y acelera el tiempo de compilación por lo que es recomendable su utilización siempre que sea posible.

\subsection{Archivos}

También están incluidos varios archivos, la mayoría de ellos son de texto plano y se puede ver su contenido en un editor de texto. Después de la compilación inicial, se verá que más archivos auxiliares son creados por \ LaTeX{} o BibTeX, pero son de uso interno y no es necesario hacer nada en particular con ellos.  Toda la información necesaria para compilar el documento se encuentra en los archivos \file{.tex}, \file{.bib}, \file{.cls} y en las imágenes de la carpeta Figures.

\keyword{referencias.bib} - este es un archivo importante que contiene toda la información de referencias bibliográficas que se utilizarán para las citas en la memoria en conjunto con BibTeX. Usted puede escribir las entradas bibliográficas en forma manual, aunque existen también programas de gestión de referencias que facilitan la creación y gestión de las referencias y permiten exportarlas en formato BibTeX.  También hay disponibles sitios web como \url{books.google.com} que permiten obtener toda la información necesaria para una cita en formato BibTeX. Ver sección \ref{sec:biblio}

\keyword{MastersDoctoralThesis.cls} -- este es un archivo importante. Es el archivos con la clase que le informa a \LaTeX{} cómo debe dar formato a la memoria. El usuario de la plantilla no debería necesitar modificar nada de este archivo.

\keyword{memoria.pdf} -- esta es su memoria con una tipografía bellamente compuesta (en formato de archivo PDF) creado por \LaTeX{}. Se distribuye con la plantilla y después de compilar por primera vez sin hacer ningún cambio se debería obtener una versión idéntica a este documento.

\keyword{memoria.tex} -- este es un archivo importante. Este es el archivo que tiene que compilar \LaTeX{} para producir la memoria como un archivo PDF. Contiene un marco de trabajo y estructuras que le indican a \LaTeX{} cómo diagramar la memoria.  Está altamente comentado para que se pueda entender qué es lo que realiza cada línea de código y por qué está incluida en ese lugar.  En este archivo se debe completar la información personalizada de las primeras sección según se indica en la sección \ref{sec:FillingFile}.

Archivos que \emph{no} forman parte de la distribución de la plantilla pero que son generados por \LaTeX{} como archivos auxiliares necesarios para la producción de la memoria.pdf son:

\keyword{memoria.aux} -- este es un archivo auxiliar generado por \LaTeX{}, si se borra \LaTeX{} simplemente lo regenera cuando se compila el archivo principal \file{memoria.tex}.

\keyword{memoria.bbl} -- este es un archivo auxiliar generado por BibTeX, si se borra BibTeX simplemente lo regenera cuando se compila el archivo principal \file{memoria.tex}. Mientras que el archivo \file{.bib} contiene todas las referencias que hay, este archivo \file{.bbl} contine sólo las referencias que han sido citadas y se utiliza para la construcción de la bibiografía.

\keyword{memoria.blg} -- este es un archivo auxiliar generado por BibTeX, si se borra BibTeX simplemente lo regenera cuando se compila el archivo principal \file{memoria.tex}.

\keyword{memoria.lof} -- este es un archivo auxiliar generado por \LaTeX{}, si se borra \LaTeX{} simplemente lo regenera cuando se compila el archivo principal \file{memoria.tex}.  Le indica a \LaTeX{} cómo construir la sección \emph{Lista de Figuras}.
 
\keyword{memoria.log} --  este es un archivo auxiliar generado por \LaTeX{}, si se borra \LaTeX{} simplemente lo regenera cuando se compila el archivo principal \file{memoria.tex}. Contiene mensajes de \LaTeX{}. Si se reciben errores o advertencias durante la compilación, se guardan en este archivo \file{.log}.

\keyword{memoria.lot} -- este es un archivo auxiliar generado por \LaTeX{}, si se borra \LaTeX{} simplemente lo regenera cuando se compila el archivo principal \file{memoria.tex}.  Le indica a \LaTeX{} cómo construir la sección \emph{Lista de Tablas}.

\keyword{memoria.out} -- este es un archivo auxiliar generado por \LaTeX{}, si se borra \LaTeX{} simplemente lo regenera cuando se compila el archivo principal \file{memoria.tex}.

De esta larga lista de archivos, sólo aquellos con la extensión \file{.bib}, \file{.cls} y \file{.tex} son importantes.  Los otros archivos auxiliares pueden ser ignorados o borrados ya que \LaTeX{} y BibTeX los regenerarán durante la compilación.

%----------------------------------------------------------------------------------------

\section{Entorno de trabajo}

Ante de comenzar a editar la plantilla debemos tener un editor \LaTeX{} instalado en nuestra computadora.  En forma análoga a lo que sucede en lenguaje C, que se puede crear y editar código con casi cualquier editor, existen ciertos entornos de trabajo que nos pueden simplificar mucho la tarea.  En este sentido, se recomienda, sobre todo para los principiantes en \LaTeX{} la utilización de TexMaker, un programa gratuito y multi-plantaforma que está disponible tanto para windows como para sistemas GNU/linux.

La versión más reciente de TexMaker es la 4.5 y se puede descargar del siguiente link: \url{http://www.xm1math.net/texmaker/download.html}. Se puede consultar el manual de usuario en el siguiente link: \url{http://www.xm1math.net/texmaker/doc.html}.
 

\subsection{Paquetes adicionales}

Si bien durante el proceso de instalación de TexMaker, o cualquier otro editor que se haya elegido, se instalarán en el sistema los paquetes básicos necesarios para trabajar con \LaTeX{}, la plantilla de los trabajos de Especialización y Maestría requieren de paquete adicionales.

Se indican a continuación los comandos que se deben introducir en la consola de Ubuntu (ctrl + alt + t) para instalarlos:

\begin{lstlisting}[language=bash]
  $ sudo apt install texlive-lang-spanish texlive-science 
  $ sudo apt install texlive-bibtex-extra biber
  $ sudo apt install texlive texlive-fonts-recommended
  $ sudo apt install texlive-latex-extra
\end{lstlisting}


\subsection{Configurando TexMaker}
\label{subsec:configurando}



Una vez instalado el programa y los paquetes adicionales se debe abrir el archivo memoria.tex con el editor para ver una pantalla similar a la que se puede apreciar en la figura \ref{fig:texmaker}. 
Una vez instalado el programa y los paquetes adicionales se debe abrir el archivo memoria.tex con el editor para ver una pantalla similar a la que se puede apreciar en la figura \ref{fig:texmaker}. 
Una vez instalado el programa y los paquetes adicionales se debe abrir el archivo memoria.tex con el editor para ver una pantalla similar a la que se puede apreciar en la figura \ref{fig:texmaker}. 
Una vez instalado el programa y los paquetes adicionales se debe abrir el archivo memoria.tex con el editor para ver una pantalla similar a la que se puede apreciar en la figura \ref{fig:texmaker}. 

\vspace{1cm}

\begin{figure}[htbp]
	\centering
	\includegraphics[width=.5\textwidth]{./Figures/texmaker.png}
	\caption{Entorno de trabajo de texMaker.}
	\label{fig:texmaker}
\end{figure}

\vspace{1cm}

Notar que existe una vista llamada Estructura a la izquierda de la interfaz que nos permite abrir desde dentro del programa los archivos individuales de los capítulos.  A la derecha se encuentra una vista con el archivo propiamente dicho para su edición. Hacia la parte inferior se encuentra una vista del log con información de los resultados de la compilación.  En esta última vista pueden aparecen advertencias o \textit{warning}, que normalmente pueden ser ignorados, y los errores que se indican en color rojo y deben resolverse para que se genere el PDF de salida.

Recordar que el archivo que se debe compilar con PDFLaTeX es \file{memoria.tex}, si se tratara de compilar alguno de los capítulos saldría un error.  Para salvar la molestia de tener que cambiar de archivo para compilar cada vez que se realice una modificación en un capítulo, se puede definir el archivo \file{memoria.tex} como ``documento maestro'' yendo al menú opciones -> ``definir documento actual como documento maestro'', lo que permite compilar con PDFLaTeX memoria.tex directamente desde cualquier archivo que se esté modificando . Se muestra esta opción en la figura \ref{fig:docMaestro}.

\begin{figure}[h]
	\centering
	\includegraphics[width=\textwidth]{./Figures/docMaestro.png}
	\caption{Definir memoria.tex como documento maestro.}
	\label{fig:docMaestro}
\end{figure}

En el menú herramientas se encuentran las opciones de compilación.  Para producir un archivo PDF a partir de un archivo .tex se debe ejecutar PDFLaTeX (el shortcut es F6). Para incorporar nueva bibliografía se debe utilizar la opción BibTeX del mismo menú herramientas (el shortcut es F11).

Notar que para actualizar las tablas de contenidos se debe ejecutar PDFLaTeX dos veces.  Esto se debe a que es necesario actualizar algunos archivos auxiliares antes de obtener el resultado final.  En forma similar, para actualizar las referencias bibliográficas se debe ejecutar primero PDFLaTeX, después BibTeX y finalmente PDFLaTeX dos veces por idénticos motivos.

\section{Personalizando la plantilla, el archivo \file{memoria.tex}}
\label{sec:FillingFile}

Para personalizar la plantilla se debe incorporar la información propia en los distintos archivos \file{.tex}. 

Primero abrir \file{memoria.tex} con TexMaker (o el editor de su preferencia). Se debe ubicar dentro del archivo el bloque de código titulado \emph{INFORMACIÓN DE LA PORTADA} donde se deben incorporar los primeros datos personales con los que se construirá automáticamente la portada.


%----------------------------------------------------------------------------------------

\section{El código del archivo \file{memoria.tex} explicado}

El archivo \file{memoria.tex} contiene la estructura del documento y es el archivo de mayor jerarquía de la memoria.  Podría ser equiparable a la función \emph{main()} de un programa en C, o mejor dicho al archivo fuente .c donde se encuentra definida la función main().

La estructura básica de cualquier documento de \LaTeX{} comienza con la definición de clase del documento, es seguida por un preámbulo donde se pueden agregar funcionalidades con el uso de \texttt{paquetes} (equiparables a bibliotecas de C), y finalmente, termina con el cuerpo del documento, donde irá el contenido de la memoria.

\lstset{%
  basicstyle=\small\ttfamily,
  language=[LaTeX]{TeX}
}

\begin{lstlisting}
\documentclass{article}  <- Definicion de clase
\usepackage{listings}	 <- Preambulo

\begin{document}	 <- Comienzo del contenido propio 
	Hello world!
\end{document}
\end{lstlisting}


El archivo \file{memoria.tex} se encuentra densamente comentado para explicar qué páginas, secciones y elementos de formato está creando el código \LaTeX{} en cada línea. El código está dividido en bloques con nombres en mayúsculas para que resulte evidente qué es lo que hace esa porción de código en particular. Inicialmente puede parecer que hay mucho código \LaTeX{}, pero es principalmente código para dar formato a la memoria por lo que no requiere intervención del usuario de la plantilla.  Sí se deben personalizar con su información los bloques indicados como:

\begin{itemize}
	\item Informacion de la memoria
	\item Resumen
	\item Agradecimientos
	\item Dedicatoria
\end{itemize}

El índice de contenidos, las listas de figura de tablas se generan en forma automática y no requieren intervención ni edición manual por parte del usuario de la plantilla. 

En la parte final del documento se encuentran los capítulos y los apéndices.  Por defecto se incluyen los 5 capítulos propuestos que se encuentran en la carpeta /Chapters. Cada capítulo se debe escribir en un archivo .tex separado y se debe poner en la carpeta \emph{Chapters} con el nombre \file{Chapter1}, \file{Chapter2}, etc\ldots El código para incluir capítulos desde archivos externos se muestra a continuación.

\begin{verbatim}
	\include{Chapters/Chapter1}
	\include{Chapters/Chapter2} 
	\include{Chapters/Chapter3}
	\include{Chapters/Chapter4} 
	\include{Chapters/Chapter5} 
\end{verbatim}

Los apéndices también deben escribirse en archivos .tex separados, que se deben ubicar dentro de la carpeta \emph{Appendices}. Los apéndices vienen comentados por defecto con el caracter \code{\%} y para incluirlos simplemente se debe eliminar dicho caracter.

Finalmente, se encuentra el código para incluir la bibliografía en el documento final.  Este código tampoco debe modificarse. La metodología para trabajar las referencias bibliográficas se desarrolla en la sección \ref{sec:biblio}.
%----------------------------------------------------------------------------------------

\section{Bibliografía}
\label{sec:biblio}

Las opciones de formato de la bibliografía se controlan a través del paquete de latex \option{biblatex} que se incluye en la memoria en el archivo memoria.tex.  Estas opciones determinan cómo se generan las citas bibliográficas en el cuerpo del documento y cómo se genera la bibliografía al final de la memoria.

En el preámbulo se puede encontrar el código que incluye el paquete biblatex, que no requiere ninguna modificación del usuario de la plantilla, y que contiene las siguientes opciones:

\begin{lstlisting}
\usepackage[backend=bibtex,
	natbib=true, 
	style=numeric, 
	sorting=none]
{biblatex}
\end{lstlisting}

En el archivo \file{reference.bib} se encuentran las referencias bibliográficas que se pueden citar en el documento.  Para incorporar una nueva cita al documento lo primero es agregarla en este archivo con todos los campos necesario.  Todas las entradas bibliográficas comienzan con $@$ y una palabra que define el formato de la entrada.  Para cada formato existen campos obligatorios que deben completarse. No importa el orden en que las entradas estén definidas en el archivo .bib.  Tampoco es importante el orden en que estén definidos los campos de una entrada bibliográfica. A continuación se muestran algunos ejemplos:

\begin{lstlisting}
@ARTICLE{ARTICLE:1,
    AUTHOR="John Doe",
    TITLE="Title",
    JOURNAL="Journal",
    YEAR="2017",
}
\end{lstlisting}


\begin{lstlisting}
@BOOK{BOOK:1,
    AUTHOR="John Doe",
    TITLE="The Book without Title",
    PUBLISHER="Dummy Publisher",
    YEAR="2100",
}
\end{lstlisting}


\begin{lstlisting}
@INBOOK{BOOK:2,
    AUTHOR="John Doe",
    TITLE="The Book without Title",
    PUBLISHER="Dummy Publisher",
    YEAR="2100",
    PAGES="100-200",
}
\end{lstlisting}


\begin{lstlisting}
@MISC{WEBSITE:1,
    HOWPUBLISHED = "\url{http://example.com}",
    AUTHOR = "Intel",
    TITLE = "Example Website",
    MONTH = "12",
    YEAR = "1988",
    URLDATE = {2012-11-26}
}
\end{lstlisting}

Se debe notar que los nombres \emph{ARTICLE:1}, \emph{BOOK:1}, \emph{BOOK:2} y \emph{WEBSITE:1} son nombres de fantasía que le sirve al autor del documento para identificar la entrada. En este sentido, se podrían reemplazar por cualquier otro nombre.  Tampoco es necesario poner : seguido de un número, en los ejemplos sólo se incluye como un posible estilo para identificar las entradas.

La entradas se citan en el documento con el comando: 

\begin{verbatim}
\citep{nombre_de_la_entrada}
\end{verbatim}

Y cuando se usan, se muestran así: \citep{ARTICLE:1}, \citep{BOOK:1}, \citep{BOOK:2}, \citep{WEBSITE:1}.  Notar cómo se conforma la sección Bibliografía al final del documento.

Finalmente y como se mencionó en la subsección \ref{subsec:configurando}, para actualizar las referencias bibliográficas tanto en la sección bibliografía como las citas en el cuerpo del documento, se deben ejecutar las herramientas de compilación PDFLaTeX, BibTeX, PDFLaTeX, PDFLaTeX, en ese orden.  Este procedimiento debería resolver cualquier mensaje "Citation xxxxx on page x undefined".

	\chapter{Introducción específica} % Main chapter title

\label{Chapter2}

%----------------------------------------------------------------------------------------
%	SECTION 1
%----------------------------------------------------------------------------------------
%Todos los capítulos deben comenzar con un breve párrafo introductorio que indique cuál es el contenido que se encontrará al leerlo.  La redacción sobre el contenido de la memoria debe hacerse en presente y todo lo referido al proyecto en pasado, siempre de modo impersonal.
En el presente capítulo se describen los componentes de hardware, software, protocolos de comunicación y plataformas utilizados para realizar el trabajo.

\section{Componentes principales del hardware}
\label{sec:hw:components}

En esta sección se describen los módulos de hardware de terceros utilizados en el trabajo.

\subsection{ESP32-DevKitC}
%https://docs.espressif.com/projects/esp-dev-kits/en/latest/esp32/esp32-devkitc/index.html

La placa ESP32-DevKitC V4, figura \ref{fig:devkit}, es una placa de desarrollo basada en el SoC (\textit{System On Chip}) ESP32 de Espressif Systems. Es una placa popular y versátil ampliamente utilizada por desarrolladores, ingenieros y aficionados para la creación de prototipos y el desarrollo de proyectos de internet de las cosas o IoT (del inglés \textit{Internet of Things}), sistemas embebidos y otras aplicaciones \cite{DEV:KIT}.

Cacterísticas generales:

\begin{itemize}
	%\item Basada en diferentes módulos ESP32.
	\item Microcontrolador con arquitectura de uno o dos núcleos de 32 bits con velocidades de reloj de hasta 240 MHz.
	\item Memoria SRAM integrada.
	%\item Memoria flash integrada para almacenamiento de firmware.
	\item Conectividad Wi-Fi 802.11 b/g/n (2.4 GHz) integrada.
	\item Conectividad bluetooth (classic y low energy) integrada.
	%\item La mayoría de los pines I/O del módulo ESP32 están disponibles a través de pines header para una fácil conexión.
	%\item Permite la conexión de periféricos con cables jumper o el montaje en una placa de pruebas.
	%\item Conector micro USB para alimentación y comunicación.
	%\item Botones de reset y boot integrados para facilitar la programación.
	%\item Compatible con múltiples entornos de desarrollo (ESP-IDF, Arduino IDE, MicroPython).
\end{itemize}

\begin{figure}[h]
\centering
\includegraphics[scale=.5]{./Figures/devkit.png}
	\caption{ESP32-DevKitC V4 con el módulo ESP32-WROOM-32 integrado\protect\footnotemark.}
	\label{fig:devkit}
\end{figure}

\footnotetext{Imagen tomada de \url{https://docs.espressif.com/projects/esp-dev-kits/en/latest/esp32/esp32-devkitc/user_guide.html\#overview}}



\subsection{Módulo sensor PH-4502C}
El sensor de pH analógico PH-4502C, figura \ref{fig:ph}, es un módulo electrónico diseñado para medir el grado de acidez de soluciones líquidas, típicamente el modelo E201-BNC \cite{PH:4502C}. Este sensor proporciona una salida de tensión analógica que es proporcional al nivel de pH detectado por el electrodo.

Características:

\begin{itemize}
	%\item Módulo electrónico compatible con electrodos de pH con conector BNC.
	\item Rango de detección de pH de 0 a 14.
	\item Salida analógica que varía, entre 0 y 5 VDC, con el pH del liquido.
	%\item Requiere una alimentación de voltaje de 5V DC.
	%\item Corriente de operación entre 5 y 10 mA.
	\item Temperatura de operación del módulo generalmente entre -10 °C y 50 °C.
	%\item Incorpora un potenciómetro para el ajuste del offset o calibración.
	%\item El electrodo tiene un tiempo de estabilización de respuesta de aproximadamente 1 minuto.
\end{itemize}

\begin{figure}[h]
\centering
\includegraphics[scale=.5]{./Figures/ph.png}
	\caption{Distribución de pines del módulo PH-4502C\protect\footnotemark.}
	\label{fig:ph}
\end{figure}

\footnotetext{Imagen tomada de \url{https://uelectronics.com/producto/sensor-de-ph-liquido/?srsltid=AfmBOop9186NUUqLe2lmdM_tZTSfz79gsdcGSMFpg6aQvBxj-Fu9oF5t}}



%https://uelectronics.com/producto/sensor-de-ph-liquido/?srsltid=AfmBOop9186NUUqLe2lmdM_tZTSfz79gsdcGSMFpg6aQvBxj-Fu9oF5t

\subsection{Módulo medidor de solidos disueltos totales}

El medidor de sólidos disueltos totales o TDS (del inglés \textit{Total Dissolved Solids}) Meter 1.0, figura \ref{fig:tds}, es un sensor o módulo electrónico diseñado para medir la cantidad total de sustancias orgánicas e inorgánicas disueltas en un líquido, expresada típicamente en partes por millón (ppm) o miligramos por litro (mg/l) \cite{TDS}. Este tipo de sensor se utiliza para evaluar la calidad del agua en diversas aplicaciones como hidroponía, acuicultura, tratamiento de aguas y monitoreo ambiental.

Características:

\begin{itemize}
	%\item Requiere alimentación de 3.3 a 5.5 VDC.
	\item Salida analógica de 0 a 2,3 VDC (proporcional a TDS).
	\item Rango de medición típico de 0 a 1000 ppm.
	%\item Incluye sonda impermeable con electrodos.
	\item Interfaz de 3 pines (VCC, GND, output).
	\item Conector para la sonda (2 pines).
	%\item Compensación de temperatura.
\end{itemize}


\begin{figure}[h]
\centering
\includegraphics[scale=.5]{./Figures/tds.png}
	\caption{Ilustración del módulo sensor TDS meter v1.0\protect\footnotemark.}
	\label{fig:tds}
\end{figure}

\footnotetext{Imagen tomada de \url{https://www.digikey.be/htmldatasheets/production/2799469/0/0/1/sen0244.html}}



\subsection{Sensor de humedad capacitivo V2.0}

El sensor de humedad capacitivo V2.0, figura \ref{fig:moisture}, mide los niveles de humedad mediante detección capacitiva en lugar de resistiva, como otros sensores disponibles. Fabricado con material resistente a la corrosión, ofrece una vida útil prolongada al insertarse en el sustrato alrededor de las plantas \cite{MOISTURE}.

\begin{itemize}
	\item Requiere alimentación de 3,3 a 5,5 VDC.
	\item Corriente de operación de 5 mA.
	\item Salida analógica.
	%\item Incluye un regulador de tensión integrado compatible con MCUs de 3.3V y 5V.
\end{itemize}


\begin{figure}[h]
\centering
\includegraphics[scale=.5]{./Figures/moisture.png}
	\caption{Módulo sensor de humedad capacitivo\protect\footnotemark.}
	\label{fig:moisture}
\end{figure}

\footnotetext{Imagen tomada de \url{https://probots.co.in/soil-moisture-sensor-capacitive-v1-2.html}}


\subsection{Sensor de temperatura digital DS18B20}

El DS18B20, figura \ref{fig:ds18b20}, es un sensor de temperatura digital que proporciona mediciones en grados celsius con una resolución configurable de 9 a 12 bits \cite{DS18B20}. %Este sensor se comunica a través de un bus 1-Wire, lo que significa que solo requiere una línea de datos (además de tierra) para la comunicación con un micontrolador. Cada DS18B20 tiene un código de serie único de 64 bits grabado en fábrica, lo que permite que múltiples sensores funcionen en el mismo bus 1-Wire, lo que permite la creación de redes de sensores de temperatura distribuidos.

Características:

\begin{itemize}
	%\item Requiere alimentación de 3.3 a 5.5 VDC.
	\item Rango de medición de temperatura: -55 °C a +125 °C.
	\item Precisión: ±0,5 °C en el rango de -10 °C a +85 °C.
	%\item Resolución configurable: 9, 10, 11 o 12 bits (por defecto 12 bits).
	\item Interfaz de comunicación: 1-Wire (requiere un solo pin digital).
	%\item Cada sensor tiene una dirección única de 64 bits.
	%\item Puede alimentarse a través de la línea de datos (\textit{parasite power}) o con una fuente externa.
	%\item Tiempo de conversión de temperatura: hasta 750 ms (para resolución de 12 bits).
	%\item Disponible en encapsulado TO-92 y en versiones con sonda impermeable.
\end{itemize}


\begin{figure}[h]
\centering
\includegraphics[scale=.5]{./Figures/ds18b20.png}
	\caption{Pinout del sensor DS18B20 y presentación del chip con su vaina protectora característica\protect\footnotemark.}
	\label{fig:ds18b20}
\end{figure}

\footnotetext{Imagen tomada de \url{https://tienda.ityt.com.ar/sensor-temp-hum-ic/1694-sensor-temperatura-ds18b20-18b20-1-wire-one-wire-itytarg.html}}


%https://tienda.ityt.com.ar/sensor-temp-hum-ic/1694-sensor-temperatura-ds18b20-18b20-1-wire-one-wire-itytarg.html

\subsection{Sensor de luz ambiental digital BH1750}

El circuito integrado BH1750, figura \ref{fig:BH1750}, es un sensor de luz ambiental digital con interfaz de bus I2C (\textit{Inter-Integrated Circuit}) \cite{BH1750}. Este sensor es capaz de detectar un amplio rango de intensidad luminosa con alta resolución, desde 1 hasta 65535 lux. El chip proporciona una salida digital directa, lo que elimina la necesidad de cálculos complejos.

Características:

\begin{itemize}
	%\item Requiere alimentación de 3.3 a 5.5 VDC.
	\item Interfaz de comunicación I2C.
	\item Rango de medición de 1 a 65535 lux.
	\item Resolución: 16 bits.
\end{itemize}

\begin{figure}[h]
\centering
\includegraphics[scale=.5]{./Figures/BH1750.png}
	\caption{Pinout del módulo de adaptación del sensor BH1750\protect\footnotemark.}
	\label{fig:BH1750}
\end{figure}

\footnotetext{Imagen tomada de \url{https://mytectutor.com/bh1750-ambient-light-sensor-with-arduino/}}


%https://mytectutor.com/bh1750-ambient-light-sensor-with-arduino/


\section{Componentes principales del software}
\label{sec:sw:components}
En esta sección se describen las herramientas de software de terceros utilizados en el trabajo.


\subsection{ESP-IDF}

ESP-IDF (\textit{Espressif IoT Development Framework}) es un conjunto de herramientas de desarrollo integral provisto por Espressif Systems. Este framework facilita la creación de firmware para su línea de SoCs ESP32 y ESP8266. Incluye un sistema operativo en tiempo real (FreeRTOS), bibliotecas con APIs para diversos periféricos y protocolos (Wi-Fi, bluetooth, TCP/IP), compilador (basado en GCC), depurador (GDB) y utilidades para la construcción, flasheo y monitoreo de proyectos. ESP-IDF permite a los desarrolladores escribir aplicaciones en C o C++, quienes aprovechan así la potencia y la conectividad de los chips de Espressif \cite{ESPIDF}.


\subsection{FreeRTOS}

FreeRTOS es un sistema operativo en tiempo real (RTOS) popular y de código abierto. Ofrece un núcleo pequeño y eficiente, apropiado para microcontroladores y sistemas con recursos limitados. Proporciona mecanismos de multitarea, como hilos (tareas), gestión de memoria, sincronización y comunicación entre tareas (semáforos, mutexes, colas). Facilita la organización y la gestión de la ejecución de múltiples funciones de manera concurrente y determinista, crucial para aplicaciones de tiempo real. Su portabilidad permite su uso en una amplia variedad de arquitecturas de procesadores \cite{FREERTOS}.

\subsection{Ceedling}

Ceedling es un framework de construcción y prueba para proyectos de software embebido en C. Automatiza tareas como la compilación, el enlazado y la ejecución de pruebas unitarias. Integra herramientas como Unity, CMock y Ruby. Facilita la adopción de prácticas de desarrollo basadas en pruebas (TDD) y asegura la calidad del código mediante la verificación automatizada de unidades de software individuales \cite{CEEDLING}.

\subsection{React Native}
React Native es un framework de código abierto desarrollado por Meta. Permite la creación de aplicaciones móviles para plataformas iOS y Android desde una única base de código JavaScript. Utiliza los mismos bloques de construcción de la interfaz de usuario que las aplicaciones nativas. Esto resulta en aplicaciones con apariencia y rendimiento nativos \cite{reactnative}.

\section{Protocolos de comunicación empleados}

A continuación, se detallan los protocolos de comunicación empleados en la realización del trabajo.

\subsection{HTTP}
HTTP (\textit{Hypertext Transfer Protocol}) es un protocolo de aplicación que define cómo los clientes (navegadores web) solicitan recursos (páginas web, imágenes, etc.) a los servidores y cómo estos responden. Utiliza un modelo de petición-respuesta. Las peticiones HTTP incluyen un método (GET, POST, PUT, DELETE, etc.) que indica la acción que el cliente desea realizar. Las respuestas HTTP contienen un código de estado que informa sobre el resultado de la petición \cite{HTTP}.

\subsection{I2C}
I2C (Inter-Integrated Circuit) es un protocolo de comunicación serial síncrono, multi-maestro/esclavo, de baja velocidad y corta distancia. Utiliza solo dos líneas bidireccionales: SDA (datos seriales) y SCL (reloj serial), ambas conectadas a través de resistencias pull-up. Permite que múltiples dispositivos se comuniquen entre sí en el mismo bus. Los maestros inician la comunicación y controlan el reloj, mientras que los esclavos responden a las peticiones de los maestros \cite{I2C}.

\subsection{1-Wire}
1-Wire es un protocolo de comunicación serial semidúplex. Utiliza un único conductor para la comunicación de datos y, en algunos casos, para la alimentación. Un maestro controla la comunicación con uno o varios dispositivos esclavos en el mismo bus. El protocolo es relativamente lento pero resulta económico para conectar sensores, memorias y otros dispositivos de baja velocidad, especialmente en aplicaciones donde el bajo número de pines es limitado \cite{1WIRE}.










 
	\chapter{Diseño e implementación} % Main chapter title

\label{Chapter3} % Change X to a consecutive number; for referencing this chapter elsewhere, use \ref{ChapterX}

En este capítulo se abordará la descripción de la arquitectura general del sistema, la arquitectura del software, los módulos componentes del software, el desarrollo del software, el diseño del hardware, la selección y la calibración de sensores y el desarrollo de la aplicación movil.
\section{Diagrama de bloques}
En la figura \ref{fig:d_bloques} se muestra el diagrama en bloques general del sistema donde se describe la arquitectura aplicada al trabajo.

\begin{figure}[h]
\centering
\includegraphics[scale=.4]{./Figures/d_bloques.png}
	\caption{Diagrama de bloques del sistema.}
	\label{fig:d_bloques}
\end{figure}

El sistema embebido implementado en este trabajo consta de una PCB centralizadora, diseñada para integrar y gestionar todos los módulos de hardware. Esta arquitectura asegura la alimentación, adaptación y protección de todos sus componentes. El sistema completo abarca desde la adquisición de datos mediante sensores hasta las acciones sobre el entorno a través de actuadores. Además, se complementa con una interfaz de usuario móvil para el monitoreo y control remoto.

A continuación, se describen brevemente los bloques y su función.

\begin{itemize}
	\item Sensores: este bloque se encarga de la transducción de magnitudes físicas en señales eléctricas, lo que permite la digitalización de variables ambientales de interés para el control del cultivo.
	\item Adaptación de señal: los módulos de adaptación acondicionan las señales provenientes de los sensores. Para garantizar la correcta interpretación por parte de la unidad de microcontrolador o MCU (del inglés \textit{MicroController Unit}), ajustan los niveles de tensión y la relación señal-ruido a valores apropiados.
	\item MCU: el núcleo del sistema, basado en el chip ESP32, orquesta la comunicación y el control de todos los módulos. Provee la capacidad de procesamiento y la conectividad inalámbrica necesarias para la automatización del cultivo y la interacción con la aplicación móvil.
	\item Interfaces de salida: proporciona aislamiento galvánico y acondicionamiento de potencia para la activación de los actuadores, lo que asegura la protección del MCU y la correcta operación de los componentes de mayor potencia.
	\item Actuadores: los actuadores (ventiladores, bomba de irrigación, luces, resistencia calefactora) ejecutan las acciones de control y modifican las condiciones ambientales del cultivo según las necesidades.
	\item Aplicación móvil de usuario: desarrollada para facilitar la interacción con el sistema, la aplicación móvil permite el monitoreo en tiempo real de las condiciones del cultivo y el control remoto de los actuadores.
	\item Interfaz de servicio web: se implementó una interfaz para la transmisión de datos a un servicio web externo que permite el almacenamiento y análisis de información del cultivo. Esta funcionalidad se encuentra fuera del alcance principal de este trabajo.

\end{itemize}

\section{Arquitectura del firmware}

En la presente sección se aborda la arquitectura del firmware del microcontrolador.

\subsection{Patrones}

A continuación, se detallan los patrones de diseño arquitectónico utilizados.

\subsubsection{Arquitectura en capas}
Se adoptó un patrón de arquitectura en capas para estructurar el software desarrollado, lo que permitió una separación de funcionalidades clara mediante niveles de abstracción. Dicha metodología divide el sistema en niveles horizontales, cada uno con responsabilidades específicas y bien definidas, que facilita el desarrollo, la mantenibilidad y la escalabilidad del código.

A continuación, se enumeran las capas de abstracción que constituyen el firmware.

\begin{itemize}
	\item Capa de aplicación.
	\item Capa de sistema operativo.
	\item Capa de abstracción de hardware (HAL).
\end{itemize}


\subsubsection{Capa de abstracción de hardware}

Para facilitar la interacción con los diversos componentes de hardware y garantizar la portabilidad del código, se implementó una capa de abstracción basada en Espressif HAL (\textit{Hardware Abstraction Layer}). Integrada dentro del SDK de ESP-IDF, esta capa proporciona una interfaz uniforme para el control de los periféricos del ESP32, independientemente de las particularidades del hardware subyacente.

\subsubsection{Control ambiental}

El patrón de control ambiental se adoptó como estrategia arquitectónica para la capa de aplicación. El sistema embebido requirió la monitorización y modificación del entorno mediante sensores y actuadores. Este patrón permitió la estructuración de la lógica de control y facilitó la gestión de las interacciones entre los componentes de hardware y la implementación de los algoritmos de control.


\subsection{Componentes}

La capa de aplicación, figura \ref{fig:arq_bloques}, está constituida por los componentes de software encargados de gestionar cada una de las funcionalidades del sistema.
A continuación, se describe brevemente la funcionalidad de cada uno de estos componentes.

\begin{itemize}
	\item Monitor de temperatura: mide la temperatura de la solución nutritiva para garantizar que las plantas crezcan en un entorno térmicamente adecuado.
	\item Monitor de conductividad eléctrica (CE): evalúa la concentración de nutrientes presentes en la solución hidropónica.
	\item Monitor de nivel: monitorea que el nivel de solución nutritiva en el sistema no caiga por debajo de un valor determinado.
	\item Monitor de pH: mide el nivel de acidez de la solución nutritiva, lo que ayuda a mantener el pH dentro del rango óptimo para el cultivo.
	\item Monitor de humedad del sustrato: mide el contenido de humedad en el sustrato del cultivo.
	\item Monitor de luz: mide la cantidad de luz disponible en el entorno a través de un sensor.
	\item Servidor web embebido: el servidor web embebido permite la comunicación con el sistema de control del cultivo a través de la red. Este servidor permite monitorear, configurar y controlar el sistema desde cualquier dispositivo con acceso a la red.
	\item Control de ciclo de luz: este módulo se encarga de gestionar el ciclo de iluminación del cultivo. La configuración de dicho ciclo se realiza por medio de la interfaz de usuario.
	\item Control de ciclo de oxigenación/ventilación: este componente activa y desactiva el flujo de aire en el sistema de cultivo según la configuración del usuario.
	\item Control de hidratación: gestiona el riego, activa la bomba de agua y ajusta los niveles de humedad del sustrato según las mediciones del sensor de humedad. Esto asegura la cantidad adecuada de agua para el desarrollo de las plantas.
	\item Administrador: es el componente encargado de la centralización y validación global de los datos suministrados por los demás módulos de software.
\end{itemize}



\begin{figure}[h]
\centering
\includegraphics[scale=.4]{./Figures/arq_bloques.png}
	\caption{Diagrama de módulos funcionales y sus interacciones.}
	\label{fig:arq_bloques}
\end{figure}



\section{Desarrollo del software}

Para facilitar la escalabilidad de la aplicación, se implementó una tarea de FreeRTOS para cada componente del firmware. Esta arquitectura modular permite modificar cada componente de forma independiente, sin afectar al resto del sistema.

\subsection{Gestión de la comunicación entre tareas}
La comunicación entre tareas se gestiona mediante un patrón de publicación/suscripción con múltiples canales, que utiliza colas de FreeRTOS como buffers para el intercambio de datos. Esta arquitectura desacopla los productores de información (ej., monitores) de los consumidores (ej., controlador, administrador), lo que garantiza la integridad de los datos y la estabilidad del sistema al evitar condiciones de carrera.

\subsection{Gestión de la prioridad}
Durante el desarrollo del firmware, se abordó la organización de las tareas considerando el clásico problema de productores-consumidores. Esta estrategia permitió estructurar el sistema en función del flujo de información entre tareas.

Se definieron los monitores como tareas productoras, encargadas de obtener datos del entorno mediante sensores y publicarlos en estructuras compartidas .

Se implementaron los controladores como tareas consumidoras, cuya función es tomar decisiones en base a los datos recibidos y accionar los actuadores correspondientes.

Se desarrolló una tarea de administrador, también consumidora, responsable de coordinar el funcionamiento general del sistema y gestionar configuraciones.

El servidor embebido fue programado con un doble rol: responde a eventos generados por el usuario (como solicitudes de información o configuraciones) y también produce eventos que otras tareas consumen, como actualizaciones de parámetros.

Se asignaron diferentes niveles de prioridad a las tareas según su disponibilidad requerida. Las tareas de monitoreo se crearon con prioridad intermedia, mientras que las tareas de control y administración se configuraron con mayor prioridad para asegurar tiempos de respuesta adecuados ante eventos prioritarios. El servidor, al depender de la interacción del usuario, fue asignado con prioridad alta.

\subsection{Controladores de dispositivos}

Se desarrollaron controladores de dispositivos (\textit{device drivers}) dedicados a gestionar directamente las unidades de hardware involucradas en el sistema.

Estos controladores permiten abstraer las operaciones de bajo nivel necesarias para interactuar con sensores y actuadores.

\subsection{Gestión de fallos}
El sistema incorpora mecanismos de detección y manejo de fallos tanto a nivel de hardware como de software. Cada tarea cuenta con rutinas de supervisión que permiten identificar comportamientos anómalos o pérdida de respuesta de los dispositivos. En caso de error, se activan procedimientos de contingencia que buscan restablecer el funcionamiento sin comprometer la estabilidad del sistema general.

Por ejemplo, si un sensor deja de enviar datos válidos durante un tiempo determinado, se genera una alerta que es registrada y comunicada al usuario. De igual modo, si un actuador no responde a una orden tras múltiples intentos, se desactiva su uso y se notifica al sistema mediante una bandera de estado.

Además, se implementó un mecanismo de \textit{watchdog} que reinicia automáticamente el microcontrolador en caso de bloqueo o error crítico no recuperable. La configuración base del usuario se almacena en memoria no volátil (NVS), lo que permite que el sistema se recupere de cortes de energía sin perder los parámetros definidos.

\subsection{Servidor embebido}
Para facilitar la interacción con el usuario, el sistema cuenta con un servidor embebido basado en HTTP alojado en la propia ESP32. Este servidor permite exponer una API REST, a través de la cual se puede monitorear el estado del cultivo, consultar los valores de los sensores en tiempo real y ejecutar acciones como el encendido manual de actuadores o la configuración de alertas.

El servidor opera de forma concurrente con el resto de las tareas del sistema gracias a la arquitectura multitarea de FreeRTOS. Para garantizar la seguridad de acceso, se implementó un sistema básico de autenticación por usuario y contraseña que protege las rutas de la API.

El servidor responde a peticiones HTTP provenientes tanto de navegadores como de la aplicación móvil desarrollada en React Native, que primero transmite las credenciales de red por Bluetooth para iniciar la conexión Wi-Fi y luego se comunica mediante la API expuesta.

Este servidor embebido representa un componente clave para el monitoreo remoto y la escalabilidad del sistema, ya que permite una integración sencilla con otras plataformas IoT o bases de datos externas si se desea en el futuro.

\subsection{Lógica de negocio e interacción con la aplicación móvil}
La interacción entre la aplicación móvil y el dispositivo se desarrolla en dos etapas principales: configuración inicial por Bluetooth y comunicación operativa vía HTTP.

\subsubsection{Fase de configuración}
Al iniciar el sistema por primera vez o tras un reinicio de red, la aplicación móvil se conecta a la ESP32 mediante \textit{Bluetooth Low Energy} (BLE). A través de esta conexión, el usuario ingresa las credenciales de su red Wi-Fi doméstica (SSID y contraseña), que son transmitidas de forma segura al microcontrolador. Una vez recibidos estos datos, el dispositivo intenta conectarse a la red y, de ser exitoso, inicia el servidor embebido.

\subsubsection{Fase operativa}
Luego de establecer conexión Wi-Fi, la ESP32 expone una API REST que permite a la aplicación móvil interactuar con el sistema. Entre las operaciones más importantes se incluyen:

\begin{itemize}
	\item Lectura de datos ambientales (temperatura, humedad, pH, EC, etc.).
	\item Consulta del estado de los actuadores (iluminación, bomba, ventilación).
	\item Configuración de parámetros personalizados y umbrales de alerta.
	\item Activación manual o automática de funciones del sistema.
	\item Acceso a registros de eventos o errores del sistema.
\end{itemize}

\subsubsection{Manejo de estados y sincronización}
El firmware mantiene una estructura centralizada de datos en memoria que refleja el estado actual del sistema. Esta información se sincroniza con la app móvil cada vez que se establece una sesión HTTP válida. De este modo, el usuario siempre visualiza información actualizada del estado del cultivo.

Además, se contempló la posibilidad de incorporar una capa adicional de persistencia remota, como sincronización con bases de datos externas o servicios en la nube, lo que permitirá futuras extensiones del sistema hacia modelos más complejos de análisis de datos o inteligencia artificial.



	\include{Chapters/Chapter4} 
	\include{Chapters/Chapter5} 
\end{verbatim}

Los apéndices también deben escribirse en archivos .tex separados, que se deben ubicar dentro de la carpeta \emph{Appendices}. Los apéndices vienen comentados por defecto con el caracter \code{\%} y para incluirlos simplemente se debe eliminar dicho caracter.

Finalmente, se encuentra el código para incluir la bibliografía en el documento final.  Este código tampoco debe modificarse. La metodología para trabajar las referencias bibliográficas se desarrolla en la sección \ref{sec:biblio}.
%----------------------------------------------------------------------------------------

\section{Bibliografía}
\label{sec:biblio}

Las opciones de formato de la bibliografía se controlan a través del paquete de latex \option{biblatex} que se incluye en la memoria en el archivo memoria.tex.  Estas opciones determinan cómo se generan las citas bibliográficas en el cuerpo del documento y cómo se genera la bibliografía al final de la memoria.

En el preámbulo se puede encontrar el código que incluye el paquete biblatex, que no requiere ninguna modificación del usuario de la plantilla, y que contiene las siguientes opciones:

\begin{lstlisting}
\usepackage[backend=bibtex,
	natbib=true, 
	style=numeric, 
	sorting=none]
{biblatex}
\end{lstlisting}

En el archivo \file{reference.bib} se encuentran las referencias bibliográficas que se pueden citar en el documento.  Para incorporar una nueva cita al documento lo primero es agregarla en este archivo con todos los campos necesario.  Todas las entradas bibliográficas comienzan con $@$ y una palabra que define el formato de la entrada.  Para cada formato existen campos obligatorios que deben completarse. No importa el orden en que las entradas estén definidas en el archivo .bib.  Tampoco es importante el orden en que estén definidos los campos de una entrada bibliográfica. A continuación se muestran algunos ejemplos:

\begin{lstlisting}
@ARTICLE{ARTICLE:1,
    AUTHOR="John Doe",
    TITLE="Title",
    JOURNAL="Journal",
    YEAR="2017",
}
\end{lstlisting}


\begin{lstlisting}
@BOOK{BOOK:1,
    AUTHOR="John Doe",
    TITLE="The Book without Title",
    PUBLISHER="Dummy Publisher",
    YEAR="2100",
}
\end{lstlisting}


\begin{lstlisting}
@INBOOK{BOOK:2,
    AUTHOR="John Doe",
    TITLE="The Book without Title",
    PUBLISHER="Dummy Publisher",
    YEAR="2100",
    PAGES="100-200",
}
\end{lstlisting}


\begin{lstlisting}
@MISC{WEBSITE:1,
    HOWPUBLISHED = "\url{http://example.com}",
    AUTHOR = "Intel",
    TITLE = "Example Website",
    MONTH = "12",
    YEAR = "1988",
    URLDATE = {2012-11-26}
}
\end{lstlisting}

Se debe notar que los nombres \emph{ARTICLE:1}, \emph{BOOK:1}, \emph{BOOK:2} y \emph{WEBSITE:1} son nombres de fantasía que le sirve al autor del documento para identificar la entrada. En este sentido, se podrían reemplazar por cualquier otro nombre.  Tampoco es necesario poner : seguido de un número, en los ejemplos sólo se incluye como un posible estilo para identificar las entradas.

La entradas se citan en el documento con el comando: 

\begin{verbatim}
\citep{nombre_de_la_entrada}
\end{verbatim}

Y cuando se usan, se muestran así: \citep{ARTICLE:1}, \citep{BOOK:1}, \citep{BOOK:2}, \citep{WEBSITE:1}.  Notar cómo se conforma la sección Bibliografía al final del documento.

Finalmente y como se mencionó en la subsección \ref{subsec:configurando}, para actualizar las referencias bibliográficas tanto en la sección bibliografía como las citas en el cuerpo del documento, se deben ejecutar las herramientas de compilación PDFLaTeX, BibTeX, PDFLaTeX, PDFLaTeX, en ese orden.  Este procedimiento debería resolver cualquier mensaje "Citation xxxxx on page x undefined".

	\chapter{Introducción específica} % Main chapter title

\label{Chapter2}

%----------------------------------------------------------------------------------------
%	SECTION 1
%----------------------------------------------------------------------------------------
%Todos los capítulos deben comenzar con un breve párrafo introductorio que indique cuál es el contenido que se encontrará al leerlo.  La redacción sobre el contenido de la memoria debe hacerse en presente y todo lo referido al proyecto en pasado, siempre de modo impersonal.
En el presente capítulo se describen los componentes de hardware, software, protocolos de comunicación y plataformas utilizados para realizar el trabajo.

\section{Componentes principales del hardware}
\label{sec:hw:components}

En esta sección se describen los módulos de hardware de terceros utilizados en el trabajo.

\subsection{ESP32-DevKitC}
%https://docs.espressif.com/projects/esp-dev-kits/en/latest/esp32/esp32-devkitc/index.html

La placa ESP32-DevKitC V4, figura \ref{fig:devkit}, es una placa de desarrollo basada en el SoC (\textit{System On Chip}) ESP32 de Espressif Systems. Es una placa popular y versátil ampliamente utilizada por desarrolladores, ingenieros y aficionados para la creación de prototipos y el desarrollo de proyectos de internet de las cosas o IoT (del inglés \textit{Internet of Things}), sistemas embebidos y otras aplicaciones \cite{DEV:KIT}.

Cacterísticas generales:

\begin{itemize}
	%\item Basada en diferentes módulos ESP32.
	\item Microcontrolador con arquitectura de uno o dos núcleos de 32 bits con velocidades de reloj de hasta 240 MHz.
	\item Memoria SRAM integrada.
	%\item Memoria flash integrada para almacenamiento de firmware.
	\item Conectividad Wi-Fi 802.11 b/g/n (2.4 GHz) integrada.
	\item Conectividad bluetooth (classic y low energy) integrada.
	%\item La mayoría de los pines I/O del módulo ESP32 están disponibles a través de pines header para una fácil conexión.
	%\item Permite la conexión de periféricos con cables jumper o el montaje en una placa de pruebas.
	%\item Conector micro USB para alimentación y comunicación.
	%\item Botones de reset y boot integrados para facilitar la programación.
	%\item Compatible con múltiples entornos de desarrollo (ESP-IDF, Arduino IDE, MicroPython).
\end{itemize}

\begin{figure}[h]
\centering
\includegraphics[scale=.5]{./Figures/devkit.png}
	\caption{ESP32-DevKitC V4 con el módulo ESP32-WROOM-32 integrado\protect\footnotemark.}
	\label{fig:devkit}
\end{figure}

\footnotetext{Imagen tomada de \url{https://docs.espressif.com/projects/esp-dev-kits/en/latest/esp32/esp32-devkitc/user_guide.html\#overview}}



\subsection{Módulo sensor PH-4502C}
El sensor de pH analógico PH-4502C, figura \ref{fig:ph}, es un módulo electrónico diseñado para medir el grado de acidez de soluciones líquidas, típicamente el modelo E201-BNC \cite{PH:4502C}. Este sensor proporciona una salida de tensión analógica que es proporcional al nivel de pH detectado por el electrodo.

Características:

\begin{itemize}
	%\item Módulo electrónico compatible con electrodos de pH con conector BNC.
	\item Rango de detección de pH de 0 a 14.
	\item Salida analógica que varía, entre 0 y 5 VDC, con el pH del liquido.
	%\item Requiere una alimentación de voltaje de 5V DC.
	%\item Corriente de operación entre 5 y 10 mA.
	\item Temperatura de operación del módulo generalmente entre -10 °C y 50 °C.
	%\item Incorpora un potenciómetro para el ajuste del offset o calibración.
	%\item El electrodo tiene un tiempo de estabilización de respuesta de aproximadamente 1 minuto.
\end{itemize}

\begin{figure}[h]
\centering
\includegraphics[scale=.5]{./Figures/ph.png}
	\caption{Distribución de pines del módulo PH-4502C\protect\footnotemark.}
	\label{fig:ph}
\end{figure}

\footnotetext{Imagen tomada de \url{https://uelectronics.com/producto/sensor-de-ph-liquido/?srsltid=AfmBOop9186NUUqLe2lmdM_tZTSfz79gsdcGSMFpg6aQvBxj-Fu9oF5t}}



%https://uelectronics.com/producto/sensor-de-ph-liquido/?srsltid=AfmBOop9186NUUqLe2lmdM_tZTSfz79gsdcGSMFpg6aQvBxj-Fu9oF5t

\subsection{Módulo medidor de solidos disueltos totales}

El medidor de sólidos disueltos totales o TDS (del inglés \textit{Total Dissolved Solids}) Meter 1.0, figura \ref{fig:tds}, es un sensor o módulo electrónico diseñado para medir la cantidad total de sustancias orgánicas e inorgánicas disueltas en un líquido, expresada típicamente en partes por millón (ppm) o miligramos por litro (mg/l) \cite{TDS}. Este tipo de sensor se utiliza para evaluar la calidad del agua en diversas aplicaciones como hidroponía, acuicultura, tratamiento de aguas y monitoreo ambiental.

Características:

\begin{itemize}
	%\item Requiere alimentación de 3.3 a 5.5 VDC.
	\item Salida analógica de 0 a 2,3 VDC (proporcional a TDS).
	\item Rango de medición típico de 0 a 1000 ppm.
	%\item Incluye sonda impermeable con electrodos.
	\item Interfaz de 3 pines (VCC, GND, output).
	\item Conector para la sonda (2 pines).
	%\item Compensación de temperatura.
\end{itemize}


\begin{figure}[h]
\centering
\includegraphics[scale=.5]{./Figures/tds.png}
	\caption{Ilustración del módulo sensor TDS meter v1.0\protect\footnotemark.}
	\label{fig:tds}
\end{figure}

\footnotetext{Imagen tomada de \url{https://www.digikey.be/htmldatasheets/production/2799469/0/0/1/sen0244.html}}



\subsection{Sensor de humedad capacitivo V2.0}

El sensor de humedad capacitivo V2.0, figura \ref{fig:moisture}, mide los niveles de humedad mediante detección capacitiva en lugar de resistiva, como otros sensores disponibles. Fabricado con material resistente a la corrosión, ofrece una vida útil prolongada al insertarse en el sustrato alrededor de las plantas \cite{MOISTURE}.

\begin{itemize}
	\item Requiere alimentación de 3,3 a 5,5 VDC.
	\item Corriente de operación de 5 mA.
	\item Salida analógica.
	%\item Incluye un regulador de tensión integrado compatible con MCUs de 3.3V y 5V.
\end{itemize}


\begin{figure}[h]
\centering
\includegraphics[scale=.5]{./Figures/moisture.png}
	\caption{Módulo sensor de humedad capacitivo\protect\footnotemark.}
	\label{fig:moisture}
\end{figure}

\footnotetext{Imagen tomada de \url{https://probots.co.in/soil-moisture-sensor-capacitive-v1-2.html}}


\subsection{Sensor de temperatura digital DS18B20}

El DS18B20, figura \ref{fig:ds18b20}, es un sensor de temperatura digital que proporciona mediciones en grados celsius con una resolución configurable de 9 a 12 bits \cite{DS18B20}. %Este sensor se comunica a través de un bus 1-Wire, lo que significa que solo requiere una línea de datos (además de tierra) para la comunicación con un micontrolador. Cada DS18B20 tiene un código de serie único de 64 bits grabado en fábrica, lo que permite que múltiples sensores funcionen en el mismo bus 1-Wire, lo que permite la creación de redes de sensores de temperatura distribuidos.

Características:

\begin{itemize}
	%\item Requiere alimentación de 3.3 a 5.5 VDC.
	\item Rango de medición de temperatura: -55 °C a +125 °C.
	\item Precisión: ±0,5 °C en el rango de -10 °C a +85 °C.
	%\item Resolución configurable: 9, 10, 11 o 12 bits (por defecto 12 bits).
	\item Interfaz de comunicación: 1-Wire (requiere un solo pin digital).
	%\item Cada sensor tiene una dirección única de 64 bits.
	%\item Puede alimentarse a través de la línea de datos (\textit{parasite power}) o con una fuente externa.
	%\item Tiempo de conversión de temperatura: hasta 750 ms (para resolución de 12 bits).
	%\item Disponible en encapsulado TO-92 y en versiones con sonda impermeable.
\end{itemize}


\begin{figure}[h]
\centering
\includegraphics[scale=.5]{./Figures/ds18b20.png}
	\caption{Pinout del sensor DS18B20 y presentación del chip con su vaina protectora característica\protect\footnotemark.}
	\label{fig:ds18b20}
\end{figure}

\footnotetext{Imagen tomada de \url{https://tienda.ityt.com.ar/sensor-temp-hum-ic/1694-sensor-temperatura-ds18b20-18b20-1-wire-one-wire-itytarg.html}}


%https://tienda.ityt.com.ar/sensor-temp-hum-ic/1694-sensor-temperatura-ds18b20-18b20-1-wire-one-wire-itytarg.html

\subsection{Sensor de luz ambiental digital BH1750}

El circuito integrado BH1750, figura \ref{fig:BH1750}, es un sensor de luz ambiental digital con interfaz de bus I2C (\textit{Inter-Integrated Circuit}) \cite{BH1750}. Este sensor es capaz de detectar un amplio rango de intensidad luminosa con alta resolución, desde 1 hasta 65535 lux. El chip proporciona una salida digital directa, lo que elimina la necesidad de cálculos complejos.

Características:

\begin{itemize}
	%\item Requiere alimentación de 3.3 a 5.5 VDC.
	\item Interfaz de comunicación I2C.
	\item Rango de medición de 1 a 65535 lux.
	\item Resolución: 16 bits.
\end{itemize}

\begin{figure}[h]
\centering
\includegraphics[scale=.5]{./Figures/BH1750.png}
	\caption{Pinout del módulo de adaptación del sensor BH1750\protect\footnotemark.}
	\label{fig:BH1750}
\end{figure}

\footnotetext{Imagen tomada de \url{https://mytectutor.com/bh1750-ambient-light-sensor-with-arduino/}}


%https://mytectutor.com/bh1750-ambient-light-sensor-with-arduino/


\section{Componentes principales del software}
\label{sec:sw:components}
En esta sección se describen las herramientas de software de terceros utilizados en el trabajo.


\subsection{ESP-IDF}

ESP-IDF (\textit{Espressif IoT Development Framework}) es un conjunto de herramientas de desarrollo integral provisto por Espressif Systems. Este framework facilita la creación de firmware para su línea de SoCs ESP32 y ESP8266. Incluye un sistema operativo en tiempo real (FreeRTOS), bibliotecas con APIs para diversos periféricos y protocolos (Wi-Fi, bluetooth, TCP/IP), compilador (basado en GCC), depurador (GDB) y utilidades para la construcción, flasheo y monitoreo de proyectos. ESP-IDF permite a los desarrolladores escribir aplicaciones en C o C++, quienes aprovechan así la potencia y la conectividad de los chips de Espressif \cite{ESPIDF}.


\subsection{FreeRTOS}

FreeRTOS es un sistema operativo en tiempo real (RTOS) popular y de código abierto. Ofrece un núcleo pequeño y eficiente, apropiado para microcontroladores y sistemas con recursos limitados. Proporciona mecanismos de multitarea, como hilos (tareas), gestión de memoria, sincronización y comunicación entre tareas (semáforos, mutexes, colas). Facilita la organización y la gestión de la ejecución de múltiples funciones de manera concurrente y determinista, crucial para aplicaciones de tiempo real. Su portabilidad permite su uso en una amplia variedad de arquitecturas de procesadores \cite{FREERTOS}.

\subsection{Ceedling}

Ceedling es un framework de construcción y prueba para proyectos de software embebido en C. Automatiza tareas como la compilación, el enlazado y la ejecución de pruebas unitarias. Integra herramientas como Unity, CMock y Ruby. Facilita la adopción de prácticas de desarrollo basadas en pruebas (TDD) y asegura la calidad del código mediante la verificación automatizada de unidades de software individuales \cite{CEEDLING}.

\subsection{React Native}
React Native es un framework de código abierto desarrollado por Meta. Permite la creación de aplicaciones móviles para plataformas iOS y Android desde una única base de código JavaScript. Utiliza los mismos bloques de construcción de la interfaz de usuario que las aplicaciones nativas. Esto resulta en aplicaciones con apariencia y rendimiento nativos \cite{reactnative}.

\section{Protocolos de comunicación empleados}

A continuación, se detallan los protocolos de comunicación empleados en la realización del trabajo.

\subsection{HTTP}
HTTP (\textit{Hypertext Transfer Protocol}) es un protocolo de aplicación que define cómo los clientes (navegadores web) solicitan recursos (páginas web, imágenes, etc.) a los servidores y cómo estos responden. Utiliza un modelo de petición-respuesta. Las peticiones HTTP incluyen un método (GET, POST, PUT, DELETE, etc.) que indica la acción que el cliente desea realizar. Las respuestas HTTP contienen un código de estado que informa sobre el resultado de la petición \cite{HTTP}.

\subsection{I2C}
I2C (Inter-Integrated Circuit) es un protocolo de comunicación serial síncrono, multi-maestro/esclavo, de baja velocidad y corta distancia. Utiliza solo dos líneas bidireccionales: SDA (datos seriales) y SCL (reloj serial), ambas conectadas a través de resistencias pull-up. Permite que múltiples dispositivos se comuniquen entre sí en el mismo bus. Los maestros inician la comunicación y controlan el reloj, mientras que los esclavos responden a las peticiones de los maestros \cite{I2C}.

\subsection{1-Wire}
1-Wire es un protocolo de comunicación serial semidúplex. Utiliza un único conductor para la comunicación de datos y, en algunos casos, para la alimentación. Un maestro controla la comunicación con uno o varios dispositivos esclavos en el mismo bus. El protocolo es relativamente lento pero resulta económico para conectar sensores, memorias y otros dispositivos de baja velocidad, especialmente en aplicaciones donde el bajo número de pines es limitado \cite{1WIRE}.










 
	\chapter{Diseño e implementación} % Main chapter title

\label{Chapter3} % Change X to a consecutive number; for referencing this chapter elsewhere, use \ref{ChapterX}

En este capítulo se abordará la descripción de la arquitectura general del sistema, la arquitectura del software, los módulos componentes del software, el desarrollo del software, el diseño del hardware, la selección y la calibración de sensores y el desarrollo de la aplicación movil.
\section{Diagrama de bloques}
En la figura \ref{fig:d_bloques} se muestra el diagrama en bloques general del sistema donde se describe la arquitectura aplicada al trabajo.

\begin{figure}[h]
\centering
\includegraphics[scale=.4]{./Figures/d_bloques.png}
	\caption{Diagrama de bloques del sistema.}
	\label{fig:d_bloques}
\end{figure}

El sistema embebido implementado en este trabajo consta de una PCB centralizadora, diseñada para integrar y gestionar todos los módulos de hardware. Esta arquitectura asegura la alimentación, adaptación y protección de todos sus componentes. El sistema completo abarca desde la adquisición de datos mediante sensores hasta las acciones sobre el entorno a través de actuadores. Además, se complementa con una interfaz de usuario móvil para el monitoreo y control remoto.

A continuación, se describen brevemente los bloques y su función.

\begin{itemize}
	\item Sensores: este bloque se encarga de la transducción de magnitudes físicas en señales eléctricas, lo que permite la digitalización de variables ambientales de interés para el control del cultivo.
	\item Adaptación de señal: los módulos de adaptación acondicionan las señales provenientes de los sensores. Para garantizar la correcta interpretación por parte de la unidad de microcontrolador o MCU (del inglés \textit{MicroController Unit}), ajustan los niveles de tensión y la relación señal-ruido a valores apropiados.
	\item MCU: el núcleo del sistema, basado en el chip ESP32, orquesta la comunicación y el control de todos los módulos. Provee la capacidad de procesamiento y la conectividad inalámbrica necesarias para la automatización del cultivo y la interacción con la aplicación móvil.
	\item Interfaces de salida: proporciona aislamiento galvánico y acondicionamiento de potencia para la activación de los actuadores, lo que asegura la protección del MCU y la correcta operación de los componentes de mayor potencia.
	\item Actuadores: los actuadores (ventiladores, bomba de irrigación, luces, resistencia calefactora) ejecutan las acciones de control y modifican las condiciones ambientales del cultivo según las necesidades.
	\item Aplicación móvil de usuario: desarrollada para facilitar la interacción con el sistema, la aplicación móvil permite el monitoreo en tiempo real de las condiciones del cultivo y el control remoto de los actuadores.
	\item Interfaz de servicio web: se implementó una interfaz para la transmisión de datos a un servicio web externo que permite el almacenamiento y análisis de información del cultivo. Esta funcionalidad se encuentra fuera del alcance principal de este trabajo.

\end{itemize}

\section{Arquitectura del firmware}

En la presente sección se aborda la arquitectura del firmware del microcontrolador.

\subsection{Patrones}

A continuación, se detallan los patrones de diseño arquitectónico utilizados.

\subsubsection{Arquitectura en capas}
Se adoptó un patrón de arquitectura en capas para estructurar el software desarrollado, lo que permitió una separación de funcionalidades clara mediante niveles de abstracción. Dicha metodología divide el sistema en niveles horizontales, cada uno con responsabilidades específicas y bien definidas, que facilita el desarrollo, la mantenibilidad y la escalabilidad del código.

A continuación, se enumeran las capas de abstracción que constituyen el firmware.

\begin{itemize}
	\item Capa de aplicación.
	\item Capa de sistema operativo.
	\item Capa de abstracción de hardware (HAL).
\end{itemize}


\subsubsection{Capa de abstracción de hardware}

Para facilitar la interacción con los diversos componentes de hardware y garantizar la portabilidad del código, se implementó una capa de abstracción basada en Espressif HAL (\textit{Hardware Abstraction Layer}). Integrada dentro del SDK de ESP-IDF, esta capa proporciona una interfaz uniforme para el control de los periféricos del ESP32, independientemente de las particularidades del hardware subyacente.

\subsubsection{Control ambiental}

El patrón de control ambiental se adoptó como estrategia arquitectónica para la capa de aplicación. El sistema embebido requirió la monitorización y modificación del entorno mediante sensores y actuadores. Este patrón permitió la estructuración de la lógica de control y facilitó la gestión de las interacciones entre los componentes de hardware y la implementación de los algoritmos de control.


\subsection{Componentes}

La capa de aplicación, figura \ref{fig:arq_bloques}, está constituida por los componentes de software encargados de gestionar cada una de las funcionalidades del sistema.
A continuación, se describe brevemente la funcionalidad de cada uno de estos componentes.

\begin{itemize}
	\item Monitor de temperatura: mide la temperatura de la solución nutritiva para garantizar que las plantas crezcan en un entorno térmicamente adecuado.
	\item Monitor de conductividad eléctrica (CE): evalúa la concentración de nutrientes presentes en la solución hidropónica.
	\item Monitor de nivel: monitorea que el nivel de solución nutritiva en el sistema no caiga por debajo de un valor determinado.
	\item Monitor de pH: mide el nivel de acidez de la solución nutritiva, lo que ayuda a mantener el pH dentro del rango óptimo para el cultivo.
	\item Monitor de humedad del sustrato: mide el contenido de humedad en el sustrato del cultivo.
	\item Monitor de luz: mide la cantidad de luz disponible en el entorno a través de un sensor.
	\item Servidor web embebido: el servidor web embebido permite la comunicación con el sistema de control del cultivo a través de la red. Este servidor permite monitorear, configurar y controlar el sistema desde cualquier dispositivo con acceso a la red.
	\item Control de ciclo de luz: este módulo se encarga de gestionar el ciclo de iluminación del cultivo. La configuración de dicho ciclo se realiza por medio de la interfaz de usuario.
	\item Control de ciclo de oxigenación/ventilación: este componente activa y desactiva el flujo de aire en el sistema de cultivo según la configuración del usuario.
	\item Control de hidratación: gestiona el riego, activa la bomba de agua y ajusta los niveles de humedad del sustrato según las mediciones del sensor de humedad. Esto asegura la cantidad adecuada de agua para el desarrollo de las plantas.
	\item Administrador: es el componente encargado de la centralización y validación global de los datos suministrados por los demás módulos de software.
\end{itemize}



\begin{figure}[h]
\centering
\includegraphics[scale=.4]{./Figures/arq_bloques.png}
	\caption{Diagrama de módulos funcionales y sus interacciones.}
	\label{fig:arq_bloques}
\end{figure}



\section{Desarrollo del software}

Para facilitar la escalabilidad de la aplicación, se implementó una tarea de FreeRTOS para cada componente del firmware. Esta arquitectura modular permite modificar cada componente de forma independiente, sin afectar al resto del sistema.

\subsection{Gestión de la comunicación entre tareas}
La comunicación entre tareas se gestiona mediante un patrón de publicación/suscripción con múltiples canales, que utiliza colas de FreeRTOS como buffers para el intercambio de datos. Esta arquitectura desacopla los productores de información (ej., monitores) de los consumidores (ej., controlador, administrador), lo que garantiza la integridad de los datos y la estabilidad del sistema al evitar condiciones de carrera.

\subsection{Gestión de la prioridad}
Durante el desarrollo del firmware, se abordó la organización de las tareas considerando el clásico problema de productores-consumidores. Esta estrategia permitió estructurar el sistema en función del flujo de información entre tareas.

Se definieron los monitores como tareas productoras, encargadas de obtener datos del entorno mediante sensores y publicarlos en estructuras compartidas .

Se implementaron los controladores como tareas consumidoras, cuya función es tomar decisiones en base a los datos recibidos y accionar los actuadores correspondientes.

Se desarrolló una tarea de administrador, también consumidora, responsable de coordinar el funcionamiento general del sistema y gestionar configuraciones.

El servidor embebido fue programado con un doble rol: responde a eventos generados por el usuario (como solicitudes de información o configuraciones) y también produce eventos que otras tareas consumen, como actualizaciones de parámetros.

Se asignaron diferentes niveles de prioridad a las tareas según su disponibilidad requerida. Las tareas de monitoreo se crearon con prioridad intermedia, mientras que las tareas de control y administración se configuraron con mayor prioridad para asegurar tiempos de respuesta adecuados ante eventos prioritarios. El servidor, al depender de la interacción del usuario, fue asignado con prioridad alta.

\subsection{Controladores de dispositivos}

Se desarrollaron controladores de dispositivos (\textit{device drivers}) dedicados a gestionar directamente las unidades de hardware involucradas en el sistema.

Estos controladores permiten abstraer las operaciones de bajo nivel necesarias para interactuar con sensores y actuadores.

\subsection{Gestión de fallos}
El sistema incorpora mecanismos de detección y manejo de fallos tanto a nivel de hardware como de software. Cada tarea cuenta con rutinas de supervisión que permiten identificar comportamientos anómalos o pérdida de respuesta de los dispositivos. En caso de error, se activan procedimientos de contingencia que buscan restablecer el funcionamiento sin comprometer la estabilidad del sistema general.

Por ejemplo, si un sensor deja de enviar datos válidos durante un tiempo determinado, se genera una alerta que es registrada y comunicada al usuario. De igual modo, si un actuador no responde a una orden tras múltiples intentos, se desactiva su uso y se notifica al sistema mediante una bandera de estado.

Además, se implementó un mecanismo de \textit{watchdog} que reinicia automáticamente el microcontrolador en caso de bloqueo o error crítico no recuperable. La configuración base del usuario se almacena en memoria no volátil (NVS), lo que permite que el sistema se recupere de cortes de energía sin perder los parámetros definidos.

\subsection{Servidor embebido}
Para facilitar la interacción con el usuario, el sistema cuenta con un servidor embebido basado en HTTP alojado en la propia ESP32. Este servidor permite exponer una API REST, a través de la cual se puede monitorear el estado del cultivo, consultar los valores de los sensores en tiempo real y ejecutar acciones como el encendido manual de actuadores o la configuración de alertas.

El servidor opera de forma concurrente con el resto de las tareas del sistema gracias a la arquitectura multitarea de FreeRTOS. Para garantizar la seguridad de acceso, se implementó un sistema básico de autenticación por usuario y contraseña que protege las rutas de la API.

El servidor responde a peticiones HTTP provenientes tanto de navegadores como de la aplicación móvil desarrollada en React Native, que primero transmite las credenciales de red por Bluetooth para iniciar la conexión Wi-Fi y luego se comunica mediante la API expuesta.

Este servidor embebido representa un componente clave para el monitoreo remoto y la escalabilidad del sistema, ya que permite una integración sencilla con otras plataformas IoT o bases de datos externas si se desea en el futuro.

\subsection{Lógica de negocio e interacción con la aplicación móvil}
La interacción entre la aplicación móvil y el dispositivo se desarrolla en dos etapas principales: configuración inicial por Bluetooth y comunicación operativa vía HTTP.

\subsubsection{Fase de configuración}
Al iniciar el sistema por primera vez o tras un reinicio de red, la aplicación móvil se conecta a la ESP32 mediante \textit{Bluetooth Low Energy} (BLE). A través de esta conexión, el usuario ingresa las credenciales de su red Wi-Fi doméstica (SSID y contraseña), que son transmitidas de forma segura al microcontrolador. Una vez recibidos estos datos, el dispositivo intenta conectarse a la red y, de ser exitoso, inicia el servidor embebido.

\subsubsection{Fase operativa}
Luego de establecer conexión Wi-Fi, la ESP32 expone una API REST que permite a la aplicación móvil interactuar con el sistema. Entre las operaciones más importantes se incluyen:

\begin{itemize}
	\item Lectura de datos ambientales (temperatura, humedad, pH, EC, etc.).
	\item Consulta del estado de los actuadores (iluminación, bomba, ventilación).
	\item Configuración de parámetros personalizados y umbrales de alerta.
	\item Activación manual o automática de funciones del sistema.
	\item Acceso a registros de eventos o errores del sistema.
\end{itemize}

\subsubsection{Manejo de estados y sincronización}
El firmware mantiene una estructura centralizada de datos en memoria que refleja el estado actual del sistema. Esta información se sincroniza con la app móvil cada vez que se establece una sesión HTTP válida. De este modo, el usuario siempre visualiza información actualizada del estado del cultivo.

Además, se contempló la posibilidad de incorporar una capa adicional de persistencia remota, como sincronización con bases de datos externas o servicios en la nube, lo que permitirá futuras extensiones del sistema hacia modelos más complejos de análisis de datos o inteligencia artificial.



	\include{Chapters/Chapter4} 
	\include{Chapters/Chapter5} 
\end{verbatim}

Los apéndices también deben escribirse en archivos .tex separados, que se deben ubicar dentro de la carpeta \emph{Appendices}. Los apéndices vienen comentados por defecto con el caracter \code{\%} y para incluirlos simplemente se debe eliminar dicho caracter.

Finalmente, se encuentra el código para incluir la bibliografía en el documento final.  Este código tampoco debe modificarse. La metodología para trabajar las referencias bibliográficas se desarrolla en la sección \ref{sec:biblio}.
%----------------------------------------------------------------------------------------

\section{Bibliografía}
\label{sec:biblio}

Las opciones de formato de la bibliografía se controlan a través del paquete de latex \option{biblatex} que se incluye en la memoria en el archivo memoria.tex.  Estas opciones determinan cómo se generan las citas bibliográficas en el cuerpo del documento y cómo se genera la bibliografía al final de la memoria.

En el preámbulo se puede encontrar el código que incluye el paquete biblatex, que no requiere ninguna modificación del usuario de la plantilla, y que contiene las siguientes opciones:

\begin{lstlisting}
\usepackage[backend=bibtex,
	natbib=true, 
	style=numeric, 
	sorting=none]
{biblatex}
\end{lstlisting}

En el archivo \file{reference.bib} se encuentran las referencias bibliográficas que se pueden citar en el documento.  Para incorporar una nueva cita al documento lo primero es agregarla en este archivo con todos los campos necesario.  Todas las entradas bibliográficas comienzan con $@$ y una palabra que define el formato de la entrada.  Para cada formato existen campos obligatorios que deben completarse. No importa el orden en que las entradas estén definidas en el archivo .bib.  Tampoco es importante el orden en que estén definidos los campos de una entrada bibliográfica. A continuación se muestran algunos ejemplos:

\begin{lstlisting}
@ARTICLE{ARTICLE:1,
    AUTHOR="John Doe",
    TITLE="Title",
    JOURNAL="Journal",
    YEAR="2017",
}
\end{lstlisting}


\begin{lstlisting}
@BOOK{BOOK:1,
    AUTHOR="John Doe",
    TITLE="The Book without Title",
    PUBLISHER="Dummy Publisher",
    YEAR="2100",
}
\end{lstlisting}


\begin{lstlisting}
@INBOOK{BOOK:2,
    AUTHOR="John Doe",
    TITLE="The Book without Title",
    PUBLISHER="Dummy Publisher",
    YEAR="2100",
    PAGES="100-200",
}
\end{lstlisting}


\begin{lstlisting}
@MISC{WEBSITE:1,
    HOWPUBLISHED = "\url{http://example.com}",
    AUTHOR = "Intel",
    TITLE = "Example Website",
    MONTH = "12",
    YEAR = "1988",
    URLDATE = {2012-11-26}
}
\end{lstlisting}

Se debe notar que los nombres \emph{ARTICLE:1}, \emph{BOOK:1}, \emph{BOOK:2} y \emph{WEBSITE:1} son nombres de fantasía que le sirve al autor del documento para identificar la entrada. En este sentido, se podrían reemplazar por cualquier otro nombre.  Tampoco es necesario poner : seguido de un número, en los ejemplos sólo se incluye como un posible estilo para identificar las entradas.

La entradas se citan en el documento con el comando: 

\begin{verbatim}
\citep{nombre_de_la_entrada}
\end{verbatim}

Y cuando se usan, se muestran así: \citep{ARTICLE:1}, \citep{BOOK:1}, \citep{BOOK:2}, \citep{WEBSITE:1}.  Notar cómo se conforma la sección Bibliografía al final del documento.

Finalmente y como se mencionó en la subsección \ref{subsec:configurando}, para actualizar las referencias bibliográficas tanto en la sección bibliografía como las citas en el cuerpo del documento, se deben ejecutar las herramientas de compilación PDFLaTeX, BibTeX, PDFLaTeX, PDFLaTeX, en ese orden.  Este procedimiento debería resolver cualquier mensaje "Citation xxxxx on page x undefined".

	\chapter{Introducción específica} % Main chapter title

\label{Chapter2}

%----------------------------------------------------------------------------------------
%	SECTION 1
%----------------------------------------------------------------------------------------
%Todos los capítulos deben comenzar con un breve párrafo introductorio que indique cuál es el contenido que se encontrará al leerlo.  La redacción sobre el contenido de la memoria debe hacerse en presente y todo lo referido al proyecto en pasado, siempre de modo impersonal.
En el presente capítulo se describen los componentes de hardware, software, protocolos de comunicación y plataformas utilizados para realizar el trabajo.

\section{Componentes principales del hardware}
\label{sec:hw:components}

En esta sección se describen los módulos de hardware de terceros utilizados en el trabajo.

\subsection{ESP32-DevKitC}
%https://docs.espressif.com/projects/esp-dev-kits/en/latest/esp32/esp32-devkitc/index.html

La placa ESP32-DevKitC V4, figura \ref{fig:devkit}, es una placa de desarrollo basada en el SoC (\textit{System On Chip}) ESP32 de Espressif Systems. Es una placa popular y versátil ampliamente utilizada por desarrolladores, ingenieros y aficionados para la creación de prototipos y el desarrollo de proyectos de internet de las cosas o IoT (del inglés \textit{Internet of Things}), sistemas embebidos y otras aplicaciones \cite{DEV:KIT}.

Cacterísticas generales:

\begin{itemize}
	%\item Basada en diferentes módulos ESP32.
	\item Microcontrolador con arquitectura de uno o dos núcleos de 32 bits con velocidades de reloj de hasta 240 MHz.
	\item Memoria SRAM integrada.
	%\item Memoria flash integrada para almacenamiento de firmware.
	\item Conectividad Wi-Fi 802.11 b/g/n (2.4 GHz) integrada.
	\item Conectividad bluetooth (classic y low energy) integrada.
	%\item La mayoría de los pines I/O del módulo ESP32 están disponibles a través de pines header para una fácil conexión.
	%\item Permite la conexión de periféricos con cables jumper o el montaje en una placa de pruebas.
	%\item Conector micro USB para alimentación y comunicación.
	%\item Botones de reset y boot integrados para facilitar la programación.
	%\item Compatible con múltiples entornos de desarrollo (ESP-IDF, Arduino IDE, MicroPython).
\end{itemize}

\begin{figure}[h]
\centering
\includegraphics[scale=.5]{./Figures/devkit.png}
	\caption{ESP32-DevKitC V4 con el módulo ESP32-WROOM-32 integrado\protect\footnotemark.}
	\label{fig:devkit}
\end{figure}

\footnotetext{Imagen tomada de \url{https://docs.espressif.com/projects/esp-dev-kits/en/latest/esp32/esp32-devkitc/user_guide.html\#overview}}



\subsection{Módulo sensor PH-4502C}
El sensor de pH analógico PH-4502C, figura \ref{fig:ph}, es un módulo electrónico diseñado para medir el grado de acidez de soluciones líquidas, típicamente el modelo E201-BNC \cite{PH:4502C}. Este sensor proporciona una salida de tensión analógica que es proporcional al nivel de pH detectado por el electrodo.

Características:

\begin{itemize}
	%\item Módulo electrónico compatible con electrodos de pH con conector BNC.
	\item Rango de detección de pH de 0 a 14.
	\item Salida analógica que varía, entre 0 y 5 VDC, con el pH del liquido.
	%\item Requiere una alimentación de voltaje de 5V DC.
	%\item Corriente de operación entre 5 y 10 mA.
	\item Temperatura de operación del módulo generalmente entre -10 °C y 50 °C.
	%\item Incorpora un potenciómetro para el ajuste del offset o calibración.
	%\item El electrodo tiene un tiempo de estabilización de respuesta de aproximadamente 1 minuto.
\end{itemize}

\begin{figure}[h]
\centering
\includegraphics[scale=.5]{./Figures/ph.png}
	\caption{Distribución de pines del módulo PH-4502C\protect\footnotemark.}
	\label{fig:ph}
\end{figure}

\footnotetext{Imagen tomada de \url{https://uelectronics.com/producto/sensor-de-ph-liquido/?srsltid=AfmBOop9186NUUqLe2lmdM_tZTSfz79gsdcGSMFpg6aQvBxj-Fu9oF5t}}



%https://uelectronics.com/producto/sensor-de-ph-liquido/?srsltid=AfmBOop9186NUUqLe2lmdM_tZTSfz79gsdcGSMFpg6aQvBxj-Fu9oF5t

\subsection{Módulo medidor de solidos disueltos totales}

El medidor de sólidos disueltos totales o TDS (del inglés \textit{Total Dissolved Solids}) Meter 1.0, figura \ref{fig:tds}, es un sensor o módulo electrónico diseñado para medir la cantidad total de sustancias orgánicas e inorgánicas disueltas en un líquido, expresada típicamente en partes por millón (ppm) o miligramos por litro (mg/l) \cite{TDS}. Este tipo de sensor se utiliza para evaluar la calidad del agua en diversas aplicaciones como hidroponía, acuicultura, tratamiento de aguas y monitoreo ambiental.

Características:

\begin{itemize}
	%\item Requiere alimentación de 3.3 a 5.5 VDC.
	\item Salida analógica de 0 a 2,3 VDC (proporcional a TDS).
	\item Rango de medición típico de 0 a 1000 ppm.
	%\item Incluye sonda impermeable con electrodos.
	\item Interfaz de 3 pines (VCC, GND, output).
	\item Conector para la sonda (2 pines).
	%\item Compensación de temperatura.
\end{itemize}


\begin{figure}[h]
\centering
\includegraphics[scale=.5]{./Figures/tds.png}
	\caption{Ilustración del módulo sensor TDS meter v1.0\protect\footnotemark.}
	\label{fig:tds}
\end{figure}

\footnotetext{Imagen tomada de \url{https://www.digikey.be/htmldatasheets/production/2799469/0/0/1/sen0244.html}}



\subsection{Sensor de humedad capacitivo V2.0}

El sensor de humedad capacitivo V2.0, figura \ref{fig:moisture}, mide los niveles de humedad mediante detección capacitiva en lugar de resistiva, como otros sensores disponibles. Fabricado con material resistente a la corrosión, ofrece una vida útil prolongada al insertarse en el sustrato alrededor de las plantas \cite{MOISTURE}.

\begin{itemize}
	\item Requiere alimentación de 3,3 a 5,5 VDC.
	\item Corriente de operación de 5 mA.
	\item Salida analógica.
	%\item Incluye un regulador de tensión integrado compatible con MCUs de 3.3V y 5V.
\end{itemize}


\begin{figure}[h]
\centering
\includegraphics[scale=.5]{./Figures/moisture.png}
	\caption{Módulo sensor de humedad capacitivo\protect\footnotemark.}
	\label{fig:moisture}
\end{figure}

\footnotetext{Imagen tomada de \url{https://probots.co.in/soil-moisture-sensor-capacitive-v1-2.html}}


\subsection{Sensor de temperatura digital DS18B20}

El DS18B20, figura \ref{fig:ds18b20}, es un sensor de temperatura digital que proporciona mediciones en grados celsius con una resolución configurable de 9 a 12 bits \cite{DS18B20}. %Este sensor se comunica a través de un bus 1-Wire, lo que significa que solo requiere una línea de datos (además de tierra) para la comunicación con un micontrolador. Cada DS18B20 tiene un código de serie único de 64 bits grabado en fábrica, lo que permite que múltiples sensores funcionen en el mismo bus 1-Wire, lo que permite la creación de redes de sensores de temperatura distribuidos.

Características:

\begin{itemize}
	%\item Requiere alimentación de 3.3 a 5.5 VDC.
	\item Rango de medición de temperatura: -55 °C a +125 °C.
	\item Precisión: ±0,5 °C en el rango de -10 °C a +85 °C.
	%\item Resolución configurable: 9, 10, 11 o 12 bits (por defecto 12 bits).
	\item Interfaz de comunicación: 1-Wire (requiere un solo pin digital).
	%\item Cada sensor tiene una dirección única de 64 bits.
	%\item Puede alimentarse a través de la línea de datos (\textit{parasite power}) o con una fuente externa.
	%\item Tiempo de conversión de temperatura: hasta 750 ms (para resolución de 12 bits).
	%\item Disponible en encapsulado TO-92 y en versiones con sonda impermeable.
\end{itemize}


\begin{figure}[h]
\centering
\includegraphics[scale=.5]{./Figures/ds18b20.png}
	\caption{Pinout del sensor DS18B20 y presentación del chip con su vaina protectora característica\protect\footnotemark.}
	\label{fig:ds18b20}
\end{figure}

\footnotetext{Imagen tomada de \url{https://tienda.ityt.com.ar/sensor-temp-hum-ic/1694-sensor-temperatura-ds18b20-18b20-1-wire-one-wire-itytarg.html}}


%https://tienda.ityt.com.ar/sensor-temp-hum-ic/1694-sensor-temperatura-ds18b20-18b20-1-wire-one-wire-itytarg.html

\subsection{Sensor de luz ambiental digital BH1750}

El circuito integrado BH1750, figura \ref{fig:BH1750}, es un sensor de luz ambiental digital con interfaz de bus I2C (\textit{Inter-Integrated Circuit}) \cite{BH1750}. Este sensor es capaz de detectar un amplio rango de intensidad luminosa con alta resolución, desde 1 hasta 65535 lux. El chip proporciona una salida digital directa, lo que elimina la necesidad de cálculos complejos.

Características:

\begin{itemize}
	%\item Requiere alimentación de 3.3 a 5.5 VDC.
	\item Interfaz de comunicación I2C.
	\item Rango de medición de 1 a 65535 lux.
	\item Resolución: 16 bits.
\end{itemize}

\begin{figure}[h]
\centering
\includegraphics[scale=.5]{./Figures/BH1750.png}
	\caption{Pinout del módulo de adaptación del sensor BH1750\protect\footnotemark.}
	\label{fig:BH1750}
\end{figure}

\footnotetext{Imagen tomada de \url{https://mytectutor.com/bh1750-ambient-light-sensor-with-arduino/}}


%https://mytectutor.com/bh1750-ambient-light-sensor-with-arduino/


\section{Componentes principales del software}
\label{sec:sw:components}
En esta sección se describen las herramientas de software de terceros utilizados en el trabajo.


\subsection{ESP-IDF}

ESP-IDF (\textit{Espressif IoT Development Framework}) es un conjunto de herramientas de desarrollo integral provisto por Espressif Systems. Este framework facilita la creación de firmware para su línea de SoCs ESP32 y ESP8266. Incluye un sistema operativo en tiempo real (FreeRTOS), bibliotecas con APIs para diversos periféricos y protocolos (Wi-Fi, bluetooth, TCP/IP), compilador (basado en GCC), depurador (GDB) y utilidades para la construcción, flasheo y monitoreo de proyectos. ESP-IDF permite a los desarrolladores escribir aplicaciones en C o C++, quienes aprovechan así la potencia y la conectividad de los chips de Espressif \cite{ESPIDF}.


\subsection{FreeRTOS}

FreeRTOS es un sistema operativo en tiempo real (RTOS) popular y de código abierto. Ofrece un núcleo pequeño y eficiente, apropiado para microcontroladores y sistemas con recursos limitados. Proporciona mecanismos de multitarea, como hilos (tareas), gestión de memoria, sincronización y comunicación entre tareas (semáforos, mutexes, colas). Facilita la organización y la gestión de la ejecución de múltiples funciones de manera concurrente y determinista, crucial para aplicaciones de tiempo real. Su portabilidad permite su uso en una amplia variedad de arquitecturas de procesadores \cite{FREERTOS}.

\subsection{Ceedling}

Ceedling es un framework de construcción y prueba para proyectos de software embebido en C. Automatiza tareas como la compilación, el enlazado y la ejecución de pruebas unitarias. Integra herramientas como Unity, CMock y Ruby. Facilita la adopción de prácticas de desarrollo basadas en pruebas (TDD) y asegura la calidad del código mediante la verificación automatizada de unidades de software individuales \cite{CEEDLING}.

\subsection{React Native}
React Native es un framework de código abierto desarrollado por Meta. Permite la creación de aplicaciones móviles para plataformas iOS y Android desde una única base de código JavaScript. Utiliza los mismos bloques de construcción de la interfaz de usuario que las aplicaciones nativas. Esto resulta en aplicaciones con apariencia y rendimiento nativos \cite{reactnative}.

\section{Protocolos de comunicación empleados}

A continuación, se detallan los protocolos de comunicación empleados en la realización del trabajo.

\subsection{HTTP}
HTTP (\textit{Hypertext Transfer Protocol}) es un protocolo de aplicación que define cómo los clientes (navegadores web) solicitan recursos (páginas web, imágenes, etc.) a los servidores y cómo estos responden. Utiliza un modelo de petición-respuesta. Las peticiones HTTP incluyen un método (GET, POST, PUT, DELETE, etc.) que indica la acción que el cliente desea realizar. Las respuestas HTTP contienen un código de estado que informa sobre el resultado de la petición \cite{HTTP}.

\subsection{I2C}
I2C (Inter-Integrated Circuit) es un protocolo de comunicación serial síncrono, multi-maestro/esclavo, de baja velocidad y corta distancia. Utiliza solo dos líneas bidireccionales: SDA (datos seriales) y SCL (reloj serial), ambas conectadas a través de resistencias pull-up. Permite que múltiples dispositivos se comuniquen entre sí en el mismo bus. Los maestros inician la comunicación y controlan el reloj, mientras que los esclavos responden a las peticiones de los maestros \cite{I2C}.

\subsection{1-Wire}
1-Wire es un protocolo de comunicación serial semidúplex. Utiliza un único conductor para la comunicación de datos y, en algunos casos, para la alimentación. Un maestro controla la comunicación con uno o varios dispositivos esclavos en el mismo bus. El protocolo es relativamente lento pero resulta económico para conectar sensores, memorias y otros dispositivos de baja velocidad, especialmente en aplicaciones donde el bajo número de pines es limitado \cite{1WIRE}.










 
	\chapter{Diseño e implementación} % Main chapter title

\label{Chapter3} % Change X to a consecutive number; for referencing this chapter elsewhere, use \ref{ChapterX}

En este capítulo se abordará la descripción de la arquitectura general del sistema, la arquitectura del software, los módulos componentes del software, el desarrollo del software, el diseño del hardware, la selección y la calibración de sensores y el desarrollo de la aplicación movil.
\section{Diagrama de bloques}
En la figura \ref{fig:d_bloques} se muestra el diagrama en bloques general del sistema donde se describe la arquitectura aplicada al trabajo.

\begin{figure}[h]
\centering
\includegraphics[scale=.4]{./Figures/d_bloques.png}
	\caption{Diagrama de bloques del sistema.}
	\label{fig:d_bloques}
\end{figure}

El sistema embebido implementado en este trabajo consta de una PCB centralizadora, diseñada para integrar y gestionar todos los módulos de hardware. Esta arquitectura asegura la alimentación, adaptación y protección de todos sus componentes. El sistema completo abarca desde la adquisición de datos mediante sensores hasta las acciones sobre el entorno a través de actuadores. Además, se complementa con una interfaz de usuario móvil para el monitoreo y control remoto.

A continuación, se describen brevemente los bloques y su función.

\begin{itemize}
	\item Sensores: este bloque se encarga de la transducción de magnitudes físicas en señales eléctricas, lo que permite la digitalización de variables ambientales de interés para el control del cultivo.
	\item Adaptación de señal: los módulos de adaptación acondicionan las señales provenientes de los sensores. Para garantizar la correcta interpretación por parte de la unidad de microcontrolador o MCU (del inglés \textit{MicroController Unit}), ajustan los niveles de tensión y la relación señal-ruido a valores apropiados.
	\item MCU: el núcleo del sistema, basado en el chip ESP32, orquesta la comunicación y el control de todos los módulos. Provee la capacidad de procesamiento y la conectividad inalámbrica necesarias para la automatización del cultivo y la interacción con la aplicación móvil.
	\item Interfaces de salida: proporciona aislamiento galvánico y acondicionamiento de potencia para la activación de los actuadores, lo que asegura la protección del MCU y la correcta operación de los componentes de mayor potencia.
	\item Actuadores: los actuadores (ventiladores, bomba de irrigación, luces, resistencia calefactora) ejecutan las acciones de control y modifican las condiciones ambientales del cultivo según las necesidades.
	\item Aplicación móvil de usuario: desarrollada para facilitar la interacción con el sistema, la aplicación móvil permite el monitoreo en tiempo real de las condiciones del cultivo y el control remoto de los actuadores.
	\item Interfaz de servicio web: se implementó una interfaz para la transmisión de datos a un servicio web externo que permite el almacenamiento y análisis de información del cultivo. Esta funcionalidad se encuentra fuera del alcance principal de este trabajo.

\end{itemize}

\section{Arquitectura del firmware}

En la presente sección se aborda la arquitectura del firmware del microcontrolador.

\subsection{Patrones}

A continuación, se detallan los patrones de diseño arquitectónico utilizados.

\subsubsection{Arquitectura en capas}
Se adoptó un patrón de arquitectura en capas para estructurar el software desarrollado, lo que permitió una separación de funcionalidades clara mediante niveles de abstracción. Dicha metodología divide el sistema en niveles horizontales, cada uno con responsabilidades específicas y bien definidas, que facilita el desarrollo, la mantenibilidad y la escalabilidad del código.

A continuación, se enumeran las capas de abstracción que constituyen el firmware.

\begin{itemize}
	\item Capa de aplicación.
	\item Capa de sistema operativo.
	\item Capa de abstracción de hardware (HAL).
\end{itemize}


\subsubsection{Capa de abstracción de hardware}

Para facilitar la interacción con los diversos componentes de hardware y garantizar la portabilidad del código, se implementó una capa de abstracción basada en Espressif HAL (\textit{Hardware Abstraction Layer}). Integrada dentro del SDK de ESP-IDF, esta capa proporciona una interfaz uniforme para el control de los periféricos del ESP32, independientemente de las particularidades del hardware subyacente.

\subsubsection{Control ambiental}

El patrón de control ambiental se adoptó como estrategia arquitectónica para la capa de aplicación. El sistema embebido requirió la monitorización y modificación del entorno mediante sensores y actuadores. Este patrón permitió la estructuración de la lógica de control y facilitó la gestión de las interacciones entre los componentes de hardware y la implementación de los algoritmos de control.


\subsection{Componentes}

La capa de aplicación, figura \ref{fig:arq_bloques}, está constituida por los componentes de software encargados de gestionar cada una de las funcionalidades del sistema.
A continuación, se describe brevemente la funcionalidad de cada uno de estos componentes.

\begin{itemize}
	\item Monitor de temperatura: mide la temperatura de la solución nutritiva para garantizar que las plantas crezcan en un entorno térmicamente adecuado.
	\item Monitor de conductividad eléctrica (CE): evalúa la concentración de nutrientes presentes en la solución hidropónica.
	\item Monitor de nivel: monitorea que el nivel de solución nutritiva en el sistema no caiga por debajo de un valor determinado.
	\item Monitor de pH: mide el nivel de acidez de la solución nutritiva, lo que ayuda a mantener el pH dentro del rango óptimo para el cultivo.
	\item Monitor de humedad del sustrato: mide el contenido de humedad en el sustrato del cultivo.
	\item Monitor de luz: mide la cantidad de luz disponible en el entorno a través de un sensor.
	\item Servidor web embebido: el servidor web embebido permite la comunicación con el sistema de control del cultivo a través de la red. Este servidor permite monitorear, configurar y controlar el sistema desde cualquier dispositivo con acceso a la red.
	\item Control de ciclo de luz: este módulo se encarga de gestionar el ciclo de iluminación del cultivo. La configuración de dicho ciclo se realiza por medio de la interfaz de usuario.
	\item Control de ciclo de oxigenación/ventilación: este componente activa y desactiva el flujo de aire en el sistema de cultivo según la configuración del usuario.
	\item Control de hidratación: gestiona el riego, activa la bomba de agua y ajusta los niveles de humedad del sustrato según las mediciones del sensor de humedad. Esto asegura la cantidad adecuada de agua para el desarrollo de las plantas.
	\item Administrador: es el componente encargado de la centralización y validación global de los datos suministrados por los demás módulos de software.
\end{itemize}



\begin{figure}[h]
\centering
\includegraphics[scale=.4]{./Figures/arq_bloques.png}
	\caption{Diagrama de módulos funcionales y sus interacciones.}
	\label{fig:arq_bloques}
\end{figure}



\section{Desarrollo del software}

Para facilitar la escalabilidad de la aplicación, se implementó una tarea de FreeRTOS para cada componente del firmware. Esta arquitectura modular permite modificar cada componente de forma independiente, sin afectar al resto del sistema.

\subsection{Gestión de la comunicación entre tareas}
La comunicación entre tareas se gestiona mediante un patrón de publicación/suscripción con múltiples canales, que utiliza colas de FreeRTOS como buffers para el intercambio de datos. Esta arquitectura desacopla los productores de información (ej., monitores) de los consumidores (ej., controlador, administrador), lo que garantiza la integridad de los datos y la estabilidad del sistema al evitar condiciones de carrera.

\subsection{Gestión de la prioridad}
Durante el desarrollo del firmware, se abordó la organización de las tareas considerando el clásico problema de productores-consumidores. Esta estrategia permitió estructurar el sistema en función del flujo de información entre tareas.

Se definieron los monitores como tareas productoras, encargadas de obtener datos del entorno mediante sensores y publicarlos en estructuras compartidas .

Se implementaron los controladores como tareas consumidoras, cuya función es tomar decisiones en base a los datos recibidos y accionar los actuadores correspondientes.

Se desarrolló una tarea de administrador, también consumidora, responsable de coordinar el funcionamiento general del sistema y gestionar configuraciones.

El servidor embebido fue programado con un doble rol: responde a eventos generados por el usuario (como solicitudes de información o configuraciones) y también produce eventos que otras tareas consumen, como actualizaciones de parámetros.

Se asignaron diferentes niveles de prioridad a las tareas según su disponibilidad requerida. Las tareas de monitoreo se crearon con prioridad intermedia, mientras que las tareas de control y administración se configuraron con mayor prioridad para asegurar tiempos de respuesta adecuados ante eventos prioritarios. El servidor, al depender de la interacción del usuario, fue asignado con prioridad alta.

\subsection{Controladores de dispositivos}

Se desarrollaron controladores de dispositivos (\textit{device drivers}) dedicados a gestionar directamente las unidades de hardware involucradas en el sistema.

Estos controladores permiten abstraer las operaciones de bajo nivel necesarias para interactuar con sensores y actuadores.

\subsection{Gestión de fallos}
El sistema incorpora mecanismos de detección y manejo de fallos tanto a nivel de hardware como de software. Cada tarea cuenta con rutinas de supervisión que permiten identificar comportamientos anómalos o pérdida de respuesta de los dispositivos. En caso de error, se activan procedimientos de contingencia que buscan restablecer el funcionamiento sin comprometer la estabilidad del sistema general.

Por ejemplo, si un sensor deja de enviar datos válidos durante un tiempo determinado, se genera una alerta que es registrada y comunicada al usuario. De igual modo, si un actuador no responde a una orden tras múltiples intentos, se desactiva su uso y se notifica al sistema mediante una bandera de estado.

Además, se implementó un mecanismo de \textit{watchdog} que reinicia automáticamente el microcontrolador en caso de bloqueo o error crítico no recuperable. La configuración base del usuario se almacena en memoria no volátil (NVS), lo que permite que el sistema se recupere de cortes de energía sin perder los parámetros definidos.

\subsection{Servidor embebido}
Para facilitar la interacción con el usuario, el sistema cuenta con un servidor embebido basado en HTTP alojado en la propia ESP32. Este servidor permite exponer una API REST, a través de la cual se puede monitorear el estado del cultivo, consultar los valores de los sensores en tiempo real y ejecutar acciones como el encendido manual de actuadores o la configuración de alertas.

El servidor opera de forma concurrente con el resto de las tareas del sistema gracias a la arquitectura multitarea de FreeRTOS. Para garantizar la seguridad de acceso, se implementó un sistema básico de autenticación por usuario y contraseña que protege las rutas de la API.

El servidor responde a peticiones HTTP provenientes tanto de navegadores como de la aplicación móvil desarrollada en React Native, que primero transmite las credenciales de red por Bluetooth para iniciar la conexión Wi-Fi y luego se comunica mediante la API expuesta.

Este servidor embebido representa un componente clave para el monitoreo remoto y la escalabilidad del sistema, ya que permite una integración sencilla con otras plataformas IoT o bases de datos externas si se desea en el futuro.

\subsection{Lógica de negocio e interacción con la aplicación móvil}
La interacción entre la aplicación móvil y el dispositivo se desarrolla en dos etapas principales: configuración inicial por Bluetooth y comunicación operativa vía HTTP.

\subsubsection{Fase de configuración}
Al iniciar el sistema por primera vez o tras un reinicio de red, la aplicación móvil se conecta a la ESP32 mediante \textit{Bluetooth Low Energy} (BLE). A través de esta conexión, el usuario ingresa las credenciales de su red Wi-Fi doméstica (SSID y contraseña), que son transmitidas de forma segura al microcontrolador. Una vez recibidos estos datos, el dispositivo intenta conectarse a la red y, de ser exitoso, inicia el servidor embebido.

\subsubsection{Fase operativa}
Luego de establecer conexión Wi-Fi, la ESP32 expone una API REST que permite a la aplicación móvil interactuar con el sistema. Entre las operaciones más importantes se incluyen:

\begin{itemize}
	\item Lectura de datos ambientales (temperatura, humedad, pH, EC, etc.).
	\item Consulta del estado de los actuadores (iluminación, bomba, ventilación).
	\item Configuración de parámetros personalizados y umbrales de alerta.
	\item Activación manual o automática de funciones del sistema.
	\item Acceso a registros de eventos o errores del sistema.
\end{itemize}

\subsubsection{Manejo de estados y sincronización}
El firmware mantiene una estructura centralizada de datos en memoria que refleja el estado actual del sistema. Esta información se sincroniza con la app móvil cada vez que se establece una sesión HTTP válida. De este modo, el usuario siempre visualiza información actualizada del estado del cultivo.

Además, se contempló la posibilidad de incorporar una capa adicional de persistencia remota, como sincronización con bases de datos externas o servicios en la nube, lo que permitirá futuras extensiones del sistema hacia modelos más complejos de análisis de datos o inteligencia artificial.



	\include{Chapters/Chapter4} 
	\include{Chapters/Chapter5} 
\end{verbatim}

Los apéndices también deben escribirse en archivos .tex separados, que se deben ubicar dentro de la carpeta \emph{Appendices}. Los apéndices vienen comentados por defecto con el caracter \code{\%} y para incluirlos simplemente se debe eliminar dicho caracter.

Finalmente, se encuentra el código para incluir la bibliografía en el documento final.  Este código tampoco debe modificarse. La metodología para trabajar las referencias bibliográficas se desarrolla en la sección \ref{sec:biblio}.
%----------------------------------------------------------------------------------------

\section{Bibliografía}
\label{sec:biblio}

Las opciones de formato de la bibliografía se controlan a través del paquete de latex \option{biblatex} que se incluye en la memoria en el archivo memoria.tex.  Estas opciones determinan cómo se generan las citas bibliográficas en el cuerpo del documento y cómo se genera la bibliografía al final de la memoria.

En el preámbulo se puede encontrar el código que incluye el paquete biblatex, que no requiere ninguna modificación del usuario de la plantilla, y que contiene las siguientes opciones:

\begin{lstlisting}
\usepackage[backend=bibtex,
	natbib=true, 
	style=numeric, 
	sorting=none]
{biblatex}
\end{lstlisting}

En el archivo \file{reference.bib} se encuentran las referencias bibliográficas que se pueden citar en el documento.  Para incorporar una nueva cita al documento lo primero es agregarla en este archivo con todos los campos necesario.  Todas las entradas bibliográficas comienzan con $@$ y una palabra que define el formato de la entrada.  Para cada formato existen campos obligatorios que deben completarse. No importa el orden en que las entradas estén definidas en el archivo .bib.  Tampoco es importante el orden en que estén definidos los campos de una entrada bibliográfica. A continuación se muestran algunos ejemplos:

\begin{lstlisting}
@ARTICLE{ARTICLE:1,
    AUTHOR="John Doe",
    TITLE="Title",
    JOURNAL="Journal",
    YEAR="2017",
}
\end{lstlisting}


\begin{lstlisting}
@BOOK{BOOK:1,
    AUTHOR="John Doe",
    TITLE="The Book without Title",
    PUBLISHER="Dummy Publisher",
    YEAR="2100",
}
\end{lstlisting}


\begin{lstlisting}
@INBOOK{BOOK:2,
    AUTHOR="John Doe",
    TITLE="The Book without Title",
    PUBLISHER="Dummy Publisher",
    YEAR="2100",
    PAGES="100-200",
}
\end{lstlisting}


\begin{lstlisting}
@MISC{WEBSITE:1,
    HOWPUBLISHED = "\url{http://example.com}",
    AUTHOR = "Intel",
    TITLE = "Example Website",
    MONTH = "12",
    YEAR = "1988",
    URLDATE = {2012-11-26}
}
\end{lstlisting}

Se debe notar que los nombres \emph{ARTICLE:1}, \emph{BOOK:1}, \emph{BOOK:2} y \emph{WEBSITE:1} son nombres de fantasía que le sirve al autor del documento para identificar la entrada. En este sentido, se podrían reemplazar por cualquier otro nombre.  Tampoco es necesario poner : seguido de un número, en los ejemplos sólo se incluye como un posible estilo para identificar las entradas.

La entradas se citan en el documento con el comando: 

\begin{verbatim}
\citep{nombre_de_la_entrada}
\end{verbatim}

Y cuando se usan, se muestran así: \citep{ARTICLE:1}, \citep{BOOK:1}, \citep{BOOK:2}, \citep{WEBSITE:1}.  Notar cómo se conforma la sección Bibliografía al final del documento.

Finalmente y como se mencionó en la subsección \ref{subsec:configurando}, para actualizar las referencias bibliográficas tanto en la sección bibliografía como las citas en el cuerpo del documento, se deben ejecutar las herramientas de compilación PDFLaTeX, BibTeX, PDFLaTeX, PDFLaTeX, en ese orden.  Este procedimiento debería resolver cualquier mensaje "Citation xxxxx on page x undefined".
